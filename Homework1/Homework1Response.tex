% Author: Connor Baker
% Date Created: January 2, 2016
% Last Edited: January 18, 2016
% Version: 0.5c

% Declare type of document
\documentclass[10pt]{article}

% Import Packages
\usepackage[utf8]{inputenc}
\usepackage{amsfonts,amsmath,amssymb,amsthm}
\usepackage{mathtools,mathdots}
\usepackage{enumitem}
\usepackage{array}

% Page Formatting
% These settings let you manipulate the margins on the paper, and provide more options than you might be used to using in a word style document.  For example, the settings \oddsidemargin and \evensidemargin are allowed to be adjusted separately in case you are binding a book together.
\topmargin -0.25in \oddsidemargin -.25in \evensidemargin -.25in
\textheight 9in \textwidth 6.75in \headheight 0in \headsep .35in
\parindent 0in

% Define the basic math environments
\theoremstyle{definition}
\newtheorem{definition}[equation]{Definition}
\newtheorem{example}[equation]{Example}
\theoremstyle{plain}
\newtheorem{theorem}[equation]{Theorem}
\newtheorem{proposition}[equation]{Proposition}
\newtheorem{lemma}[equation]{Lemma}
\newtheorem{corollary}[equation]{Corollary}
\newtheorem{conjecture}[equation]{Conjecture}

% Define frequently used commands
\newcommand{\N}{\mathbb{N}}
\newcommand{\Z}{\mathbb{Z}}
\newcommand{\R}{\mathbb{R}}

% Begin the document
\begin{document}
% Create the Header
\begin{center}
  \subsection*{Homework 1\\Connor Baker, January 2017}
\end{center}

\begin{enumerate}
  % Problem 1
  \item Prove by contradiction that if $a-b$ is odd, then $a+b$ is odd.
\end{enumerate}

% Proof 1
\begin{proof}
  Assume that $a,b\in\Z$, $a-b$ is odd, and $a+b$ is even. If $a-b$ is odd, then by definition, $a-b=2x+1$ for some number $x\in\Z$. If $a+b$ is even, then by definition, $a+b=2y$ for some number $y\in\Z$. \\

  \par Combining the system of equations with addition yields $2a=2x+2y+1$. This can be rewritten as $2a = 2(x+y) + 1$. This implies that an even number (the product $2a$) can be equal to an odd number (the even numer resulting from the product of $2(x+y)$ plus one), which is a contradicition. \\

  \par Therefore, through by law of the excluded middle, if $a-b$ is odd, then $a+b$ is odd.
\end{proof}



\pagebreak



\begin{enumerate}
  % Problem 2
  \item[2.] Write a proof by contrapositive to show that if $xy$ is odd, then both $x$ and $y$ are odd.
\end{enumerate}

% Proof 2
\begin{proof}
  We will prove that if $x$ or $y$ is even, then the product $xy$ is even. Assume that $x$ is even, and that $x,y\in\Z$. By definition, $x=2k$ for all $k\in\Z$. Then $xy=2ky$ is even. So, regardless of the parity of $y$, the product $xy$ will be even so long as at least one is even. If $x$ was odd, and $y$ was even, then the above would still hold, due to multiplication being commutative. \\

  \par Since either $x$ or $y$ is even, and $xy$ is even, we can infer by the contrapositive that if $xy$ is odd, then both $x$ and $y$ are odd.
\end{proof}



\pagebreak



\begin{enumerate}
  % Problem 3
  \item[3.] Prove that there do not exist integers $m$ and $n$ such that $12m + 15n = 1$.
\end{enumerate}

% Proof 3
\begin{proof}
  The equation $12m + 15n = 1$ is equivalent to $3(4m+5n)=1$. For this statement to be true, $4m+5n$ must be the multiplicative inverse of $3$, which is not in the set of natural numbers. Therefore, there do not exist integers $m$ and $n$ such that $12m + 15n = 1$.
\end{proof}



\pagebreak



\begin{enumerate}
  % Problem 4
  \item[4.] Prove there is a natural number $M$ such that for every natural number $n$, $\frac{1}{n} < M$.
\end{enumerate}

% Proof 4
\begin{proof}
  Because $n\in\N$, $n\geq 1$ for all choices of $n$. As a result, $1/n \leq 1$, for all choices of $n$. Therefore, $M$ can be any number such that $M\geq 2$.
\end{proof}



\pagebreak



\begin{enumerate}
  % Problem 5
  \item[5.] Prove that if $-2 < x < 1$ or $x > 3$, then $\frac{(x-1)(x+2)}{(x-3)(x+4)} > 0$.
\end{enumerate}

% Proof 5
\begin{proof}
  Let the function $f(x)$ be as follows:
  \begin{equation*}
    f(x) = \frac{(x-1)(x+2)}{(x-3)(x+4)}
  \end{equation*}
  Then $f(x)$ has two $x$-intercepts at $x=-2,1$, and two vertical asymptotes at $x=-4,3$. By the Intermediate Value Theorem, those four $x$-values are the only places that the function can change the sign of its output. As such, it has been established that $f(x)$ does not change sign over the intervals $(-\infty,-4),(-4,-2),(-2,1),(1,3),(3,\infty)$. \\

  \par By picking a point on the intervals $(-2,1)$ and $(3,\infty)$ and verifying the sign, then by the Intermediate Value Theorem proves, the function value has the same sign on the entirety of the interval. \\

  \par Let $x=0$. Then, on the interval $(-2,1)$, the function is positive. \\

  \par Let $x=4$. Then, on the interval $(3,\infty)$, the function is positive. \\

  \par Therefore, by the Intermediate Value Theorem, if $-2 < x < 1$ or $x > 3$, then $\frac{(x-1)(x+2)}{(x-3)(x+4)} > 0$.
\end{proof}
\end{enumerate} % End problem set

\end{document} % End document
