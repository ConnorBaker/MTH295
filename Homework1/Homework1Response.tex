% Author: Connor Baker
% Date Created: January 2, 2016
% Last Edited: January 18, 2016
% Version: 0.4e

\documentclass[10pt]{article}
% Import Packages

\usepackage[utf8]{inputenc}
\usepackage{amsfonts,amsmath,amssymb,amsthm}
\usepackage{mathtools,mathdots}
\usepackage{enumitem}
\usepackage{array}

% Page Formatting
% These settings let you manipulate the margins on the paper, and provide more options than you might be used to using in a word style document.  For example, the settings \oddsidemargin and \evensidemargin are allowed to be adjusted separately in case you are binding a book together.
\topmargin -0.25in \oddsidemargin -.25in \evensidemargin -.25in
\textheight 9in \textwidth 6.75in \headheight 0in \headsep .35in
\parindent 0in

% Define the basic math environments
\theoremstyle{definition}
\newtheorem{definition}[equation]{Definition}
\newtheorem{example}[equation]{Example}
\theoremstyle{plain}
\newtheorem{theorem}[equation]{Theorem}
\newtheorem{proposition}[equation]{Proposition}
\newtheorem{lemma}[equation]{Lemma}
\newtheorem{corollary}[equation]{Corollary}
\newtheorem{conjecture}[equation]{Conjecture}

% Define frequently used commands
\newcommand{\N}{\mathbb{N}}
\newcommand{\Z}{\mathbb{Z}}
\newcommand{\R}{\mathbb{R}}

% Use a black square for QED
\renewcommand\qedsymbol{$\blacksquare$}

% Begin the document
\begin{document}
% Create the Header
\begin{center}
\subsection*{Homework 1\\Connor Baker, January 2017}
\end{center}

\begin{enumerate}
% Problem 1
\item Prove by contradiction that if $a-b$ is odd, then $a+b$ is odd.
  % Proof 1
  \begin{proof}
    Assume that $a,b\in\Z$, $a-b$ is even, and $a+b$ is odd. If $a-b$ is even, then by definition, $a-b=2x$ for some number $x\in\Z$. If $a+b$ is odd, then by definition, $a+b=2y+1$ for some number $y\in\Z$.

    \par Combining the system of equations with addition yields $2a=2x+2y+1$. By the definition of an even number, the product $2a$ will be even, as will the products $2x$ and $2y$. The sum of the two even products $2x$ and $2y$ is even. By definition, an even number plus one is odd. As a result, the equation is a false: an even integer cannot equal an odd integer.

    \par This is contradiction of our original assumption. Therefore, through proof by condtradiction, if $a-b$ is odd, then $a+b$ must odd.
  \end{proof}

% Problem 2
\item Write a proof by contrapositive to show that if $xy$ is odd, then both $x$ and $y$ are odd.
  % Proof 2
  \begin{proof}
    We will prove that if $x$ or $y$ is even, then the product $xy$ is even. Assume that $x$ is even, and that $x,y\in\Z$. By definition, $x=2k$ for all $k\in\Z$. Then $xy=2ky$, which, since $k,y\in\Z$, is by definition, even. So, regardless of the parity of $y$, the product $xy$ will be even so long as at least one is even. If $x$ was odd, and $y$ was even, then the above would still hold, due to multiplication being commutative.

    \par Since either $x$ or $y$ is even, and $xy$ is even, we can infer by the contrapositive that if $xy$ is odd, then both $x$ and $y$ are odd.
  \end{proof}

% Problem 3
\item Prove that there do not exist integers $m$ and $n$ such that $12m + 15n = 1$.
  % Proof 3
  \begin{proof}
    The equation $12m + 15n = 1$ is equivalent to $3(4m+5n)=1$. For this statement to be true, $4m+5n$ must be the multiplicative inverse of $3$, which is not in the set of natural numbers. Therefore, there do not exist integers $m$ and $n$ such that $12m + 15n = 1$.
  \end{proof}

% Problem 4
\item Prove there is a natural number $M$ such that for every natural number $n$, $\frac{1}{n} < M$.
  % Proof 4
  \begin{proof}
    $1/n \leq 1 \forall n\in\N$. As such, $M$ can be any number in the set of natural numbers that is larger than or equal to two

    \par Therefore, the first natural number $M$ larger than $1/n$ for all $n\geq 1$ is $2$.
  \end{proof}

% Problem 5
\item Prove that if $-2 < x < 1$ or $x > 3$, then $\frac{(x-1)(x+2)}{(x-3)(x+4)} > 0$.
  % Proof 5
  \begin{proof}
    The function has two $x$-intercepts, and two vertical asymptotes, at $x=-2,1$ and $x=-4,3$ respectively. By the Intermediate Value Theorem, those four $x$-values are the only places that the function can change the sign of its output.

    \par The sign of the function on the interval $(-\infty, -2)$ is not of concern: we care only about the interval $(-2,1)$ and $(3,\infty)$. By picking a point on the intervals $(-2,1)$ and $(3,\infty)$ and verifying the sign, the Intermediate Value Theorem proves that the function value has the same sign on the entirety of the interval.

    \par Let $x=0$. Then, on the interval $(-2,1)$, the function is positive.

    \par Let $x=4$. Then, on the interval $(3,\infty)$, the function is positive.

    \par Therefore, by the Intermediate Value Theorem, if $-2 < x < 1$ or $x > 3$, then $\frac{(x-1)(x+2)}{(x-3)(x+4)} > 0$.
  \end{proof}
\end{enumerate} % End problem set

\end{document} % End document
