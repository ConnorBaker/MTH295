% Author: Connor Baker
% Date Created: January 2, 2016
% Last Edited: January 27, 2018
% Version: 0.6a

% Declare type of document
\documentclass[10pt]{article}

% Import Packages
\usepackage[utf8]{inputenc}
\usepackage[mathscr]{euscript}
\usepackage{amsfonts,amsmath,amssymb,amsthm}
\usepackage{mathtools,mathdots}
\usepackage{enumitem}
\usepackage{array}
\usepackage{longtable}
\usepackage{geometry}
\geometry{left=1in,right=1in,top=1in,bottom=1in}
\usepackage{dirtytalk}
\usepackage[pdftex]{graphicx}
\usepackage{pgfplots}
\pgfplotsset{compat=1.11}
\usepackage{url}
\usepackage{caption}
\usepackage[inline]{asymptote}
\usepackage{lmodern}


% Define the basic math environments with both number and unnumbered environments
\theoremstyle{definition}
\newtheorem{definition}[equation]{Definition}
\newtheorem*{definition*}{Definition}
\newtheorem{example}[equation]{Example}
\newtheorem{problem}[equation]{Problem}
\newtheorem*{problem*}{Problem}
\newtheorem*{example*}{Example}
\newtheorem{axiom}[equation]{Axiom}
\newtheorem*{axiom*}{Axiom}
\theoremstyle{plain}
\newtheorem{theorem}[equation]{Theorem}
\newtheorem*{theorem*}{Theorem}
\newtheorem{proposition}[equation]{Proposition}
\newtheorem*{proposition*}{Proposition}
\newtheorem{lemma}[equation]{Lemma}
\newtheorem*{lemma*}{Lemma}
\newtheorem{corollary}[equation]{Corollary}
\newtheorem*{corollary*}{Corollary}
\newtheorem{conjecture}[equation]{Conjecture}
\newtheorem*{conjecture*}{Conjecture}
\newtheorem*{remark}{Remark}

% Define the basic math environments without numbering
% \theoremstyle{definition}
% \newtheorem*{definition}{Definition}
% \newtheorem*{example}{Example}
% \newtheorem*{problem*}{Problem}
% \newtheorem*{axiom}{Axiom}
% \newtheorem*{remark}{Remark}
% \theoremstyle{plain}
% \newtheorem*{theorem}{Theorem}
% \newtheorem*{proposition}{Proposition}
% \newtheorem*{lemma}{Lemma}
% \newtheorem*{corollary}{Corollary}
% \newtheorem*{conjecture}{Conjecture}

% Define frequently used commands
\newcommand{\N}{\mathbb{N}}
\newcommand{\Z}{\mathbb{Z}}
\newcommand{\Q}{\mathbb{Q}}
\newcommand{\R}{\mathbb{R}}
\newcommand{\C}{\mathbb{C}}
\DeclareMathOperator\dom{dom}
\DeclareMathOperator\rang{rang}
\newcommand{\ds}{\displaystyle}

% Defines the command for the quotient bar
\newcommand\Mydiv[2]{%
$\strut#1$\kern.25em\smash{\raise.3ex\hbox{$\big)$}}$\mkern-8mu
        \overline{\enspace\strut#2}$}

\makeatletter
\def\imod#1{\allowbreak\mkern10mu({\operator@font mod}\,\,#1)}
\makeatother

\Urlmuskip=0mu plus 2mu

\title{MTH 295: Homework 1}
\author{Connor Baker}
\date{January 2018}

% Begin the document
\begin{document}

\maketitle

\begin{enumerate}
  % Problem 1
  \item Prove by contradiction that if $a-b$ is odd, then $a+b$ is odd.
\end{enumerate}

% Proof 1
\begin{proof}
Assume that both $a$ and $b$ are integers. We proceed using proof by contradiction, and assume that $a-b$ is even implies that $a+b$ is odd.

Since $a-b$ is even, it can be written as 
\begin{equation}
    a-b = 2k
\end{equation}
for some integer $k$, and since $a+b$ is odd, it can be written as
\begin{equation}
    a+b = 2j+1
\end{equation}
for some integer $j$. Solving both equations for $a$ yields
\begin{subequations}
\begin{align}
    a &= 2k+b \\
    a &= 2j+1-b
\end{align}
and setting them equal to the other (by transitivity) gives:
\begin{equation}
    2k+b = 2j+1 - b.
\end{equation}
Simplifying by means of collecting $b$ on the left hand side and factoring out a two gives
\begin{equation}
    2(k+b) = 2j+1.
\end{equation}
\end{subequations}
Since an even number can never be equal to an odd number (by definition), we have arrived at a contradiction, and our original assumption that $a-b$ is even implies that $a+b$ is odd must be incorrect.

Therefore, by means of proof by contradiction, if $a-b$ is odd, then $a+b$ is odd.
\end{proof}



\pagebreak



\begin{enumerate}
  % Problem 2
  \item[2.] Write a proof by contrapositive to show that if $xy$ is odd, then both $x$ and $y$ are odd.
\end{enumerate}

% Proof 2
\setcounter{equation}{0}
\begin{proof}
Assume that $x$ and $y$ are integers. We proceed with proof by contrapositive, so we assume that if $x$ or $y$ are even, then $xy$ is even.

We begin with the case in which $x$ is even. Then, $x$ can be rewritten as $x=2k$ for some integer $k$. Then the product $xy$ can be written as
\begin{align}
    xy &= 2k\cdot y \\
       &= 2(ky)
\end{align}
which is even, by definition. Without loss of generality, the case in which $y$ is even is the same (simply swap the variables $x$ and $y$).

Finally, we have the case in which both $x$ and $y$ are even. In this case, $x$ and $y$ can be written
\begin{align}
    x &= 2k \\
    y &= 2j
\end{align}
for some integers $k$ and $j$. With these equivalencies, we can rewrite the product $xy$ as follows:
\begin{align}
    xy &= 2k\cdot 2j \\
       &= 2(2kj)
\end{align}
which is even, by definition.

Since either $x$ or $y$ being even implies $xy$ is even, we can infer by the contrapositive that if $xy$ is odd, then both $x$ and $y$ are odd.
\end{proof}



\pagebreak



\begin{enumerate}
  % Problem 3
  \item[3.] Prove that there do not exist integers $m$ and $n$ such that $12m + 15n = 1$.
\end{enumerate}

% Proof 3
\setcounter{equation}{0}
\begin{proof}
  The equation $12m + 15n = 1$ is equivalent to $3(4m+5n)=1$. For this statement to be true, $4m+5n$ must be the multiplicative inverse of $3$, which is not in the set of natural numbers. Therefore, there do not exist integers $m$ and $n$ such that $12m + 15n = 1$.
\end{proof}



\pagebreak



\begin{enumerate}
  % Problem 4
  \item[4.] Prove there is a natural number $M$ such that for every natural number $n$, $\frac{1}{n} < M$.
\end{enumerate}

% Proof 4
\setcounter{equation}{0}
\begin{proof}
  Because $n\in\N$, $n\geq 1$ for all choices of $n$. As a result, $1/n \leq 1$, for all choices of $n$. Therefore, $M$ can be any number such that $M\geq 2$.
\end{proof}



\pagebreak



\begin{enumerate}
  % Problem 5
  \item[5.] Prove that if $-2 < x < 1$ or $x > 3$, then $\frac{(x-1)(x+2)}{(x-3)(x+4)} > 0$.
\end{enumerate}

% Proof 5
\setcounter{equation}{0}
\begin{proof}
  Let the function $f(x)$ be as follows:
  \begin{equation}
    f(x) = \frac{(x-1)(x+2)}{(x-3)(x+4)}
  \end{equation}
Then $f(x)$ has two $x$-intercepts at $x=-2$ and at $x=1$, and two vertical asymptotes at $x=-4$ and $x=3$. By the Intermediate Value Theorem, those four $x$-values are the only places that the function can change sign. As such, it has been established that $f(x)$ does not change sign over the intervals 
$$
(-\infty,-4), (-4,-2), (-2,1), (1,3), (3,\infty).
$$
By picking a point on the intervals $(-2,1)$ and $(3,\infty)$ and verifying the sign, then by the Intermediate Value Theorem proves, the function value has the same sign on the entirety of the interval.

Let $x=0$. Then, on the interval $(-2,1)$, the function is positive.

Let $x=4$. Then, on the interval $(3,\infty)$, the function is positive.

Therefore, by the Intermediate Value Theorem, if $-2 < x < 1$ or $x > 3$, then $\frac{(x-1)(x+2)}{(x-3)(x+4)} > 0$.
\end{proof}

\end{document} % End document