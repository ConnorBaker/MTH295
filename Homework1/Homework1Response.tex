% Author: Connor Baker
% Date Created: January 2, 2016
% Last Edited: January 12, 2016
% Version: 0.3a

\documentclass[12pt]{article}
% Import Packages
\usepackage[utf8]{inputenc}
\usepackage[english]{babel}
\usepackage{amsfonts,amsmath,amssymb,amsthm}
\usepackage{mathtools}
\usepackage{enumitem}
\usepackage{array}
\usepackage{gensymb}
\usepackage{caption}
\usepackage{tocloft}
\usepackage[left=1.5in,right=1.5in,top=1.5in,bottom=1.5in]{geometry}

\begin{document}
% Create the Header
\begin{center}
\subsection*{Homework 1\\Connor Baker, January 2017}
\end{center}

% Problem 1
\begin{enumerate}
\item Prove by contradiction that if $a-b$ is odd, then $a+b$ is odd.
\end{enumerate}

% Proof
\begin{enumerate}
  \item[\textbf{Proof}] Assume that $a,b\in\mathbb{Z}$, $a-b$ is even, and $a+b$ is odd. If $a-b$ is even, then by definition, $a-b=2x$ for some number $x\in\mathbb{Z}$. If $a+b$ is odd, then by definition, $a+b=2y+1$ for some number $y\in\mathbb{Z}$.
  \item[] Combining the system of equations with addition yields $2a=2x+2y+1$. By the definition of an even number, the product $2a$ will be positive, as will the products $2x$ and $2y$. The sum of the two even products $2x$ and $2y$ is even. By definition, an even number plus one is odd. As a result, the equation is a false: an even integer cannot equal an odd integer.
  \item[] This is indicative of our original assumption that $a-b$ is even, and $a+b$ is odd being false. Therefore, through proof by condtradiction, if $a-b$ is odd, then $a+b$ must odd.
\end{enumerate}

% Problem 2
\begin{enumerate}
\item[2.] Write a proof by contrapositive to show that if $xy$ is odd, then both $x$ and $y$ are odd.
\end{enumerate}

% Proof
\begin{enumerate}
  \item[\textbf{Proof}] Assume that $x$ is even, and that $x,y\in\mathbb{Z}$. By definition, $x=2k$ for all $k\in\mathbb{Z}$. Then $xy=2ky$, which, since $k,y\in\mathbb{Z}$, is by definition, even. So, regardless of the parity of $y$, the product $xy$ will be even so long as the multiplier and or multiplicand is even. If $x$ was odd, and $y$ was even, then the above would still hold, due to multiplication being commutative.
  \item[] Since either $x$ or $y$ is even, and $xy$ is even, we can infer by the contrapositive that if $xy$ is odd, then both $x$ and $y$ are odd.
\end{enumerate}

% Problem 3
\begin{enumerate}
\item[3.] Prove that there do not exist integers $m$ and $n$ such that $12m + 15n = 1$.
\end{enumerate}

% Proof
\begin{enumerate}
  \item[\textbf{Proof}] The equation $12m + 15n = 1$ is equivalent to $3(4m+5n)=1$. For this statement to be true, $4m+5n$ must be the multiplicative inverse of $3$, which is not in the set of natural numbers. Therefore, there do not exist integers $m$ and $n$ such that $12m + 15n = 1$.
\end{enumerate}

% Problem 4
\begin{enumerate}
\item[4.] Prove there is a natural number $M$ such that for every natural number $n$, $\frac{1}{n} < M$.
\end{enumerate}

% Proof
\begin{enumerate}
  \item[\textbf{Proof}] Let $n=1$. Then $1/n=1$. As $n$ increases, the value of the ratio decreases since the top is constant. As such, since $n\in\mathbb{N}$, for all $n>2$, $1/n < 1$.
  \item[] Therefore, the first natural number $M$ larger than $1/n$ for all $n\geq 1$ is $2$.
\end{enumerate}

% Problem 5
\begin{enumerate}
\item[5.] Prove that if $-2 < x < 1$ or $x > 3$, then $\frac{(x-1)(x+2)}{(x-3)(x+4)} > 0$.
\end{enumerate}

% Proof
\begin{enumerate}
  \item[\textbf{Proof}]
\end{enumerate}

\end{document}
