% Author: Connor Baker
% Date Created: January 2, 2016
% Last Edited: January 11, 2016
% Version: 0.1b

\documentclass[12pt]{article}
% Import Packages
\usepackage[utf8]{inputenc}
\usepackage[english]{babel}
\usepackage{amsfonts,amsmath,amssymb,amsthm}
\usepackage{mathtools}
\usepackage{enumitem}
\usepackage{array}
\usepackage{gensymb}
\usepackage{caption}
\usepackage{tocloft}
\usepackage[left=1.5in,right=1.5in,top=1.5in,bottom=1.5in]{geometry}

\begin{document}
% Create the Header
\begin{center}
\subsection*{Homework 1\\Connor Baker, January 2017}
\end{center}

\begin{enumerate}
\item Prove by contradiction that if $a-b$ is odd, then $a+b$ is odd.
\end{enumerate}

% Assumptions
\begin{enumerate}
  \item[\textbf{Assumptions}] Assume that $a-b$ is even, and $a+b$ is odd.
\end{enumerate}

% Claims
\begin{enumerate}
  \item[\textbf{Claim 1}] Any even number less another even number is even. Any odd number less an odd number is even. Any even number less an odd number is odd. Any odd number less an even number is odd.
  \item[\textbf{Claim 2}] Any even number plus another even number is even. Any even number plus an odd number is odd (as is odd plus even because addition is commutative). Any odd number plus an odd number is even.
\end{enumerate}

% Lemmas
\begin{enumerate}
  \item[\textbf{Lemma 1}] Let $n,x,y \in\mathbb{N}$, $x=2n+1$, $y=2n$. For all $n$, $x$ is odd, and $y$ is even (any number with a factor of two is defined as even, and any even number plus one is defined as odd). Then, $x-x=0$, $x-y=1$, $y-x=-1$, and $y-y=0$. As such, an odd less an odd is even (will result in a multiple of two), an odd less an even is odd (will result in another odd number), an even less an odd is odd (will result in another odd number), and an even less an even is even (will result in a multiple of two).
  \item[\textbf{Lemma 2}] Let $n,x,y \in\mathbb{N}$, $x=2n+1$, $y=2n$. For all $n$, $x$ is odd, and $y$ is even (any number with a factor of two is defined as even, and any even number plus one is defined as odd). Then, $x+x=2x$, $x+y=4n+1=2y+1$, $y+x=4n+1=2y+1$, and $y+y=2y$. As such, an odd plus an odd is even (any number with a factor of two is even), an odd plus an even is odd (factor of two plus one is odd), an even plus an odd is the same because addition is commutative, and an even plus an even is even (any number with a factor of two is even).
\end{enumerate}

\begin{enumerate}
  \item[\textbf{Proof}] If $a-b$ is even, then either $a$ is even or odd, and $b$ is either even or odd. These vary inversely: if $a$ is even, $b$ must be odd, and vice versa.
  \item[] If $a+b$ is even, then either $a$ is even or odd, and $b$ is either even or odd. These vary directly: if $a$ is even, $b$ must be even, and vice versa.
  \item[] Here we see a contradiction: if $a$ and $b$ are even and odd respectively, then $a-b$ is even. However, this means that they are not varying directly: $a$ is even and $b$ is not even. They cannot satisfy both conditions simultaneously.
  \item[] Therefore, by contradiction, if $a-b$ is odd, $a+b$ is odd.
\end{enumerate}


\begin{enumerate}
\item[2.] Write a proof by contrapositive to show that if $xy$ is odd, then both $x$ and $y$ are odd.
\end{enumerate}

\begin{enumerate}
\item[3.] Prove that there do not exist integers $m$ and $n$ such that $12m + 15n = 1$.
\end{enumerate}

\begin{enumerate}
\item[4.] Prove there is a natural number $M$  such that for every natural number $n$, $\frac{1}{n} < M$.
\end{enumerate}

\begin{enumerate}
\item[5.] Prove that if $-2 < x < 1$ or $x > 3$, then $\frac{(x-1)(x+2)}{(x-3)(x+4)} > 0.$
\end{enumerate}

\end{document}
