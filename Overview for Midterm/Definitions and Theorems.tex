% Author: Connor Baker
% Date Created: March 9, 2017
% Last Edited: March 10, 2017
% Version: 0.2a

% Declare type of document
\documentclass[10pt]{article}

% Import Packages
\usepackage[utf8]{inputenc}
\usepackage{amsfonts,amsmath,amssymb,amsthm}
\usepackage{mathtools,mathdots}
\usepackage{enumitem}
\usepackage{array}

% Page Formatting
% These settings let you manipulate the margins on the paper, and provide more options than you might be used to using in a word style document.  For example, the settings \oddsidemargin and \evensidemargin are allowed to be adjusted separately in case you are binding a book together.
\topmargin -0.25in \oddsidemargin -.25in \evensidemargin -.25in
\textheight 9in \textwidth 6.75in \headheight 0in \headsep .35in
\parindent 0in

% Define the basic math environments
\theoremstyle{definition}
\newtheorem{definition}[equation]{Definition}
\newtheorem{example}[equation]{Example}
\newtheorem{theorem}[equation]{Theorem}
\newtheorem{proposition}[equation]{Proposition}
\newtheorem{lemma}[equation]{Lemma}
\newtheorem{corollary}[equation]{Corollary}
\newtheorem{conjecture}[equation]{Conjecture}

% Define frequently used commands
\newcommand{\N}{\mathbb{N}}
\newcommand{\Z}{\mathbb{Z}}
\newcommand{\R}{\mathbb{R}}
\DeclareMathOperator\dom{dom}

\makeatletter
\def\imod#1{\allowbreak\mkern10mu({\operator@font mod}\,\,#1)}
\makeatother

% Begin the document
\begin{document}
% Create the Header
\begin{center}
  \subsection*{Definitions and Theorems\\Connor Baker, March 2017}
\end{center}

\begin{definition}[Statement]
  Any sentence which can be evaluated as either true or false.
\end{definition}

\begin{definition}[Compound Statement]
  A statement made up of one or more component statements connected by logical connectors.
\end{definition}

\begin{definition}[Equivalence of Logical Operators]
  Two sets of logical operators are said to be equivalent if they produce the same output.
\end{definition}

\begin{definition}[Tautology]
  A statement that's always true.
\end{definition}

\begin{definition}[Contradiction]
  A statement that's always false.
\end{definition}

\begin{definition}[A Set]
  Any collection of objects.
\end{definition}

\begin{definition}[Set Builder Notation]
  \{expression: rule\}
\end{definition}

\begin{definition}[Universal Set]
  The given or implied set that contains all other sets in the problem. This set fixes Russel's Paradox.
\end{definition}

\begin{definition}[Tautology]
  A statement that's always true.
\end{definition}

\begin{definition}[Natural Numbers]
  The set $\N: \{1,2,3, \dots \}$.
\end{definition}

\begin{definition}[Integers]
  The set $\Z: \{\dots, -1,0,2, \dots \}$.
\end{definition}

\begin{definition}[Rational Numbers]
  The set $\mathbb{Q}: \{\frac{a}{b}: a\in\Z$ and $b\in\N \}$.
\end{definition}


\begin{definition}[Real Numbers]
  The set $\R: \{a_n a_{n-1}\dots a_1a_0a_{-1}a_{-2} \dots:n\in\N\cup\{ 0\}$ and $a_i \in \{ 0,\dots,9\} \}$.
\end{definition}

\begin{definition}[Complex Numbers]
  The set $\mathbb{C}: \{a+bi:i^2 = -1$ and $a,b \in\R\}$.
\end{definition}

\begin{definition}[Subset]
  Given two sets $A$ and $B$, $A\subseteq B \iff \forall a\in A \implies a\in B$.
\end{definition}


\begin{definition}[Open Sentence (AKA Predicate)]
  A statement that contains a variable. The truth value depends on the variable.
\end{definition}

\begin{definition}[Truth Set]
  The set of values that make the statement true.
\end{definition}

\begin{definition}[Quantifiers and Negations]
  Logical Quantifiers and Negators:
  \begin{enumerate}
    \item Universal Quantifier: $\forall$ -- Must be true for all $x$ in the universal set such that  $P(x)$ is true: $(\forall x)(P(x))$.
    \item Existential Quantifier: $\exists$ -- True if for at least one $x$ in the universal set such that $P(x)$ is true: $(\exists x)(P(x))$.
    \item Unique Quantifier: $\exists!$ -- True if there exists only one $x$ in the universal set such that $P(x)$ is true: $(\exists! x)(P(x))$.
    \item Negation of the Universal Quantifier: $\sim(\forall x)(P(x))$ is $(\exists x)(\sim P(x))$.
    \item Negation of the Existential Quantifier: $\sim(\exists x)(P(x))$ is $(\forall x)(\sim P(x))$.
  \end{enumerate}
\end{definition}

\begin{definition}[Direct Proof]
  $P\implies Q$.
\end{definition}

\begin{definition}[Contrapositive Proof]
  $(\sim Q) \implies (\sim P)$.
\end{definition}

\begin{definition}[Proof by Contradiction]
  We start with $P \implies Q$. Assume that $\sim P \land Q$ is true. Then $\sim P \implies A_1 \implies A_2 \implies \dots \implies R$. And, if $Q \implies B_1 \implies B_2 \implies \dots \implies \sim R$. Then,$\sim R \land R$ must be true, which is a contradiction, so the original assumption is false, and $P \implies Q$.
\end{definition}

\begin{definition}[Axioms of the Natural Numbers]
  The following are axioms for the set of the Natural Numbers, $\N$:
  \begin{enumerate}
    \item Successor property
    \begin{enumerate}
      \item One is a natural number
      \item One is not the successor of any number
      \item Every natural number has a unique successor
    \end{enumerate}
    \item Closure under addition and multiplication
    \item Associativity
    \item Commutativity
    \item Distribution of multiplication over addition
    \item Cancellation
    \begin{enumerate}
      \item Real numbers have this property unless the number being cancelled is a zero
      \item Matrix multiplication does not have this property
    \end{enumerate}
  \end{enumerate}
\end{definition}

\begin{definition}[Divisible]
  Let $a,b\in\N$. Then $a|b$ if $\exists k\in\N:ak=b$.
\end{definition}

\begin{definition}[Prime]
    A number $p$, where $p\in\N$, is prime if $p>1$ and its only divisors are one and itself.
\end{definition}

\begin{definition}[Factor]
  A number $q$, where $q\in\N$, is a factor of $r$ if $q|r$.
\end{definition}

\begin{definition}[Prime Factor Decomposition]
  Let $p_1,p_2,\dots,p_k$ be all primes less than $q$. Then, the prime factor decomposition of $q$ is $p_1^{n_1} p_2^{n_2},\dots, p_k^{n_k}$ where $n_i\in(\N\cup \{0\})$.
\end{definition}

\begin{theorem}[Fundamental Theorem of Arithmetic]
  All natural numbers have a unique prime factorization up to commutativity.
\end{theorem}

\begin{definition}[Union over $\mathcal{A}$]
  Let $\mathcal{A}$ be a family of sets. The union over $\mathcal{A}$ is defined as:
  $$\bigcup_{A\in\mathcal{A}} = \{x: (\exists A\in\mathcal{A})(x\in A)$$
  which is equivalent to:
  $$\bigcup_{A\in\mathcal{A}} = \{x: (\exists A)((A\in\mathcal{A})\land(x\in A))$$
\end{definition}

\begin{definition}[Intersection over $\mathcal{A}$]
  Let $\mathcal{A}$ be a family of sets. The intersection over $\mathcal{A}$ is defined as:
  $$\bigcap_{A\in\mathcal{A}} = \{x: (\forall A\in\mathcal{A})(x\in A)$$
  which is equivalent to:
  $$\bigcap_{A\in\mathcal{A}} = \{x: (\forall A)((A\in\mathcal{A})\implies(x\in A))$$
\end{definition}

\begin{theorem}[Relative Cardinality of Intersection and Union]
  For every set $B\in\mathcal{A}$:
    $$B\subseteq \bigcup_{A\in\mathcal{A}} A,$$
    $$\bigcap_{A\in\mathcal{A}} A\subseteq B.$$
  The intersection is no bigger than the smallest set, and the union is no smaller than the biggest set.

  Assume that $\mathcal{A} \neq \emptyset$. Then:
    $$\bigcap_{A\in\mathcal{A}} A\subseteq \bigcup_{A\in\mathcal{A}} A.$$
  If $\mathcal{A} \neq \emptyset$, the union isn't a problem but the intersection would be the set of all sets, and as such is undefined.
\end{theorem}

\begin{definition}[Indexed Family of Sets]
  Let $\Delta$ be a nonempty set. Then, $\forall\alpha\in\Delta$, there is a corresponding set $A_\alpha$. The family of sets $\mathcal{A} = \{A_\alpha: \alpha\in\Delta\}$.
\end{definition}

\begin{definition}[Union and Intersection over an Indexed Family of Sets $\mathcal{A}$]
  Let $\mathcal{A}$ be a family of sets with indicies $\alpha\in\Delta$. Then, the union over $A_\alpha$ is defined as:
  $$\bigcap_{\alpha\in\Delta} A_\alpha = \{x: (\exists\alpha\in\Delta)(x\in A_\alpha)$$
  and the intersection is defined as:
  $$\bigcup_{\alpha\in\Delta} A_\alpha = \{x: (\forall\alpha\in\Delta)(x\in A_\alpha)$$
\end{definition}

\begin{theorem}[Relative Cardinality of Intersection and Union over Indexed Family of Sets]
  For every set $\beta\in\Delta$:
    $$A_\beta \subseteq \bigcup_{\alpha\in\Delta} A_\alpha,$$
    $$\bigcap_{\alpha\in\Delta} A_\alpha \subseteq A_\beta.$$
    $$\overline{\bigcup_{\alpha\in\Delta} A_\alpha} = \bigcap_{\alpha\in\Delta} \overline{A_\alpha}$$
    $$\overline{\bigcap_{\alpha\in\Delta} A_\alpha} = \bigcup_{\alpha\in\Delta} \overline{A_\alpha}$$
\end{theorem}

\begin{definition}[Pairwise Disjoint]
  Let $\mathcal{A} = \{A_\alpha : \alpha\in\Delta\}$. Then $\mathcal{A}$ is pairwise disjoint if $\forall\alpha,\beta\in\Delta$ with $A_\alpha\neq A_\beta$, $A_\alpha \cap A_\beta = \emptyset$.
\end{definition}

\begin{theorem}[Order Properties of the Natural Numbers]
  Let $x,y,z\in\N$. Then, $\forall x,y,z$:
  \begin{enumerate}
    \item $x<y \iff \exists w\in\N: x+w=y$
    \item $x\leq y \iff x=y or x<y$
    \item if $x<y$ and $y<z$, then $x<z$ (transitivity)
    \item if $x\leq y$ and $y\leq x$, then $x=y$
    \item if $x<y$, then $x+z < y+z$ and $xz<yz$
  \end{enumerate}
\end{theorem}

\begin{theorem}[Principle of Mathematical Induction (PMI)]
  If $S$ is any subset of the natural numbers, with the properties that:
  \begin{enumerate}
    \item $1\in S$
    \item if $k\in S$, then $(k+1)\in S$
  \end{enumerate}
  then $S=\N$.

  The general process of mathematical induction is as follows:
  \begin{enumerate}
    \item Define $S=\{n\in\N:$ some statement is true$\}$
    \begin{enumerate}
      \item Prove that the basis case holds: that means that $1\in S$
      \item Assume $k\in S$. Then, based on this assumption, prove it to be the case that $(k+1)\in S$.
      \item Conclude that by the Principle of Mathematical Induction, $S=\N$.
    \end{enumerate}
  \end{enumerate}
\end{theorem}

\begin{definition}[Inductive Set]
  A set $S\subseteq\N$ is inductive if whenever $n\in S$, then $(n+1)\in S$.
\end{definition}

\begin{definition}[Factorial]
  If $n\in\N$, then $n! = n(n-1)!$.
\end{definition}

\begin{definition}[Zero Factorial]
  $0! = 1$.
\end{definition}

\begin{definition}[General Principle of Mathematical Induction]
  $S\subseteq\N$ where $k\in S$ and if $j\in S$, then $(j+1)\in S$, and it is true for all $\{k, k+1, \dots \}$, then $S$ is inductive.
\end{definition}

\begin{theorem}[Principle of Strong Mathematical Induction (PSMI)]
    If $S\subseteq\N$ with the property that $\forall m\in\N$, if $\{1,2,\dots,m-1\}\subseteq S$, then $m\subseteq S$, then $S=\N$.

    PSMI is different from PMI because with PMI we assume that we can start at a value and carrying forward from that value something holds. With PSMI, we assume that it holds over an interval.
\end{theorem}

\begin{theorem}[Well Ordering Principle (WOP)]
  Every nonempty subset of $\N$ has a least element.
\end{theorem}

\begin{theorem}[The Division Algorithm]
  Let $a,b\in\N$, with $b\leq a$. Then we will prove that $\exists q\in \N$ and $r\in\N\cup\{0\}:a=bq+r$ where $0\leq r < b$.

  Consider all multiples of $b>a$. Let $S=\{s\in\N:sb>a\}$. By (WOP), $S$ has a least element $q+1$, so $q\not\in S$. Therefore, $qb\leq a$.

  Let $r=q-qb$. Since $qb\leq a$, $a-qb \geq 0$, so it must be the case that $r\geq 0$.

  If $r\geq b$, then $r=q-qb \geq b \implies a-qb-b \geq 0 \implies q-b(q+1) \geq 0$. So, $a\geq b(q+1)$. But, for $(q+1)\in S$, it must be that $b(q+1) > a$. Then, $(q+1)\in S$. This is a contradiction. Therefore, $r<b$.

  Furthermore, $q$ and $r$ are unique.

  Assume $\exists q_1,r_1$ with $a=bq_1 + r_1$ where $0\leq r_1 < b$. Then $a=bq+r$, $a=bq_1+r_1$. This implies that $0=b(q-q_1) + (r-r_1), b\neq 0$. If it is the case that $q-q_1\neq 0$, then $|q-q_1|\in\N$. Then $r_1 > r$ and $|r_1-r|=mb$ for some $m\in\N, m=|q-q_1|$. Thus, $r_1 \geq mb\implies r_1\geq b$, which is a contradition such that $q-q_1$ would be zero and $r_1-r=0$.
\end{theorem}

\begin{definition}[Greatest Common Divisor (GCD)]
  For $a,b\in\N$, the GCD of $a$ and $b$ can be written as the linear combination of $a$ and $b$ -- that is, if:

  $$d=\text{GCD}(a,b), \exists x,y\in\Z: xa+yb=d$$
\end{definition}

\begin{definition}[Ordered Pair]
  A set whose order matters. $(x,y) = \{x,\{x,y\}\}$.
  The reason the ordered pair translates to this is because $x$ is the first coordinate, since it is an element at every level of the set.
\end{definition}

\begin{definition}[n-tuples]
  A set of n-tuples $(x_1,x_2,\dots,x_n)$ can be re-written as a set like so: \\ $\{x_1,\{x_1,x_2\},\{x_1,\{x_1,x_2\},\{x_1,x_2,x_3\}\},\dots, \{x_1,\{x_1,x_2\},\dots, \{x_1,x_2,\dots, x_n\}\}\}$.
\end{definition}

\begin{definition}[Cartesian Product]
  let $A,B$ be sets. Then, the Cartesian product $A\times B$ is the set:
  $$A\times B = \{(a,b): (a\in A) \land (b\in B)\}$$
  In general, $A\times B \neq B\times A$.
  While it might be tempting, note that (again, in general) $A\times B\times C \neq (A\times B) \times C \neq A\times (B\times C)$.
\end{definition}

\begin{theorem}[Properties of the Cartesian Product]
  Let $A,B,C,D$ be sets. Then:
  \begin{enumerate}
    \item $A\times(B\cup C) = (A\times B)\cup(A\times C)$
    \item $A\times(B\cap C) = (A\times B)\cap(A\times C)$
    \item $A\times \emptyset = \emptyset$
    \item $(A\times B)\cap(C\times D) = (A\cap C)\times(B\cap D)$
    \item $(A\times B)\cup(C\times D) \subseteq (A\cup C)\times(B\cup D)$
    \item $(A\times B)\cap(B\times A) = (A\cap B)\times(A\cap B)$
  \end{enumerate}
\end{theorem}

\begin{definition}[Relation]
  A relation $R$ from a set $A$ to a set $B$ is any subset of $A\times B$.
  If $a\in A$, and $b\in B$, then:
  \begin{enumerate}
    \item $aRb \iff (a,b)\in R$
    \item $a\not{R}b \iff (a,b)\not\in R$
  \end{enumerate}
\end{definition}

\begin{definition}[Domain]
  The domain of a relation $R$ from a set $A$ to a set $B$ $(R: A\rightarrow B)$ is $\dom(R) = \{x\in A: \exists y\in B : xRy\}$.
\end{definition}

\begin{definition}[Domain]
  The range of a relation $R: A\rightarrow B)$ is rang$(R) = \{y\in B: \exists x\in A : xRy\}$.
\end{definition}

\begin{definition}[Identity Relation]
  Let $A$ be any set. Then:
  $$I_A = \{(x,x): x\in A\}$$
\end{definition}

\begin{definition}[Inverse Relation]
  Let $R:A\rightarrow B$. Then $R^{-1} = \{(y,x): xRy\}$.
\end{definition}

\begin{theorem}[Range and Domain of Inverse Relation]
  Let $R:A\rightarrow B$. Then:
  \begin{enumerate}
    \item $\dom(R) = \text{rang}(R^{-1})$
    \item $\dom(R^{-1}) = \text{rang}(R)$
  \end{enumerate}
\end{theorem}

\begin{definition}[Composition of Relations]
  Let $R:A\rightarrow B$, and $S:B\rightarrow BC$. Then:
  \begin{enumerate}
    \item $S\circ R: A\rightarrow C$
    \item $S\circ R=\{(x,z): \exists y\in B: (xRy) \land (ySz)$
  \end{enumerate}
\end{definition}

\begin{theorem}[Properties of Relations]
  Assuming that all compositions are well defined:
  \begin{enumerate}
    \item $(R^{-1})^{-1} = R$
    \item $T\circ (S\circ R) = (T\circ S)\circ R$ (associativity)
    \item $I_B \circ R = R \circ I_A = R$ (identity relation)
    \item $(S\circ R)^{-1} = R^{-1} \circ S^{-1}$
  \end{enumerate}
\end{theorem}

\begin{definition}[Unary Operator]
  Requires only a single argument.
\end{definition}

\begin{definition}[Binary Operator]
  Requires two arguments.
\end{definition}

\begin{definition}[Reflexive Relation]
  A relation $R$ on $A$ is reflexive if, $\forall x\in A, xRx$.
\end{definition}

\begin{definition}[Symmetric Relation]
  A relation $R$ on $A$ is symmetric if $xRy\implies xRx$.
\end{definition}

\begin{definition}[Transitive Relation]
  A relation $R$ on $A$ is transitive if $((xRy)\land(yRz))\implies xRz$.
\end{definition}

\begin{definition}[Equivalence Relation]
  A relation $R$ on $A$ is an equivalence relation if $R$ is reflexive, symmetric, and transitive.
\end{definition}

\begin{definition}[Equivalence Class]
  Given $R$, a equivalence relation on $A$, for any $x\in A$, the equivalence class of $x$, denoted $[x]$, is the set $\{y\in A: xRy\}$.
\end{definition}

\begin{definition}[Partition]
  Let $A$ be a set. Then a partition of $AS$ is a collection of sets $\mathcal{X}$, which is a subset of $\mathcal{P}(A)$ with the following conditions.
  \begin{enumerate}
    \item $\emptyset\not\in\mathcal{X}$
    \item $\forall\mathcal{C,D}\in\mathcal{X}, \mathcal{C}\cap\mathcal{D}=\emptyset,$ when $\mathcal{C}\neq\mathcal{D}$ (piece-wise disjoint)
    \item $\cap_{\mathcal{C}\in\mathcal{X}} \mathcal{C} = A$
  \end{enumerate}

  For every equivalence class of a set, they form a partition. Conversely, equivalence classes and partitions generate the other.
\end{definition}

\begin{theorem}[Properties of Equivalence Relation]
  Suppose that $R$ is an equivalence relation on $A$, $A\neq\emptyset$ (which means that $R\neq\emptyset$, since everything at least relates to itself). Then:
  \begin{enumerate}
    \item $\forall x\in A, x\in[x]$ (this implies $[x]\neq \emptyset$)
    \item $[x] \subseteq A, \forall x\in A$ (the range$(R)\subseteq A$, so $[x]\subseteq A$)
    \item $xRy \iff [x]=[y]$
    \item $\cup_{x\in A} [x] = A$
    \item $x\not Ry \iff [x]\cap [y] = \emptyset$
    \item The set of all equivalence relations (A quotient R) $A/R = \{[x]:x\in A\}$
  \end{enumerate}
\end{theorem}

\begin{theorem}[Partitions on Non-empty Sets]
  Let $P$ be a partition of a non-empty set $A$. Then, $\exists$ an equivalence relation $R$ on $A$ where $A/R=P$.
\end{theorem}

\begin{definition}[Comparability]
  $R$ has the property of comparability if $\forall x,y\in A$, either $xRy$ or $yRx$. One example of a relation with comparability is the less than or greater than relation. An example of a relation that does not have comparability is the divides relation (two does not divide three, 3 does not divide two, but two and three are both in in the set $A$).
\end{definition}

\begin{definition}[Bounded Set]
  A set $A\subseteq\R$ of real numbers is bounded from above if there exists a real number $M\in\R$, called an upper bound of $A$, such that $x\leq M, \forall x\in A$. Similarly, $A$ is bounded from below if there exists $m\in\R$, called a lower bound of $A$ ,such that $x\geq m,\forall x\in A$. A set is bounded if it is bounded both from above and below.
\end{definition}

\begin{definition}[Supremum]
  The supremum of a set is its least upper bound. Suppose that $A\subseteq\R$ is a set of real numbers. If $M\in\R$ is an upper bound of $A$ such that $M\leq M′$ for every upper bound $M′$ of $A$, then $M$ is called the supremum of $A$, denoted $M = $sup$(A)$.
\end{definition}

\begin{definition}[Infemum]
  The infemum of a set is its greatest upper bound. Suppose that $A\subseteq\R$ is a set of real numbers. If $m\in\R$ is a lower bound of $A$ such that $m\geq m′$ for every lower bound $m′$ of $A$, then $m$ is called the infemum of $A$, denoted $m = $inf$(A)$.
\end{definition}

\begin{theorem}[Uniqueness of Least Upper Bound and Greatest Lower Bound]
  If $R$ is a partial order on $A$, and $B$ is a subset of $A$, then if there is a least upper bound, or greatest lower bound for $B$ in $A$, then it is unique.
\end{theorem}

\begin{definition}[Lienar Order]
  A partial order $R$ on $A$ is a linear order (or total order) if $\forall a,b\in A$, either $aRb$ or $bRa$. The Hesse Diagram of a linear order relationship looks like a line. One such example of a linear order relationship is less than or equal to.
\end{definition}

\begin{definition}[Well Ordering]
  A linear ordering $R$ on $A$ is well ordered if every nonempty subset $B$ of $A$ has a least element in $B$. Being a well ordered set is super non-trivial.
  Let $A=[0,1], A\subseteq\R$:
  \begin{enumerate}
    \item $B=(0,1)\subseteq A$
    \item inf$(B) = 0 \not\in B$
  \end{enumerate}
  so $B$ is not well ordered.
\end{definition}

\end{document} % End document
