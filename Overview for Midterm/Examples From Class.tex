% Author: Connor Baker
% Date Created: March 11, 2017
% Last Edited: March 11, 2017
% Version: 0.1a

% Declare type of document
\documentclass[10pt]{article}

% Import Packages
\usepackage[utf8]{inputenc}
\usepackage{amsfonts,amsmath,amssymb,amsthm}
\usepackage{mathtools,mathdots}
\usepackage{enumitem}
\usepackage{array}

% Page Formatting
% These settings let you manipulate the margins on the paper, and provide more options than you might be used to using in a word style document.  For example, the settings \oddsidemargin and \evensidemargin are allowed to be adjusted separately in case you are binding a book together.
\topmargin -0.25in \oddsidemargin -.25in \evensidemargin -.25in
\textheight 9in \textwidth 6.75in \headheight 0in \headsep .35in
\parindent 0in

% Define the basic math environments
\theoremstyle{definition}
\newtheorem{definition}[equation]{Definition}
\newtheorem{example}[equation]{Example}
\newtheorem{theorem}[equation]{Theorem}
\newtheorem{proposition}[equation]{Proposition}
\newtheorem{lemma}[equation]{Lemma}
\newtheorem{corollary}[equation]{Corollary}
\newtheorem{conjecture}[equation]{Conjecture}

% Define frequently used commands
\newcommand{\N}{\mathbb{N}}
\newcommand{\Z}{\mathbb{Z}}
\newcommand{\R}{\mathbb{R}}
\DeclareMathOperator\dom{dom}

\makeatletter
\def\imod#1{\allowbreak\mkern10mu({\operator@font mod}\,\,#1)}
\makeatother

% Begin the document
\begin{document}
% Create the Header
\begin{center}
  \subsection*{Examples from Class\\Connor Baker, March 2017}
\end{center}

\begin{example}[Prove that if $(3|a)\land (3|b)$ then $9|(ab)$]
  Assume that $\exists k,j\in\N:3k=a,3j=b$. Then, $ab=9kj$. Since $k,j\in\N$, and $(kj)\in\N$, then $(9kj)\in\N$. By the definition of divisibility, $9|(ab)$.
\end{example}

\begin{example}[Let $m,n\in\N$ and $q$ prime. Then $q|m\iff q|m^2$.]
  If $q|m$, then $q|m^2$. Then $\exists k\in\N:qk=m$. Then, $m^2=q^2 k^2$. By definition, since it has the same factor twice, $q|m^2$.

  If $q|m^2$, then $q|m$. Let the unique prime factor decomposition of $m=p_1^{n_1}\cdot p_2^{n_2}\cdots p_k^{n_k}$. Then $m^2 = p_1^{2n_1}\cdot p_2^{2n_2}\cdots p_k^{2n_k}$. Since $q|m^2,q=p_i^{2n_i}$ for some $i\in\N,i\leq k$. Since $q$ is prime and is in $\N,2n_i \geq 2 \implies n_i \geq 1$. Furthermore, $q$ must be in the unique prime factorization of $m$ (which we can infer from $q$'s being prime and a factor of $m^2$ -- it must have a factor of at least $q^2$). As such, $q|m$.

  Therefore, $q|m\iff q|m^2$.
\end{example}

\begin{example}[$\sqrt{2}$ is irrational]
  If $(x>0)\land (x^2=2)$, then $x$ is irrational. We will prove by contradiction that $x$ is irrational.

  Assume that $x$ is rational and $x>0,x^2=2$. Then, since $x$ is rational, $\exists m,n\in\N:x=\frac{m}{n}$, and $m,n$ have no common factors. As such, $2=x^2=\frac{m^2}{n^2}\implies m^2=2n^2$, and $2|m^2$, which by the previous example, means $2|m$.

  Since $2|m,\exists k\in\N$ where $m=2k$. As such, $x=\frac{2k}{n}\implies x^2 = \frac{4k^2}{n^2} = 2 \implies 2k^2 = n^2$.

  Therfore, $2|n^2\implies2|n$.

  So, $m,n$ both have no factors in common, yet they have a factor of two, which is a contradiction. Therefore, it must the case that if $(x>0)\land (x^2=2)$, then $x$ is irrational.
\end{example}


\end{document} % End document
