% Author: Connor Baker
% Date Created: April 18, 2017
% Last Edited: April 19, 2017
% Version: 0.1b

% Declare type of document
\documentclass[10pt]{article}

% Import Packages
\usepackage[utf8]{inputenc}
\usepackage[mathscr]{euscript}
\usepackage{amsfonts,amsmath,amssymb,amsthm}
\usepackage{mathtools,mathdots}
\usepackage{enumitem}
\usepackage{array}
\usepackage{longtable}

% Page Formatting
% These settings let you manipulate the margins on the paper, and provide more options than you might be used to using in a word style document.  For example, the settings \oddsidemargin and \evensidemargin are allowed to be adjusted separately in case you are binding a book together.
\topmargin -0.25in \oddsidemargin -.25in \evensidemargin -.25in
\textheight 9in \textwidth 6.75in \headheight 0in \headsep .35in
\parindent 0in

% Define the basic math environments
\theoremstyle{definition}
\newtheorem{definition}[equation]{Definition}
\newtheorem{example}[equation]{Example}
\theoremstyle{plain}
\newtheorem{theorem}[equation]{Theorem}
\newtheorem{proposition}[equation]{Proposition}
\newtheorem{lemma}[equation]{Lemma}
\newtheorem{corollary}[equation]{Corollary}
\newtheorem{conjecture}[equation]{Conjecture}

% Define frequently used commands
\newcommand{\N}{\mathbb{N}}
\newcommand{\Z}{\mathbb{Z}}
\newcommand{\Q}{\mathbb{Q}}
\newcommand{\R}{\mathbb{R}}
\newcommand{\C}{\mathbb{C}}
\DeclareMathOperator\dom{dom}
\DeclareMathOperator\rang{rang}
\newcommand{\ds}{\displaystyle}

\makeatletter
\def\imod#1{\allowbreak\mkern10mu({\operator@font mod}\,\,#1)}
\makeatother

\begin{document} %This is where we type the text that we plan on having in our document.

% Create the Header
\begin{center}
  \subsection*{Homework 6\\Connor Baker, April 2017}
\end{center}

\begin{enumerate}
\item An algebraic number is any number that is a root of a polynomial with rational coefficients.  Prove that the algebraic numbers are countable.  A number is transcendental if it is not algebraic.  Prove there are uncountable many transcendental numbers.
\end{enumerate}

\begin{proof}
  Let $P_n$ be the set of all polynomials that have rational coefficients, and variables raised to nonzero, whole exponents of (at most) degree $n$:
  $$P_n = \{a_k x^n + a_{k-1} x^{n-1} + \dots + a_1 x^1 + a_0\}: a_i \in\Q\}.$$

  Then, since all polynomials are functions on $\R$, $P_n$ is a set of functions (which all have solutions). Let $R_n$ be the set of all roots of $P_n$:
  $$R_n= \{z\in\C: \exists f\in P_n: f(z)= 0\}.$$

  By the fundamental theorem of algebra, every polynomial in $P_n$ has at most $n$ roots (including complex roots). Additionally, $|P_n|\ \forall n = |\Q|$. We can consider each coefficient to be a choice of an element from $\Q$, and the number of possible polynomials to be $|\cup_{i\in\N}^n \Q|$ (the union of the sets containing the coefficients we can choose from). Since this is the countable union of countable sets, the result is countable. Then since any $P_n$ is countable:
  $$\left| P_n \bigcup P_{n-1} \bigcup \dots \bigcup P_1 \right|= |\Q|.$$

  Since $P_n$ is countable, $R_n$ contains a countable number of solutions for countably many polynomials (technically, $n$ times countably infinite solutions -- which is still countably infinite). As such:
  $$\left| R_n \bigcup R_{n-1} \bigcup \dots \bigcup R_1 \right|= |\Q|.$$

  Since there are countably many roots in all, algebraic numbers are countable.

  We now prove that there are uncountable many transcendental numbers.

  Let $A$ be the set of all algebraic numbers. We know that $|A| = |\Q|$. The reals are the union of the sets of algebraic numbers $A$, and the set of transcendental numbers $T$. Since the union of any countable set with another countable set is countable, and we know that $|A\cup T| = |\R|$, $T$ must be uncountable.
\end{proof}



\pagebreak



\begin{enumerate}
\item[2.] Let $A$ be the set of all functions $f:\N \to \{0,1\}$.  Find the cardinality of $A$.
\end{enumerate}

\begin{proof}
  Let $A = \{f: f:\N \to \{0,1\}\}$. Let
  $$G=\{g:\N\to\{0,1\}\}, g(n) = \begin{cases}
      1 & rn\geq 0 \\
      0 & rn<0
   \end{cases}$$
  where $r\in\R$. Then, since $\R$ is uncountable, $G$ contains uncountable many functions.

  Since $G$ is a set of functions that map $\N$ to the set $\{0,1\}$, $G\subseteq A$. Furthermore, since $|G|=|\R|$, it must also be the case that $|A|=|\R|$.
\end{proof}



\pagebreak



\begin{enumerate}
\item[3.] Let $A$ be the set of all functions $f:\N \to \{0,1\}$ that are ``eventually zero'' (We say that $f$ is eventually zero if there is a positive integer $N$ such that $f(n) = 0$ for all $n \geq N$).  Find the cardinality of $A$.
\end{enumerate}

\begin{proof}
  Let $A = \{f: f:\N \to \{0,1\}, $``and $f$ eventually zero''$\}$. Let
  $$G=\{g:\N\to\{0,1\}\},g(n) = \begin{cases}
    1 & n\leq |r| \\
    0 & n>|r|
   \end{cases}$$

  where $r\in\R$. Then, since $\R$ is uncountable, $G$ contains uncountable many functions.

  Since $G$ is a set of functions that map $\N$ to the set $\{0,1\}$, $G\subseteq A$. Furthermore, since $|G|=|\R|$, it must also be the case that $|A|=|\R|$.
\end{proof}


\pagebreak



\begin{enumerate}
\item[4.] Use the axiom of choice to prove that if there exists $f:A \to B$ that is onto, then there exists a function $g:B \to A$ that is one-to-one.
\end{enumerate}

\begin{enumerate}
\item[5.] We say that $|A| \geq |B|$ if there exists a function $f:A \to B$ which is onto.  Prove that if $|A| \geq |B|$, and $|B| \geq |A|$, then $|A| = |B|$.  (Hint:  Use 4).
\end{enumerate}

\end{document} %Where the text for the document ends.
