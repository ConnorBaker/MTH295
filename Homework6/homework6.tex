% Author: Connor Baker
% Date Created: April 18, 2017
% Last Edited: April 28, 2017
% Version: 0.3a

% Declare type of document
\documentclass[10pt]{article}

% Import Packages
\usepackage[utf8]{inputenc}
\usepackage[mathscr]{euscript}
\usepackage{amsfonts,amsmath,amssymb,amsthm}
\usepackage{mathtools,mathdots}
\usepackage{enumitem}
\usepackage{array}
\usepackage{longtable}

% Page Formatting
% These settings let you manipulate the margins on the paper, and provide more options than you might be used to using in a word style document.  For example, the settings \oddsidemargin and \evensidemargin are allowed to be adjusted separately in case you are binding a book together.
\topmargin -0.25in \oddsidemargin -.25in \evensidemargin -.25in
\textheight 9in \textwidth 6.75in \headheight 0in \headsep .35in
\parindent 0in

% Define the basic math environments
\theoremstyle{definition}
\newtheorem{definition}[equation]{Definition}
\newtheorem{example}[equation]{Example}
\newtheorem{axiom}[equation]{Axiom}
\theoremstyle{plain}
\newtheorem{theorem}[equation]{Theorem}
\newtheorem{proposition}[equation]{Proposition}
\newtheorem{lemma}[equation]{Lemma}
\newtheorem{corollary}[equation]{Corollary}
\newtheorem{conjecture}[equation]{Conjecture}

% Define frequently used commands
\newcommand{\N}{\mathbb{N}}
\newcommand{\Z}{\mathbb{Z}}
\newcommand{\Q}{\mathbb{Q}}
\newcommand{\R}{\mathbb{R}}
\newcommand{\C}{\mathbb{C}}
\DeclareMathOperator\dom{dom}
\DeclareMathOperator\rang{rang}
\newcommand{\ds}{\displaystyle}

\makeatletter
\def\imod#1{\allowbreak\mkern10mu({\operator@font mod}\,\,#1)}
\makeatother

\begin{document} %This is where we type the text that we plan on having in our document.

% Create the Header
\begin{center}
  \subsection*{Homework 6\\Connor Baker, April 2017}
\end{center}

\begin{enumerate}
\item An algebraic number is any number that is a root of a polynomial with rational coefficients.  Prove that the algebraic numbers are countable.  A number is transcendental if it is not algebraic.  Prove there are uncountable many transcendental numbers.
\end{enumerate}

\begin{proof}
  Let $M_n$ be the set of all monomials of degree $n,n\in\Z^+$, which have rational coefficients:
  $$M_n = \{q_n x^n: q_n \in\Q\}.$$

  Let $P_n$ be the set of all polynomials of (at most) degree $n$, which have rational coefficients:
  $$P_n = \{M_n + M_{n-1} + \dots + M_1 + M_0\}.$$

  Then, $|P_n|=|M_0\cup M_1\cup \cdots \cup M_n| = |\Q|,$ since the countable union of countable sets is countable.

  Let $R_n$ be the set of all roots of $P_n$:
  $$R_n= \{x\in\R: \exists f\in P_n: f(x)= 0\}.$$

  For each $i\in\N$, let $p_i^n$ be a unique polynomial from $P_n$. Then, let $r_i^n\subseteq R_n$ be the roots of $p_i^n$. By the fundamental theorem of algebra, every polynomial of degree $n$ has at most $n$ real roots.

  Then, $$R_n = \bigcup_{i\in\N}^\infty r_i^n.$$
  As the countable union of countable sets, $R_n$ is countable, and $|R_n|=|\Q|$.

  Let $\mathbb{A} = \cup_{n\in\N} R_n$ be the set of algebraic numbers. Again, the countable union of countable sets is countable, and the algebraic numbers are countable.

  We now prove that there are uncountable many transcendental numbers by contradiction.

  Suppose that the set of all transcendental numbers $\mathbb{T}$ is countable. Then, since $\R$ is defined as $\mathbb{A}\cup\mathbb{T}$, $|\R|=|\mathbb{A}\cup\mathbb{T}|$. However, since the union of any countable set with another countable set is countable, $|A\cup T| = |\Q|$, which by transitivity means $|\Q|=|\R|$, which is a contradiction since the reals are uncountable. Therefore, by contradiction, $T$ must be uncountable.
\end{proof}



\pagebreak



\begin{enumerate}
\item[2.] Let $A$ be the set of all functions $f:\N \to \{0,1\}$.  Find the cardinality of $A$.
\end{enumerate}

\begin{proof}
  To show that $A$ has the same cardinality as $\R$, we must show that there is a bijective mapping between the two. Instead of creating a mapping to $\R$, we instead use $\{0,1\}^\N$, which has the same cardinality (as proved in class).

  Consider the function
  $$\underset{f\mapsto f(1)f(2)\dots}{g:A \to \{0,1\}^\N}.$$

  Claim: $g(f)\in\{0,1\}^\N$.
  The funciton $g$ always yields a sequence of ones and zeros. Since every sequence of ones and zeros is in $\{0,1\}^\N,$ so there is at least one solution, and $g$ is onto.

  Claim: $g(f)=g(h) \iff f=h$.
  Assume $g(f)=g(h)$. Then $f(1)=h(1), f(2)=h(2),\dots,f(n)=h(n)$. Since $\forall n, f(n)=h(n)$, and $\dom(f)=\dom(h)$, $f=h$. Then, $g$ is one-to-one.

  As such, $g$ is bijective, and $|A|=|\{0,1\}^\N|=|\R|$.
\end{proof}



\pagebreak



\begin{enumerate}
\item[3.] Let $A$ be the set of all functions $f:\N \to \{0,1\}$ that are ``eventually zero'' (We say that $f$ is eventually zero if there is a positive integer $N$ such that $f(n) = 0$ for all $n \geq N$).  Find the cardinality of $A$.
\end{enumerate}

\begin{proof}
  To show that $A$ has the same cardinality as $\N$, we must show that there is a bijective mapping between the two.

  Consider the function
  $$\underset{f\mapsto f(N-1)f(N-2)\dots f(2)f(1)}{g:A \to \N},$$
  where $N$ is the value for which all $f(n), n\geq N$ are zero.

  Claim: $g(f)\in\N.$
  The function $g$ maps $f$ to the binary representation of a number in $\N$. Since $g(f)$ is always a positive, whole binary number, $g$ is onto since there is always a solution for any $f$.

  Claim: $g(f)=g(h) \iff f=h$.
  Assume that $g(f)=g(h)$. Then $f(N-1) = h(N-1), f(N-2) = h(N-2), \dots, f(1) = h(1)$. Then, since for all $n, f(n)=h(n), f=h$, and $\dom(f)=\dom(h)$, $g$ is one-to-one.

  As such, $g$ is bijective, and $|A|=|\N|$.
\end{proof}



\pagebreak



\begin{enumerate}
\item[4.] Use the axiom of choice to prove that if there exists $f:A \to B$ that is onto, then there exists a function $g:B \to A$ that is one-to-one.
\end{enumerate}

\begin{proof}
This proof uses the axiom of choice, as defined below.

\begin{axiom}[Axiom of Choice]
  If $\mathscr{A}$ is any collection of nonempty sets, then there exists a function $F$ (called a choice function) from $\mathscr{A}$ to $\cup_{A\in\mathscr{A}} A$ such that for every $A\in\mathscr{A}, F(A)\in A.$
\end{axiom}

Let $\mathcal{F}=\{f^{-1}(\{b\}):b\in B\}.$ Then $\mathcal{F}$ is the set that contains the pre-image(s) of each element of $B$. Since $f$ is onto, every element of $\mathcal{F}$ is nonempty.

Furthermore, since $f$ is a function, everything in $\mathcal{F}$ is disjoint (because if $a\in f^{-1}(\{b_i\})$ and $a\in f^{-1}(\{b_j\})$, then $b_i=b_j$). We now use the axiom of choice.

Let $C$ be a choice function such that
$$C:\mathcal{F}\to \cup_{F\in\mathcal{F}} F.$$
Let $g$ be a function such that
$$g:\underset{g(b) \mapsto C(f^{-1}(\{b\}))}{B \to A}.$$

Claim: The function $g$ is one-to-one.
Let $b_i, b_j\in B$. Suppose that $g(b_i)=g(b_j)$. Then, $C(f^{-1}(\{b_i\})) = C(f^{-1}(\{b_j\}))$. Since $\mathcal{F}$ is disjoint, $b_i=b_j$, and $g$ is one-to-one.
\end{proof}




\pagebreak




\begin{enumerate}
\item[5.] We say that $|A| \geq |B|$ if there exists a function $f:A \to B$ which is onto.  Prove that if $|A| \geq |B|$, and $|B| \geq |A|$, then $|A| = |B|$.  (Hint:  Use 4).
\end{enumerate}

\begin{proof}
  Since $|A| \geq |B|$, there exists a function $f:A\to B$ which is onto. By the previous proof, there exists a function $f':B\to A$ which is one-to-one.

  Since $|B| \geq |A|$, there exists a function $g:B\to A$ which is onto. By the previous proof, there exists a function $g':A\to B$ is one-to-one.

  Since there exists both onto and one-to-one functions that map $A\to B$ and the same for functions that map $B\to A$, there exists a bijective mapping between the two sets.

  Since there exists a bijective mapping between the two functions, $|A| = |B|$.
\end{proof}

\end{document} %Where the text for the document ends.
