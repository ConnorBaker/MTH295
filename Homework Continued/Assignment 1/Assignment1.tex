% Author: Connor Baker
% Date Created: May 29, 2017
% Last Edited: June 22, 2017
% Version: 0.2c

% Declare type of document
\documentclass[10pt]{article}

% Import Packages
\usepackage[utf8]{inputenc}
\usepackage[mathscr]{euscript}
\usepackage{amsfonts,amsmath,amssymb,amsthm}
\usepackage{mathtools,mathdots}
\usepackage{enumitem}
\usepackage{array}
\usepackage{longtable}

% Page Formatting
% These settings let you manipulate the margins on the paper, and provide more options than you might be used to using in a word style document.  For example, the settings \oddsidemargin and \evensidemargin are allowed to be adjusted separately in case you are binding a book together.
\topmargin -0.25in \oddsidemargin -.25in \evensidemargin -.25in
\textheight 9in \textwidth 6.75in \headheight 0in \headsep .35in
\parindent 0in

% Define the basic math environments
\theoremstyle{definition}
\newtheorem{definition}[equation]{Definition}
\newtheorem{example}[equation]{Example}
\newtheorem{axiom}[equation]{Axiom}
\theoremstyle{plain}
\newtheorem{theorem}[equation]{Theorem}
\newtheorem{proposition}[equation]{Proposition}
\newtheorem{lemma}[equation]{Lemma}
\newtheorem{corollary}[equation]{Corollary}
\newtheorem{conjecture}[equation]{Conjecture}

% Define frequently used commands
\newcommand{\N}{\mathbb{N}}
\newcommand{\Z}{\mathbb{Z}}
\newcommand{\Q}{\mathbb{Q}}
\newcommand{\R}{\mathbb{R}}
\newcommand{\C}{\mathbb{C}}
\DeclareMathOperator\dom{dom}
\DeclareMathOperator\rang{rang}
\newcommand{\ds}{\displaystyle}

\makeatletter
\def\imod#1{\allowbreak\mkern10mu({\operator@font mod}\,\,#1)}
\makeatother

\begin{document} %This is where we type the text that we plan on having in our document.

% Create the Header
\begin{center}
  {\Large Rudin's \textsc{Principles of Mathematical Analysis, 3rd ed}}

  {\large Connor Baker, June 2017}

  \subsection*{Basic Topology: Selected Exercises}
\end{center}
\begin{enumerate}
\item[5.] Construct a bounded set of real numbers with exactly three limit points.
\end{enumerate}

\begin{lemma}
Consider the set $E$:
$$E = \left\{\frac{1}{n}: n\in\N\right\}.$$
Claim: $|E|=|\N|$. \\

Let $f:E\rightarrow\N$ where $\frac{1}{n} \mapsto n$. We will show that $f$ is one-to-one. \\
Assume $f(a)=f(b)$. Then $a=b$, so $f$ is one-to-one. \\
Choose $n\in\N$. Then there exists exactly one $\frac{1}{n}\in E$ such that $f(\frac{1}{n}) = n$ (since the reciprocal of any natural number is unique relative to the reciprocal of any other number in the naturals). As such, $f$ is onto. \\

Since there is a one-to-one correspondence between $E$ and $\N$, $|E|=|\N|$.
\end{lemma}


\begin{proof}[CHECK ME]
Let $E$ be the set of numbers such that:
$$E = \bigcup_{k=0}^{2} A_k, \qquad A_k = \left\{\frac{1}{n} + k: n\in\N\right\}.$$

By Lemma 1, we know that $A_0$ has the same cardinality as $N$. Since $\lim_{n\rightarrow\infty} \frac{1}{n} = 0$ the value of $k$ allows us to change the value the limit approaches. As such, every value of $A_1$ is one larger than the corresponding value of $A_0$, as is $A_2$ with respect to $A_1$. The cardinality remains unchanged.

Since the cardinality of $A_k$ is that of the naturals, the neighborhood $N_r(k)$ for some $r>0$ intersect $A_k$ is infinite.

Therefore the set of limit points of $E$ must contain at least the points 0, 1, and 2, and as such, $\{0,1,2\}\subseteq E'$. \\

We now show that $E'$ contains only 0, 1, and 2. \\

Suppose that there is some limit point $x\not\in\{0,1,2\}$. Let $\epsilon = \min(d(x,0),\ d(x,1),\ d(x,2))$ (we want the smallest radius possible). Then the neighborhood 
$$N_\frac{\epsilon}{2} (x) = \left\{y: d(x,y) < \frac{\epsilon}{2}\right\}$$
does not contain the set $(0, \frac{\epsilon}{2})\cup(1, 1+\frac{\epsilon}{2})\cup(2, 2+\frac{\epsilon}{2})$. Each interval in that union has no least lower bound, since $\N$ is not bounded above, we can pick larger and larger $n$ for $\frac{1}{n}$, which the interval is composed of. Since this portion of the interval is not included, the neighborhood is finite. Since the neighborhood is finite, the intersection with $E$ is finite and therefore $x$ is not a limit point. \\

As such, $x\not\in E'$, and $E' = \{0,1,2\}$.
\end{proof}



\pagebreak



\begin{enumerate}
\item[6.] Let $E'$ be the set of all limit points of a set $E$. Prove that $E'$ is closed. Prove that $E$ and $\bar{E}$ have the same limit points. (Recall that $\bar{E} = E \cup E'.$) Do $E$ and $E'$ always have the same limit points?
\end{enumerate}

\begin{proof}
Prove that $E'$ is closed. \\

The set $E'$ is closed if it contains its own limit points. \\

Let $x\in \bar{E}'$.

% Assume that $p$ is a limit point of $E'$. Then, $\exists x\neq p,$ (since if $x=p, d(x,p)=0$, and the neighborhood has a finite number of points, so there can not exist a limit point). We fix $r>0$, so $\exists x\in E'\cap N_{\frac{r}{2}}(p)$ and $\exists y\in E \cap N_{\frac{r}{2}}(x)$. \\

% Since $x\in E'$, $x$ is a limit point of $E$, by the definition of $E'$. As such, $\exists y\in E: d(x,y) < \frac{r}{2}$. By the triangle inequality:
% $$d(y,p) \leq d(y,x) + d(x,p).$$

% Substituting using the inequalities described above:
% $$d(y,p) < \frac{r}{2}+\frac{r}{2} =r,$$

% so $p$ is a limit point of $E$. Since $p$ is a limit point of $E$, it must be in $E'$. Therefore, since $E'$ contains its own limit points, it is closed.
\end{proof}

\begin{proof}
Proof that $E$ and $\bar{E}$ have the same limit points. \\

By the previous proof, $E'$ is closed and therefore contains its own limit points. \\

Let $\bar{E}'$ be the set of all limit points of $\bar{E}$. \\

We begin by showing that $E'\subseteq \bar{E}'$. \\

Let $x\in E'$. Then $x\in  $

\end{proof}

\begin{proof}[CHECK ME]
  Do $E$ and $E'$ always have the same limit points? \\

  Consider the set
  $$E = \left\{\frac{1}{n}: n\in\N\right\}.$$

  Since $\lim_{n\rightarrow\infty} \frac{1}{n} = 0$, and $|E|=|\N|$ (by Lemma 1), $N_r (0)\cap E$ has infinitely many points, zero must be a limit point of $E$, so $E' = \{0\}$. \\
  
  However, the limit points of $E'$ are the empty set, since no matter the point we pick to center the neighborhood, the intersection will contain at most one element (zero). As such, $E$ and $E'$ do not always have the same limit points. \\
\end{proof}



\pagebreak



\begin{enumerate}
\item[7.] Let $A_1,A_2,A_3,\dots$ be subsets of a metric space.
\begin{enumerate}
  \item If $B_n = \cup_{i=1}^{n} A_i,$ prove that $\bar{B}_n = \cup_{i=1}^{n} \bar{A}_i,$ for $n=1,2,3,\dots$.
  \item If $B = \cup_{i=1}^{\infty} A_i,$ prove that $\bar{B} \supset \cup_{i=1}^{\infty} \bar{A}_i.$
\end{enumerate}
Show, by an example, that this inclusion can be proper.
\end{enumerate}

\begin{proof}[CHECK ME]
Note that:
$$\bar{B}_n = B_n \cup B_n'$$
$$\bigcup_{i=1}^n \bar{A}_i = \bigcup_{i=1}^n \{A_i \cup A_i'\} = $$
Claim: $\bar{B}_n \subseteq \cup_{i=1}^{n} \bar{A}_i$.

Assume $x\in \bar{B}_n$. \\

Case 1: $x\in B_n$. \\

Since $B_n \subseteq \cup_{i=1}^n A_i$, and $x\in B_n$, $x\in \cup_{i=1}^n A_i \subseteq \cup_{i=1}^n \bar{A}_i$. Therefore $x\in\cup_{i=1}^n \bar{A}_i$. \\

Case 2: $x\in B_n'$. \\

Since $B_n = \cup_{i=1}^n A_i$, and $x\in B_n'$, then $x$ must be a limit point of some $A_i$ in $\cup_{i=1}^n A_i$, so $x\in A_i'$, and as such is in the union $\cup_{i=1}^n A_i'$. Therefore $x\in\cup_{i=1}^n \bar{A}_i$. \\

As such, $\bar{B}_n \subseteq \cup_{i=1}^{n} \bar{A}_i$. \\

Claim: $\cup_{i=1}^{n} \bar{A}_i \subseteq \bar{B}_n$. \\

Assume $y\in\cup_{i=1}^{n} \bar{A}_i$. \\

Case 1: $y\in\cup_{i=1}^{n} A_i$. \\

Since $B_n = \cup_{i=1}^{n} A_i, y\in B_n$.

Case 2: $y\in\cup_{i=1}^{n} A_i'$. \\

Since $B_n = \cup_{i=1}^{n} A_i$, they must have the same limit points. As such, $y\in B_n'$.

As a result, $\cup_{i=1}^{n} \bar{A}_i \subseteq \bar{B}_n$. \\

Therefore, $\bar{B}_n = \cup_{i=1}^{n} \bar{A}_i$.
\end{proof}



\pagebreak



\begin{enumerate}
\item[8.] Is every point of every open set $E \subset \R^2$ a limit point of $E$? Answer the same question for closed sets in $\R^2$.
\end{enumerate}

\begin{proof}[CHECK ME]
Since $E$ is open, all elements of $E$ are interior points. \\

Let $p\in E$. Choose $r>0$, and let $N_r(p)$ be a neighborhood of $p$. Since $p$ is an interior point, $\exists \delta > 0: N_\delta (p) \subseteq E$. \\

Choose $0<\epsilon \leq \min(r,\delta)$. Then, $N_\epsilon (p) \subseteq N_\delta (p) \subseteq E$. Since $N_\epsilon (p)$ contains infinitely many points of $E$, and $N_\epsilon (p) \subseteq N_r (p)$, $N_r(p)$ does as well. \\

Closed sets, by definition contain all of their own limit points, so any closed subset of $\R^2$ will contain its own limit points. However, it is not the case that any finite set of $E$ will have limit points since the neighborhood intersect $E$ will be finite.
\end{proof}




\pagebreak




\begin{enumerate}
\item[9.] Let $E^\circ$ denote the set of all interior points of a set $E$.
\begin{enumerate}
  \item Prove that $E^\circ$ is always open.
  \item Prove that $E$ is open if and only if $E^\circ = E$.
  \item If $G \subset E$ and $G$ is open, prove that $G\subset E^\circ$.
  \item Prove that the complement of $E^\circ$ is the closure of the complement of $E$.
  \item Do $E$ and $\bar{E}$ always have the same interiors?
  \item Do $E$ and $E^\circ$ always have the same closures?
\end{enumerate}
\end{enumerate}

\begin{proof}
Proof that $E^\circ$ is always open. \\

$E^\circ = \{p: N_r(p) \subseteq E$ for some $r\}$.
\end{proof}

\begin{proof}[CHECK ME]
Proof that $E$ is open if and only if $E^\circ = E$. \\

Assume that $E^\circ =E$. Then, $E$ is the set of all interior points of $E$. Since every point of $E$ is an interior point, $E$ is open.
\end{proof}

\begin{proof}[CHECK ME]
If $G \subset E$ and $G$ is open, prove that $G\subset E^\circ$. \\

Since $G$ is open, every point of $G$ is an interior point of $G$. Since $G\subseteq E$, $G$ is a set of some number of interior points of $E$, so $G\subseteq E^\circ$.
\end{proof}

\begin{proof}
Prove that the complement of $E^\circ$ is the closure of the complement of $E$. \\

$(E^\circ)^c = \bar{E^c} = (E^c \cup E'^c)$?
\end{proof}

\begin{proof}
Do $E$ and $\bar{E}$ always have the same interiors?
\end{proof}

\begin{proof}
Do $E$ and $E^\circ$ always have the same closures? \\

Since $\bar{E} = E\cup E'$, and $\bar{E}^\circ = E^\circ \cup E'^\circ$, we must show that $\bar{E} \subset \bar{E}^\circ$ and $\bar{E}^\circ \subset \bar{E}$. \\

$\bar{E} \subset \bar{E}^\circ$: \\

$\bar{E}^\circ \subset \bar{E}$:
\end{proof}

\end{document} %Where the text for the document ends.
