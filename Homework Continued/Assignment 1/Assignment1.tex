% Author: Connor Baker
% Date Created: May 29, 2017
% Last Edited: June 3, 2017
% Version: 0.1c

% Declare type of document
\documentclass[10pt]{article}

% Import Packages
\usepackage[utf8]{inputenc}
\usepackage[mathscr]{euscript}
\usepackage{amsfonts,amsmath,amssymb,amsthm}
\usepackage{mathtools,mathdots}
\usepackage{enumitem}
\usepackage{array}
\usepackage{longtable}

% Page Formatting
% These settings let you manipulate the margins on the paper, and provide more options than you might be used to using in a word style document.  For example, the settings \oddsidemargin and \evensidemargin are allowed to be adjusted separately in case you are binding a book together.
\topmargin -0.25in \oddsidemargin -.25in \evensidemargin -.25in
\textheight 9in \textwidth 6.75in \headheight 0in \headsep .35in
\parindent 0in

% Define the basic math environments
\theoremstyle{definition}
\newtheorem{definition}[equation]{Definition}
\newtheorem{example}[equation]{Example}
\newtheorem{axiom}[equation]{Axiom}
\theoremstyle{plain}
\newtheorem{theorem}[equation]{Theorem}
\newtheorem{proposition}[equation]{Proposition}
\newtheorem{lemma}[equation]{Lemma}
\newtheorem{corollary}[equation]{Corollary}
\newtheorem{conjecture}[equation]{Conjecture}

% Define frequently used commands
\newcommand{\N}{\mathbb{N}}
\newcommand{\Z}{\mathbb{Z}}
\newcommand{\Q}{\mathbb{Q}}
\newcommand{\R}{\mathbb{R}}
\newcommand{\C}{\mathbb{C}}
\DeclareMathOperator\dom{dom}
\DeclareMathOperator\rang{rang}
\newcommand{\ds}{\displaystyle}

\makeatletter
\def\imod#1{\allowbreak\mkern10mu({\operator@font mod}\,\,#1)}
\makeatother

\begin{document} %This is where we type the text that we plan on having in our document.

% Create the Header
\begin{center}
  {\Large Rudin's \textsc{Principles of Mathematical Analysis, 3rd ed}}

  {\large Connor Baker, May 2017}

  \subsection*{Basic Topology -- Selected Exercises}
\end{center}
\begin{enumerate}
\item[5.] Construct a bounded set of real numbers with exactly three limit points.
\end{enumerate}

\begin{proof}
  The point $p$ is a limit point of $\{\frac{1}{n}:n\in\N\}$ if $\forall r>0, N_r(p)\cap E$ has infinitely many points. \\

  Claim: Zero is a limit point of $\{\frac{1}{n}:n\in\N\}$. \\

  For zero, $N_r(0) = \{x:d(x,0) < r\} = \{x: |x-0| < r\} = \{x: -r <x< r\} = (-r,r)$. Since $(r,r)\subseteq \R, \forall r>0,$ there are uncountably infinitely many points and $(r,r)\cap\{\frac{1}{n}:n\in\N\}$ has infinitely many points. \\

  It is important to note that we can shift the location of this limit point. Consider the set $\{\frac{1}{n}+1:n\in\N\}$. For the same reasons above we can see that the limit point here is one. \\

  In general, we can make a bounded set of real numbers with exactly $p, p\in\Z^+,$ limit points with the set
  $$\bigcup_{k=0}^{p} \{\frac{1}{n}+k: n\in\N\}.$$

  So, if we want three limit points, our set is as
  $$\left\{\frac{1}{n}: n\in\N\right\}\bigcup\left\{\frac{1}{n}+1: n\in\N\right\}\bigcup\left\{\frac{1}{n}+2: n\in\N\right\}.$$
\end{proof}



\pagebreak



\begin{enumerate}
\item[6.] Let $E'$ be the set of all limit points of a set $E$. Prove that $E'$ is closed. Prove that $E$ and $\bar{E}$ have the same limit points. (Recall that $\bar{E} = E \cup E'.$) Do $E$ and $E'$ always have the same limit points?
\end{enumerate}

\begin{proof}
The set $E'$ is closed if it contains its own limit points. \\

Assume that $p$ is a limit point of $E'$. Then, $\exists x\neq p,$ (since if $x=p, d(x,p)=0$, and the neighborhood has a finite number of points, so there can not exist a limit point). We fix $r>0$, so $\exists x\in E'\cap N_{\frac{r}{2}}(p)$ and $\exists y\in E \cap N_{\frac{r}{2}}(x)$. \\

Since $x\in E'$, $x$ is a limit point of $E$, by the definition of $E'$. As such, $\exists y\in E: d(x,y) < \frac{r}{2}$. By the triangle inequality:
$$d(y,p) \leq d(y,x) + d(x,p).$$

Substituting using the inequalities described above:
$$d(y,p) < \frac{r}{2}+\frac{r}{2} =r,$$

so $p$ is a limit point of $E$. Since $p$ is a limit point of $E$, it must be in $E'$. Therefore, since $E'$ contains its own limit points, it is closed.
\end{proof}

\begin{proof}
A set and a set which is the union of that set and the set containing its limit points have the same limit points. \\

By the previous proof, $E'$ is closed and therefore contains its own limit points. \\

Let $\bar{E}'$ be the set of all limit points of $\bar{E}$. \\

We begin by showing that $E'\subseteq \bar{E}'$.

\end{proof}

\begin{proof}
  A set and the set that contains that set's limit points do not necessarily have the same limit points.

  Consider the set
  $$E = \left\{\frac{1}{n}: n\in\N\right\}.$$

  Then the limit point of that set is zero, so $E' = \{0\}$. However, the limit points of $E'$ are the empty set. We know this because to be a limit point, the intersection of the neighborhood and the set must have infinitely many points. This is not possible for our $E'$, since we have at most a single point in the intersection. \\
\end{proof}



\pagebreak



\begin{enumerate}
\item[7.] Let $A_1,A_2,A_3,\dots$ be subsets of a metric space.
\begin{enumerate}
  \item If $B_n = \cup_{i=1}^{n} A_i,$ prove that $\bar{B}_n = \cup_{i=1}^{n} \bar{A}_i,$ for $n=1,2,3,\dots$.
  \item If $B = \cup_{i=1}^{\infty} A_i,$ prove that $\bar{B} \supset \cup_{i=1}^{\infty} \bar{A}_i.$
\end{enumerate}
Show, by an example, that this inclusion can be proper.
\end{enumerate}

\begin{proof}

\end{proof}



\pagebreak



\begin{enumerate}
\item[8.] Is every point of every open set $E \subset \R^2$ a limit point of $E$? Answer the same question for closed sets in $\R^2$.
\end{enumerate}

\begin{proof}

\end{proof}




\pagebreak




\begin{enumerate}
\item[9.] Let $E^\circ$ denote the set of all interior points of a set $E$.
\begin{enumerate}
  \item Prove that $E^\circ$ is always open.
  \item Prove that $E$ is open if and only if $E^\circ = E$.
  \item If $G \subset E$ and $G$ is open, prove that $G\subset E^\circ$.
  \item Prove that the complement of $E^\circ$ is the closure of the complement of $E$.
  \item Do $E$ and $\bar{E}$ always have the same interiors?
  \item Do $E$ and $E^\circ$ always have the same closures?
\end{enumerate}
\end{enumerate}

\begin{proof}

\end{proof}

\end{document} %Where the text for the document ends.
