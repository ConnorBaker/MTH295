% Author: Connor Baker
% Date Created: May 29, 2017
% Last Edited: August 26, 2017
% Version: 0.4c

% Declare type of document
\documentclass[10pt]{article}

% Import Packages
\usepackage[utf8]{inputenc}
\usepackage[mathscr]{euscript}
\usepackage{amsfonts,amsmath,amssymb,amsthm}
\usepackage{mathtools,mathdots}
\usepackage{enumitem}
\usepackage{array}
\usepackage{longtable}
\usepackage{dirtytalk}

% Page Formatting
% These settings let you manipulate the margins on the paper, and provide more options than you might be used to using in a word style document.  For example, the settings \oddsidemargin and \evensidemargin are allowed to be adjusted separately in case you are binding a book together.
\topmargin -0.25in \oddsidemargin -.25in \evensidemargin -.25in
\textheight 9in \textwidth 6.75in \headheight 0in \headsep .35in
\parindent 0in

% Define the basic math environments
\theoremstyle{definition}
\newtheorem{definition}[equation]{Definition}
\newtheorem{example}[equation]{Example}
\newtheorem{axiom}[equation]{Axiom}
\theoremstyle{plain}
\newtheorem{theorem}[equation]{Theorem}
\newtheorem*{theorem*}{Theorem}
\newtheorem{proposition}[equation]{Proposition}
\newtheorem{lemma}[equation]{Lemma}
\newtheorem*{lemma*}{Lemma}
\newtheorem{corollary}[equation]{Corollary}
\newtheorem*{corollary*}{Corollary}
\newtheorem{conjecture}[equation]{Conjecture}

% Define frequently used commands
\newcommand{\N}{\mathbb{N}}
\newcommand{\Z}{\mathbb{Z}}
\newcommand{\Q}{\mathbb{Q}}
\newcommand{\R}{\mathbb{R}}
\newcommand{\C}{\mathbb{C}}
\DeclareMathOperator\dom{dom}
\DeclareMathOperator\rang{rang}
\newcommand{\ds}{\displaystyle}

\makeatletter
\def\imod#1{\allowbreak\mkern10mu({\operator@font mod}\,\,#1)}
\makeatother

\begin{document} %This is where we type the text that we plan on having in our document.

% Create the Header
\begin{center}
  {\Large Rudin's \textsc{Principles of Mathematical Analysis, 3rd ed}}

  {\large Connor Baker, June 2017}

  \subsection*{Basic Topology: Selected Exercises}
\end{center}
\begin{enumerate}
\item[5.] Construct a bounded set of real numbers with exactly three limit points.
\end{enumerate}

\begin{theorem*}[Limit points of $E_p$]\label{theorem:thm1}
Consider the set $E_p$:
$$E_p = \left\{\frac{1}{n}+p: n\in\N\right\}.$$

When $p\in\Z$, the limit point of $E_p$ is the point $p$.
\end{theorem*}

\begin{proof}
We begin by showing that $|E_p| = |\N|.$ \\

Claim: $|E_p|=|\N|$. \\

Let $f:E_p \rightarrow\N$ where $\frac{1}{n} + p \mapsto n$. Assume that $f(a)=f(b)$. Then $\frac{1}{a} + p =\frac{1}{b} + p$. By cancellation, $\frac{1}{a}=\frac{1}{b}$, so $a=b$, and $f$ is one-to-one. \\

Choose $n\in\N$. Since the reciprocal of any natural number is unique relative to the reciprocal of any other number in the naturals, there exists exactly one $\frac{1}{n} + p \in E_p$ such that $f(\frac{1}{n} + p) = n$. As such, $f$ is onto. \\

Since there is a one-to-one correspondence between $E$ and $\N$, $|E|=|\N|$. \\

We now show that the set $E_p$ has a limit point $p$. \\

Since the cardinality of $E_p$ is that of the naturals, for $r>0, N_r(p)\cap E_p$ is infinite (we choose $p$ because that is what our sequence converges to -- the choice of any other point $\frac{1}{n} + p$ results in $\epsilon: \forall r > \epsilon,
N_r(\frac{1}{n} + p)\cap E_p$ is finite). Therefore the set of limit points of $E_p$ must contain at least the point $p$; and as such, $\{p\}\subseteq E'$. \\

We now show that $E_p'$ contains only the point $p$. \\

Suppose that there is some limit point $x\neq p$. Let $\delta = d(x,p)$. Then the neighborhood
$$N_\frac{\delta}{2} (x) = \left\{y: d(x,y) < \frac{\delta}{2}\right\}$$
does not contain the interval $(p, x -\frac{\delta}{2}]$ (inclusive since the neighborhood includes $(x -\frac{\delta}{2}, x +\frac{\delta}{2})$, by definition). The interval $(p, x -\frac{\delta}{2}]$ is equivalent to the set $\{\frac{1}{n} + p: \frac{1}{n} \leq x -\frac{\delta}{2}\}$. This set has no least lower bound, since $\N$ is not bounded above, so we are able to pick larger and larger $n$ for $\frac{1}{n}$ which satisfy the inequality. By not including this portion of the interval, the neighborhood becomes finite. Since the neighborhood is finite, $N_\frac{\delta}{2} \cap E_p$ is finite and therefore $x$ is not a limit point of $E_p$. \\

As such, $x\not\in E_p'$, and $E_p' = \{p\}$.
\end{proof}

\begin{corollary*}[$E_p\backslash N_r(p)$ is finite]
Given the set $E_p$, excluding the neighborhood of $p$ makes the cardinality of the set finite. That is to say that $E_p\backslash N_r(p)$ is finite.
\end{corollary*}

The solution to this example is then trivial, as we can take the finite union of any $E_p$ that we choose to create sets with an arbitrary finite number of limit points. For example, the set $\cup^2_{i=0} E_i$ contains exactly three limit points (the points 0, 1, and 2).


\pagebreak



\begin{enumerate}
\item[6.] Let $E'$ be the set of all limit points of a set $E$. Prove:
\begin{enumerate}
  \item $E'$ is closed.
  \item $E$ and $\bar{E}$ have the same limit points.
  \item Whether or not $E$ and $E'$ always have the same limit points.
\end{enumerate}
\end{enumerate}

\begin{proof}
6a)

Let $p\in E''$. Then, for $r>0$,
$$\exists q\in N_{\frac{r}{2}} (p):q\in E', q\neq p.$$

Let
$$\delta = \frac{d(p,q)}{2} < \frac{r}{2}.$$

Since $q\in E'$, for $r>0$, $\exists s\in N_\delta (q): s\in E$, and  $s\neq q$. Due to the definition of the radius of $N_\delta (q), s\neq p$ since the neighborhood's radius is not large enough to contain $p$.

Then:
$$d(p,q) < \frac{r}{2}$$
$$d(q,s) < \delta$$
$$d(p,s) \leq d(p,q) + d(q,s) < \frac{r}{2} + \delta < \frac{r}{2} + \frac{r}{2} = r.$$

So, $p\in E'$, and $E'$ is closed.
\end{proof}

\begin{proof}
6b)

Proof that $E$ and $\bar{E}$ have the same limit points. \\

By the previous proof, $E'$ is closed and therefore contains its own limit points. \\

Let $\bar{E}'$ be the set of all limit points of $\bar{E}$. \\

We begin by showing that $E'\subseteq \bar{E}'$. \\

Let $x\in E'$. Then for all $r>0, N_r(x)$ contains some point $y\neq x$. Since $E\subseteq\bar{E}$, $y\in\bar{E}$, so $x\in\bar{E}'$, and $E\subseteq\bar{E}$. \\

We now show that $\bar{E}'\subseteq E'$. \\

Let $x\in\bar{E}'$. Then for all $r>0, N_r(x)$ contains some point $y\neq x$. Since $\bar{E} = E\cup E'$, $y\in E$ or $y\in E'$. If $y\in E$, then $x\in E'$. If $y\in E'$, choose $\epsilon>0$ such that $\epsilon < d(x,y)$ and $N_\epsilon(y) \subseteq N_r(x)$. Since $y$ is a limit point of the set $E$, then there is some point $z\in E$ such that $z\neq y$ and $z\in N_\epsilon(y)$. Since $z\neq x$ (due to $x\not\in N_\epsilon(y)$), and $N_\epsilon(y) \subseteq N_r(x)$, any neighborhood of $x$ contains $z\neq x$ with $x\in E$, so $x$ must be a limit point of $E$.
\end{proof}

\begin{proof}
 6c)

  Consider the set
  $$E = \left\{\frac{1}{n}: n\in\N\right\}.$$

  By the theorem Limit points of $E_p$, we know that $E' = \{0\}$. \\

  However, the limit points of $E'$ are the empty set, since no matter the point we pick to center the neighborhood, the intersection will contain at most one element (zero). As such, $E$ and $E'$ do not always have the same limit points. \\
\end{proof}



\pagebreak



\begin{enumerate}
\item[7.] Let $A_1,A_2,A_3,\dots$ be subsets of a metric space.
\begin{enumerate}
  \item If $B_n = \cup_{i=1}^{n} A_i,$ prove that $\bar{B}_n = \cup_{i=1}^{n} \bar{A}_i,$ for $n=1,2,3,\dots$.
  \item If $B = \cup_{i=1}^{\infty} A_i,$ prove that $\bar{B} \supset \cup_{i=1}^{\infty} \bar{A}_i,$ and show by an example that this inclusion can be proper.
\end{enumerate}

\end{enumerate}

\begin{lemma*}
$\overline{A\cup B} = \overline{A}\cup\overline{B}.$
\end{lemma*}

\begin{proof}
7a)

Claim: $(A\cup B)' \subseteq A' \cup B'.$

Case 1: The trivial case is that $A\cup B$ is finite. Then $(A\cup B)'$ is the empty set, since finite sets cannot have limit points, which is a subset of any set so our claim holds true.

Case 2: $A\cup B$ is not finite. Then let $x\in(A\cup B)'$. Choose any $r>0$, and consider $N_r (x) \cap A$, and $N_r (x) \cap B$. Suppose both are finite. Then $N_r (x) \cap (A\cup B)$ is finite, and $x$ is not a limit point. This is a contradiction, so our assumption that both are finite is incorrect. Therefore, at least one must be infinite, so $N_r (x) \cap A$ or $N_r (x) \cap B$ must be infinite, and therefore $x\in A'\cup B'.$

Claim: $A' \cup B' \subseteq (A\cup B)'.$

Let $x\in A'\cup B'.$ Choose any $r>0$. Then, $N_r (x) \cap A$ or $N_r (x) \cap B$  is infinite, and $x\in(A\cup B)'$.

Therefore, $(A\cup B)' = A' \cup B',$ and
$$\overline{A\cup B} = (A\cup B)\cup(A\cup B)' = A\cup A' \cup B \cup B' = \overline{A}\cup\overline{B}.$$
\end{proof}

\begin{proof}
7a) Cont.

We proceed with proof by induction.

Let $n = 1.$ Then:
$$\overline{\bigcup^1_{i=1} A_i} = \overline{A_1}=\bigcup^1_{i=1} \overline{A_i}.$$

Let $n=k$. Assume that
$$\overline{\bigcup^k_{i=1} A_i} = \bigcup^k_{i=1} \overline{A_i},$$ for all $k\in\N$.

Let $n=k+1$. Then,
$$\overline{\bigcup^{k+1}_{i=1} A_i} = \overline{\bigcup^k_{i=1} A_i \cup A_{k+1}},$$
which by the lemma above is equivalent to (using our assumption for $n=k$)
$$\overline{\bigcup^k_{i=1} A_i} \cup \overline{A_{k+1}} = \bigcup^k_{i=1} \overline{A_i} \cup \overline{A_{k+1}} = \bigcup^{k+1}_{i=1} \overline{A_i}.$$

Therefore
$$B_n = \bigcup^n_{i=1} A_i, \qquad \overline{B_n} = \overline{\bigcup^n_{i=1} A_i} = \bigcup^n_{i=1} \overline{A_i}$$
by the Principle of Mathematical Induction.
\end{proof}

\begin{proof}
7b)

Since $B_i \supseteq \cup A_i$, and $B \supseteq A_i\ \forall i$, then $\overline{B} \supseteq \overline{A}_i\ \forall i$. Then, $\overline{B} \supseteq \cup^{\infty}_{i=1} \overline{A}_i$. However, it is not necessarily true that $\overline{B} \subseteq \cup^{\infty}_{i=1} \overline{A}_i$. \\

Let $A_i = \{r_i\}$ where the set $\{r_1, r_2, \dots\}$ is an enumeration of $\Q$. Then $\overline{B} = \R$, and $A_i' = \emptyset \forall i$, so $\cup \overline{A}_i = B$. As such, $\overline{B} \supset \cup^{\infty}_{i=1} \overline{A}_i$.
\end{proof}


\pagebreak



\begin{enumerate}
\item[8.] Is every point of every open set $E \subset \R^2$ a limit point of $E$? Answer the same question for closed sets in $\R^2$.
\end{enumerate}

\begin{proof}
Since $E$ is open, all elements of $E$ are interior points. \\

Let $p\in E$. Choose $r>0$, and let $N_r(p)$ be a neighborhood of $p$. Since $p$ is an interior point, $\exists \delta > 0: N_\delta (p) \subseteq E$. \\

Choose $0<\epsilon \leq \min(r,\delta)$. Then, $N_\epsilon (p) \subseteq N_\delta (p) \subseteq E$. Since $N_\epsilon (p)$ contains infinitely many points of $E$, and $N_\epsilon (p) \subseteq N_r (p)$, $N_r(p)$ does as well. \\

It is not the case that any finite set $B$ will have limit points since the neighborhood intersect $B$ will be finite.
\end{proof}




\pagebreak




\begin{enumerate}
\item[9.] Let $E^\circ$ denote the set of all interior points of a set $E$.
\begin{enumerate}
  \item Prove that $E^\circ$ is always open.
  \item Prove that $E$ is open if and only if $E^\circ = E$.
  \item If $G \subset E$ and $G$ is open, prove that $G\subset E^\circ$.
  \item Prove that the complement of $E^\circ$ is the closure of the complement of $E$.
  \item Do $E$ and $\bar{E}$ always have the same interiors?
  \item Do $E$ and $E^\circ$ always have the same closures?
\end{enumerate}
\end{enumerate}

\begin{proof}
9a) \textit{Didn't show that all points of $E^\circ$ are interior to $E^\circ$.}

Since every point of $E^\circ$ is an interior point (by definition of $E^\circ$), the set $E^\circ$ is open.
\end{proof}

\begin{proof}
9b) \textit{Must prove both directions.}

A set is open if every point of the set is an interior point of the set (the definition of open). If a set $E$ is equal to the set of it's interior points, $E^\circ$, then every point of $E$ is an interior point of $E$, and $E$ is open. As such, $E$ is open if and only if $E^\circ = E$.
\end{proof}

\begin{proof}
9c) \textit{Proof is lacking.}

'$\exists\delta > 0: N_\delta (x) \subseteq G$ but $N_\delta (x) \subseteq E, x\in E^\circ$'. \\

Must review following for accuracy. \\

Since $G$ is open, $G=G^\circ$. \\

Claim: $G^\circ \subseteq E^\circ$. \\

Assume $x\in G^\circ$. Then $x\in E$.
Assume that $x\not\in E^\circ.$ Then $x$ is not an interior point of $E$, and will not be contained in any open subset of $E$, like $G^\circ$, which is a contradiction. \\

Therefore $x\in E^\circ$ and $G^\circ \subseteq E^\circ$. Since $G^\circ = G$, and $G^\circ \subseteq E^\circ, G\subseteq E^\circ$.
\end{proof}

\begin{proof}
9d)

Claim: $(E^\circ)^c \subseteq \overline{(E^c)}$.

Claim: $\overline{(E^c)} \subseteq (E^\circ)^c$.
\end{proof}

\begin{proof}
9e)

Pick two open intervals, say $(0,1)\cup(1,2)$. Then 1 is not an interior point, and the closure of the set is $[0,1]\cup[1,2] = [0,2]$, where one is an interior point.

As such, the set and its closure do not necessarily have the same interiors.
\end{proof}


\begin{proof}
9f)

Let $E=\{1\}$.

Then $E^\circ = \emptyset$, $\overline{E^\circ} = \overline{\emptyset} = \emptyset$.

In addition, $\overline{E} = \{1\} \neq \emptyset$.

As such, the set and the set of it's interior points do not necessarily have the same closures.
\end{proof}




\pagebreak




\begin{enumerate}
\item[12.] Let $K\subseteq \R^1$ consist of zero and the numbers 1/$n$ for $n=1,2,3,\dots.$ Prove that $K$ is compact directly from the definition (without using Heine-Borel theorem).
\end{enumerate}

\begin{proof}
Let $\mathcal{C}$ is an open covering of $K$.

Since $0\in K$, there must be some $G_{\alpha,0} \in\mathcal{C}$ that contains zero as well. Since all $G$ are open, there exists $r>0$ such that $N_r (0)\subseteq G_{\alpha,0}$.

Since 0 is the only limit point of $K$ (by the theorem Limit points of $E_p$), $K\backslash N_r(0)$ is finite (by the corollary to the theorem Limit points of $E_p$), and can be can therefore be covered by finite number of subcovers, $\cup_\alpha^n G_\alpha$.

Therefore, $K$ is compact.
\end{proof}




\pagebreak




\begin{enumerate}
\item[13.] Construct a compact set of real numbers whose limit points form a countable set.
\end{enumerate}

\begin{proof}
\textit{Limit points are $\R$ in $(0,1)$, not contained in the union, so the set is not compact. Try again.}

The theorem Limit points of $E_p$ can easily be trivially extended to handle $p\in\Q$, so that the limit point of a set $E_p$ is known to be $p$. \\

Let $E_p$, where $p\in\Q$, be defined as follows:
$$E_p = \left\{\frac{1}{n} + p: n\in\N, n\geq 2\right\}.$$

Let $\{r_1,r_2,\dots\}$ be an indexed set of the rationals such that $0\leq r_i<\frac{1}{2}$. \\

Then the set $A$ such that
$$A=\bigcup_{i=1}^\infty (E_{r_i} \cup {r_i})$$
has limit points $\{r_1,r_2,\dots\},$ which is a set containing countably many points (since there are infinitely many rational numbers between zero and one-half). \\

The set $A$ is bounded on both sides (zero and one respectively), and since it contains it's own limit points, it's closed.\\

Since $A$ is closed and bounded, it is compact.
\end{proof}

% Backup proof of 13
% \begin{proof}
% If we have
% $$E = \left\{\frac{1}{2^n}: n\in\N \right\},$$
% then to fill the gaps between the numbers, we can choose
% $$\delta_n = \frac{\frac{1}{n+1}+\frac{1}{n}}{2},$$
% which gives us the midpoint between two consecutive points in $E$. Then, between those two points:
% $$\frac{1}{n+1} \leq \frac{1}{n+1} + \frac{\delta_n}{2^m} \leq \frac{1}{n}.$$
% \end{proof}




\pagebreak




\begin{enumerate}
\item[14.] Give an example of an open cover of the segment $(0,1)$ which has no finite subcover.
\end{enumerate}

\begin{proof}
\textit{Well Ordering Principle when used to order the rationals does not preserve the less than or greater than operator. Proof does not work. Try again.}

Let $R$ be the indexed list of rationals $\{r_0, r_1, \dots\}$ such that $0\leq r_i < r_{i+1}  1$. \\

Let $E=\cup_{i=0}^\infty (r_i, r_{i+2}).$ Then $E=(0,1)$, since $r_0=0$ and $r_\infty = 1$ (\textit{note: would it be better to write that the sequence $r_i$ approaches 1 as $i\rightarrow\infty$?}). \\

By using the indices $i$ and $i+2$, we create overlap between any two intervals. Should we exclude some interval (like taking the finite union), we will have a gap in our subcover. As such, this open cover has no finite subcover.
\end{proof}




\pagebreak





\begin{enumerate}
\item[15.] Show that Theorem 2.36 and its Corollary become false (in $\R^1$, for example) if the word \say{compact} is replaced by \say{closed} or \say{bounded.}
\end{enumerate}
\setcounter{equation}{2}
\begin{theorem*}
If $\{K_\alpha\}$ is a collection of compact subsets of a metric space $X$ such that the intersection of every finite subcollection of $\{K_\alpha\}$ is nonempty, then $\cap K_\alpha$ is nonempty.
\end{theorem*}

\begin{corollary*}
If $\{K_n\}$ is a sequence of nonempty compact sets such that $K_n \supset K_{n+1}$ ($n=1,2,3,\dots$), then $\cap_1^\infty K_n$ is not empty.
\end{corollary*}

\begin{proof}
Consider the closed set $A_n=[n,\infty)$. For any choice of $n$, this set has no limit point since it is a list of natural numbers, and we can very easily choose any $r < 1$ such that $N_r(p)\cap A_n$ is finite. As such, it contains its own non-existent limit points and is closed. \\

Since the intersection of any finite collection of $A_n$s will have a nonzero number of elements in common, then $\cap A_n$ should be nonempty. However, $\cap_{i=1}^\infty A_n = \emptyset$. \\

Therefore, Theorem 2.36 and its Corollary become false if the word \say{compact} is replaced by \say{closed.}
\end{proof}

\begin{proof}
Consider the bounded set $A_n = (0, \frac{1}{n})$. It is bounded since for any $n$, we can choose a number $M: d(p,q) < M, \forall p,q\in A_n$. \\

Since the intersection of any finite collection of $A_n$ will have a nonzero number of element ins common, then $\cap A_n$ should be nonempty. However, $\cap_{i=1}^\infty A_n = \emptyset$, since as we near $n\rightarrow\infty, \frac{1}{n}\rightarrow 0$, yielding the interval $(0,0)$, which is equivalent to the empty set, and the empty set intersect any set is the empty set. \\

Therefore, Theorem 2.36 and its Corollary become false if the word \say{compact} is replaced by \say{bounded.}
\end{proof}




\pagebreak



\begin{enumerate}
\item[16.] Regard $\Q$, the set of all rational numbers, as a metric space, with $d(p,q)=|p-q|$. Let $E$ be the set of all $p\in \Q$ such that $2<p^2<3$. Show:
\begin{enumerate}
\item $E$ is closed and bounded in $\Q$, but that $E$ is not compact.
\item Whether $E$ is open in $\Q$.
\end{enumerate}
\end{enumerate}

\begin{proof}
16a)


\end{proof}

\begin{proof}
16b)
\end{proof}



\pagebreak



\begin{enumerate}
\item[17.] Let $E = \{x: x\in [0,1]$ and $x$'s decimal expansion contains only the digits 4 and 7$\}$. Show:
\begin{enumerate}
\item Whether $E$ is countable.
\item Whether $E$ is dense in $[0,1]$.
\item Whether $E$ is compact.
\item Whether $E$ is perfect.
\end{enumerate}
\end{enumerate}

\begin{proof}
17a)

$E$ is uncountably infinite since we can create an element that is not in any list containing the elements of $E$ (similar to the example we had done in class to prove that $\R$ is uncountable).
\end{proof}

\begin{proof}
17b)

$E$ is not dense in $[0,1]$ since not every point in $[0,1]$ is a point or limit point of $E$ (for example, $0.1$).
\end{proof}

\begin{proof}
17c)

Claim: $E$ is bounded. \\
Let $M=2,$ and $p,q\in E.$ Since $d(p,q)<2$ for all $p,q\in E$, $E$ is bounded. \\

Claim: $E$ is closed. \\
Let $a,b\in E, a\neq b$. For two numbers to be distinct, their decimal expansions must vary in at least one place. Define $a$ and $b$ as the non-terminating decimal expansions below:
\begin{center}
\begin{tabular}{c}
$a = 0.a_1a_2a_3\dots$ \\
$b = 0.b_1b_2b_3\dots$
\end{tabular}.
\end{center}
Since $a,b\in E$, $a_i,b_i\in\{4,7\}.$ Assume that $a$ and $b$ vary in only once place. Then:
$$|a-b| = \frac{3}{10^i}.$$
As $i\rightarrow\infty$, the difference between the two numbers approaches zero. Since these expansions are non-terminating, we can set the place that they vary farther and farther out. As long as we choose $i:\forall r>0, \frac{3}{10^i} < r,$ then $N_r(a)\cup E$ will contain at least one point $b, b\neq a$: as such, $a$ is a limit point of $E$.
Since $a\in E,$ $E$ is closed. \\

Claim: $E$ is compact. \\

Since $E$ is closed and bounded, it is compact.
\end{proof}

\begin{proof}
17d)

To be a perfect set, every point of $E$ must be a limit point of $E$ (we've already met the criteria that $E$ is closed). \\

Given any $a\in E,\exists b\in E$: $b_i\neq a_i$ as $i\rightarrow\infty$. As such, $\forall r>0$, $b\in N_r(a)\cap E,$ $b\neq a$. Then $a\in E'$. \\

Therefore, every $a\in E$ is a limit point of $E$, and $E$ is perfect.
\end{proof}




\pagebreak




\begin{enumerate}
\item[18.] Is there a nonempty perfect set in $\R^1$ which contains no rational number?
\end{enumerate}

\begin{proof}

\end{proof}




\pagebreak




\begin{enumerate}
\item[19.]
\begin{enumerate}
\item If $A$ and $B$ are disjoint closed sets in some metric space $X$, prove that they are separated.
\item Prove the same for disjoint open sets.
\item Fix $p\in X, \delta >0,$ define $A$ to be the set of all $q\in X$ for which $d(p,q)<\delta$, define $B$ similarly, with $>$ in place of $<$. Prove that $A$ and $B$ are separated.
\item Prove that every connected metric space with at least two points in uncountable. \textit{Hint}: Use (c).
\end{enumerate}
\end{enumerate}

\begin{proof}
19a)

For $A$ and $B$ to be separated, they must be disjoint and share no limit points. Since $A$ and $B$ are disjoint, $A\cap B=\emptyset$. Since $A$ and $B$ are closed, $A=\overline{A}$ and $B=\overline{B}$, $\overline{A}\cap B=\emptyset$, and $A\cap\overline{B}=\emptyset$. Therefore, $A$ and $B$ are separated.
\end{proof}

\begin{proof}
19b)


\end{proof}


\begin{proof}
19c)


\end{proof}


\begin{proof}
19d)


\end{proof}


\end{document} %Where the text for the document ends.
