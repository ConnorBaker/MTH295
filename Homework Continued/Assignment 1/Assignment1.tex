% Author: Connor Baker
% Date Created: May 29, 2017
% Last Edited: July 30, 2017
% Version: 0.3a

% Declare type of document
\documentclass[10pt]{article}

% Import Packages
\usepackage[utf8]{inputenc}
\usepackage[mathscr]{euscript}
\usepackage{amsfonts,amsmath,amssymb,amsthm}
\usepackage{mathtools,mathdots}
\usepackage{enumitem}
\usepackage{array}
\usepackage{longtable}

% Page Formatting
% These settings let you manipulate the margins on the paper, and provide more options than you might be used to using in a word style document.  For example, the settings \oddsidemargin and \evensidemargin are allowed to be adjusted separately in case you are binding a book together.
\topmargin -0.25in \oddsidemargin -.25in \evensidemargin -.25in
\textheight 9in \textwidth 6.75in \headheight 0in \headsep .35in
\parindent 0in

% Define the basic math environments
\theoremstyle{definition}
\newtheorem{definition}[equation]{Definition}
\newtheorem{example}[equation]{Example}
\newtheorem{axiom}[equation]{Axiom}
\theoremstyle{plain}
\newtheorem{theorem}[equation]{Theorem}
\newtheorem{proposition}[equation]{Proposition}
\newtheorem{lemma}[equation]{Lemma}
\newtheorem{corollary}[equation]{Corollary}
\newtheorem{conjecture}[equation]{Conjecture}

% Define frequently used commands
\newcommand{\N}{\mathbb{N}}
\newcommand{\Z}{\mathbb{Z}}
\newcommand{\Q}{\mathbb{Q}}
\newcommand{\R}{\mathbb{R}}
\newcommand{\C}{\mathbb{C}}
\DeclareMathOperator\dom{dom}
\DeclareMathOperator\rang{rang}
\newcommand{\ds}{\displaystyle}

\makeatletter
\def\imod#1{\allowbreak\mkern10mu({\operator@font mod}\,\,#1)}
\makeatother

\begin{document} %This is where we type the text that we plan on having in our document.

% Create the Header
\begin{center}
  {\Large Rudin's \textsc{Principles of Mathematical Analysis, 3rd ed}}

  {\large Connor Baker, June 2017}

  \subsection*{Basic Topology: Selected Exercises}
\end{center}
\begin{enumerate}
\item[5.] Construct a bounded set of real numbers with exactly three limit points.
\end{enumerate}

\begin{lemma}
Consider the set $E$:
$$E = \left\{\frac{1}{n}: n\in\N\right\}.$$
Claim: $|E|=|\N|$. \\

Let $f:E\rightarrow\N$ where $\frac{1}{n} \mapsto n$. We will show that $f$ is one-to-one. \\
Assume $f(a)=f(b)$. Then $a=b$, so $f$ is one-to-one. \\
Choose $n\in\N$. Then there exists exactly one $\frac{1}{n}\in E$ such that $f(\frac{1}{n}) = n$ (since the reciprocal of any natural number is unique relative to the reciprocal of any other number in the naturals). As such, $f$ is onto. \\

Since there is a one-to-one correspondence between $E$ and $\N$, $|E|=|\N|$.
\end{lemma}


\begin{proof}
Let $E$ be the set of numbers such that:
$$E = \bigcup_{k=0}^{2} A_k, \qquad A_k = \left\{\frac{1}{n} + k: n\in\N\right\}.$$

By Lemma 1, we know that $A_0$ has the same cardinality as $N$. Since $\lim_{n\rightarrow\infty} \frac{1}{n} = 0$ the value of $k$ allows us to change the value the limit approaches. As such, every value of $A_1$ is one larger than the corresponding value of $A_0$, as is $A_2$ with respect to $A_1$. The cardinality remains unchanged.

Since the cardinality of $A_k$ is that of the naturals, the neighborhood $N_r(k)$ for some $r>0$ intersect $A_k$ is infinite.

Therefore the set of limit points of $E$ must contain at least the points 0, 1, and 2, and as such, $\{0,1,2\}\subseteq E'$. \\

We now show that $E'$ contains only 0, 1, and 2. \\

Suppose that there is some limit point $x\not\in\{0,1,2\}$. Let $\epsilon = \min(d(x,0),\ d(x,1),\ d(x,2))$ (we want the smallest radius possible). Then the neighborhood
$$N_\frac{\epsilon}{2} (x) = \left\{y: d(x,y) < \frac{\epsilon}{2}\right\}$$
does not contain the set $(0, \frac{\epsilon}{2})\cup(1, 1+\frac{\epsilon}{2})\cup(2, 2+\frac{\epsilon}{2})$. Each interval in that union has no least lower bound, since $\N$ is not bounded above, we can pick larger and larger $n$ for $\frac{1}{n}$, which the interval is composed of. Since this portion of the interval is not included, the neighborhood is finite. Since the neighborhood is finite, the intersection with $E$ is finite and therefore $x$ is not a limit point. \\

As such, $x\not\in E'$, and $E' = \{0,1,2\}$.
\end{proof}



\pagebreak



\begin{enumerate}
\item[6.] Let $E'$ be the set of all limit points of a set $E$. Prove:
\begin{enumerate}
  \item $E'$ is closed.
  \item $E$ and $\bar{E}$ have the same limit points.
  \item Whether or not $E$ and $E'$ always have the same limit points.
\end{enumerate}
\end{enumerate}

\begin{proof}
6a)

Let $p\in E''$. Then, for $r>0$,
$$\exists q\in N_{\frac{r}{2}} (p):q\in E', q\neq p.$$

Let
$$\delta = \frac{d(p,q)}{2} < \frac{r}{2}.$$

Since $q\in E'$, for $r>0$, $\exists s\in N_\delta (q): s\in E$, and  $s\neq q$. Due to the definition of the radius of $N_\delta (q), s\neq p$ since the neighborhood's radius is not large enough to contain $p$.

Then:
$$d(p,q) < \frac{r}{2}$$
$$d(q,s) < \delta$$
$$d(p,s) \leq d(p,q) + d(q,s) < \frac{r}{2} + \delta < \frac{r}{2} + \frac{r}{2} = r.$$

So, $p\in E'$, and $E'$ is closed.
\end{proof}

\begin{proof}
6b)

Proof that $E$ and $\bar{E}$ have the same limit points. \\

By the previous proof, $E'$ is closed and therefore contains its own limit points. \\

Let $\bar{E}'$ be the set of all limit points of $\bar{E}$. \\

We begin by showing that $E'\subseteq \bar{E}'$. \\

Let $x\in E'$. Then $x\in  $

\end{proof}

\begin{proof}
 6c)

  Consider the set
  $$E = \left\{\frac{1}{n}: n\in\N\right\}.$$

  Since $\lim_{n\rightarrow\infty} \frac{1}{n} = 0$, and $|E|=|\N|$ (by Lemma 1), $N_r (0)\cap E$ has infinitely many points, zero must be a limit point of $E$, so $E' = \{0\}$. \\

  However, the limit points of $E'$ are the empty set, since no matter the point we pick to center the neighborhood, the intersection will contain at most one element (zero). As such, $E$ and $E'$ do not always have the same limit points. \\
\end{proof}



\pagebreak



\begin{enumerate}
\item[7.] Let $A_1,A_2,A_3,\dots$ be subsets of a metric space.
\begin{enumerate}
  \item If $B_n = \cup_{i=1}^{n} A_i,$ prove that $\bar{B}_n = \cup_{i=1}^{n} \bar{A}_i,$ for $n=1,2,3,\dots$.
  \item If $B = \cup_{i=1}^{\infty} A_i,$ prove that $\bar{B} \supset \cup_{i=1}^{\infty} \bar{A}_i,$ and show by an example that this inclusion can be proper.
\end{enumerate}

\end{enumerate}

\begin{lemma}
$\overline{A\cup B} = \overline{A}\cup\overline{B}.$
\end{lemma}

\begin{proof}
6a)

Claim: $(A\cup B)' \subseteq A' \cup B'.$

Case 1: The trivial case is that $A\cup B$ is finite. Then $(A\cup B)'$ is the empty set, since finite sets cannot have limit points, which is a subset of any set so our claim holds true.

Case 2: $A\cup B$ is not finite. Then let $x\in(A\cup B)'$. Choose any $r>0$, and consider $N_r (x) \cap A$, and $N_r (x) \cap B$. Suppose both are finite. Then $N_r (x) \cap (A\cup B)$ is finite, and $x$ is not a limit point. This is a contradiction, so our assumption that both are finite is incorrect. Therefore, at least one must be infinite, so $N_r (x) \cap A$ or $N_r (x) \cap B$ must be infinite, and therefore $x\in A'\cup B'.$

Claim: $A' \cup B' \subseteq (A\cup B)'.$

Let $x\in A'\cup B'.$ Choose any $r>0$. Then, $N_r (x) \cap A$ or $N_r (x) \cap B$  is infinite, and $x\in(A\cup B)'$.

Therefore, $(A\cup B)' = A' \cup B',$ and
$$\overline{A\cup B} = (A\cup B)\cup(A\cup B)' = A\cup A' \cup B \cup B' = \overline{A}\cup\overline{B}.$$
\end{proof}

\begin{proof}
6a) Cont.

We proceed with proof by induction.

Let $n = 1.$ Then:
$$\overline{\bigcup^1_{i=1} A_i} = \overline{A_1}=\bigcup^1_{i=1} \overline{A_i}.$$

Let $n=k$. Assume that
$$\overline{\bigcup^k_{i=1} A_i} = \bigcup^k_{i=1} \overline{A_i},$$ for all $k\in\N$.

Let $n=k+1$. Then,
$$\overline{\bigcup^{k+1}_{i=1} A_i} = \overline{\bigcup^k_{i=1} A_i \cup A_{k+1}},$$
which by the lemma above is equivalent to (using our assumption for $n=k$)
$$\overline{\bigcup^k_{i=1} A_i} \cup \overline{A_{k+1}} = \bigcup^k_{i=1} \overline{A_i} \cup \overline{A_{k+1}} = \bigcup^{k+1}_{i=1} \overline{A_i}.$$

Therefore
$$B_n = \bigcup^n_{i=1} A_i, \qquad \overline{B_n} = \overline{\bigcup^n_{i=1} A_i} = \bigcup^n_{i=1} \overline{A_i}$$
by the Principle of Mathematical Induction.
\end{proof}



\pagebreak



\begin{enumerate}
\item[8.] Is every point of every open set $E \subset \R^2$ a limit point of $E$? Answer the same question for closed sets in $\R^2$.
\end{enumerate}

\begin{proof}
Since $E$ is open, all elements of $E$ are interior points. \\

Let $p\in E$. Choose $r>0$, and let $N_r(p)$ be a neighborhood of $p$. Since $p$ is an interior point, $\exists \delta > 0: N_\delta (p) \subseteq E$. \\

Choose $0<\epsilon \leq \min(r,\delta)$. Then, $N_\epsilon (p) \subseteq N_\delta (p) \subseteq E$. Since $N_\epsilon (p)$ contains infinitely many points of $E$, and $N_\epsilon (p) \subseteq N_r (p)$, $N_r(p)$ does as well. \\

Closed sets, by definition contain all of their own limit points, so any closed subset of $\R^2$ will contain its own limit points. However, it is not the case that any finite set of $E$ will have limit points since the neighborhood intersect $E$ will be finite.
\end{proof}




\pagebreak




\begin{enumerate}
\item[9.] Let $E^\circ$ denote the set of all interior points of a set $E$.
\begin{enumerate}
  \item Prove that $E^\circ$ is always open.
  \item Prove that $E$ is open if and only if $E^\circ = E$.
  \item If $G \subset E$ and $G$ is open, prove that $G\subset E^\circ$.
  \item Prove that the complement of $E^\circ$ is the closure of the complement of $E$.
  \item Do $E$ and $\bar{E}$ always have the same interiors?
  \item Do $E$ and $E^\circ$ always have the same closures?
\end{enumerate}
\end{enumerate}

\begin{proof}
9a)
\end{proof}

\begin{proof}
9b)
\end{proof}

\begin{proof}
9c)

Let $p\in G$, where $G$ is an open set. Since $G$ is open, $p$ is an interior point of $G$. Choose any $r>0$, and $N_r (p) \subseteq G.$

\texttt{Next step: Show that $N_r(p)\subseteq E$, so that $p\in E$, so $G\subseteq E^\circ$.}
\end{proof}

\begin{proof}
9d)

Claim: $(E^\circ)^c \subseteq \overline{(E^c)}$. Since $E^\circ \subseteq E$ (why exactly is this?), $(E^\circ)^c$ is closed since it is the complement of an open set, and is equal to $(E^\circ)^c$, which is a super-set of $\overline{(E^c)}.$

Claim: $\overline{(E^c)} \subseteq (E^\circ)^c$.
\end{proof}

\begin{proof}
9e)

Pick two open intervals, say $(0,1)\cup(1,2)$. Then 1 is not an interior point, and the closure of the set is $([0,1]\cup[1,2] = [0,2]$, where one is an interior point.

As such, the set and its closure do not necessarily have the same interiors.
\end{proof}


\begin{proof}
9f)

Let $E=\{1\}$.

Then $E^\circ = \emptyset$, $\overline{E^\circ} = \overline{\emptyset} = \emptyset$.

In addition, $\overline{E} = \{1\} \neq \emptyset$.

As such, the set and the set of it's interior points do not necessarily have the same closures.
\end{proof}




\pagebreak




\begin{enumerate}
\item[12.] Let $K\subseteq \R^1$ consist of zero and the numbers 1/$n$ for $n=1,2,3,\dots.$ Prove that $K$ is compact directly from the definition (without using Heine-Borel theorem).
\end{enumerate}

\begin{proof}
Suppose that $K\subseteq \mathcal{C}$, where $\mathcal{C}$ is an open covering of $K$. Since $\mathcal{C}$ is an open covering, it is the set of $\cup_\alpha G_\alpha$ where $K\subseteq \cup_\alpha G_\alpha$.

Since $0\in K$, there must be some $G_{\alpha,0}$ that contains zero as well. Since all $G$ are open, there exists $r>0$ such that $N_r (0)\subseteq G_{\alpha,0}$.

Since 0 is the only limit point of $K$ (by Lemma 1), $K\backslash N_r(0)$ is finite, and can be can therefore be covered by finite number of subcovers, $\cup_\alpha^n G_\alpha$.

Therefore, $K$ is compact.
\end{proof}




\pagebreak




\begin{enumerate}
\item[13.] Construct a compact set of real numbers whose limit points form a countable set.
\end{enumerate}

\begin{proof}

\end{proof}




\pagebreak




\begin{enumerate}
\item[14.] Give an example of an open cover of the segment $(0,1)$ which has no finite subcover.
\end{enumerate}

\begin{proof}

\end{proof}




\pagebreak





\begin{enumerate}
\item[15.] Show that Theorem 2.36 and its Corollary become false (in $\R^1$, for example) if the word "compact" is replaced by "closed" or "bounded".
\end{enumerate}
\setcounter{equation}{2}
\begin{theorem}
If $\{K_\alpha\}$ is a collection of compact subsets of a metric space $X$ such that the intersection of every
\end{theorem}
\begin{proof}

\end{proof}




\pagebreak



\begin{enumerate}
\item[16.] Regard $\Q$, the set of all rational numbers, as a metric space, with $d(p,q)=|p-q|$. Let $E$ be the set of all $p\in \Q$ such that $2<p^2<3$. Show:
\begin{enumerate}
\item $E$ is closed and bounded in $\Q$, but that $E$ is not compact.
\item Whether $E$ is open in $\Q$.
\end{enumerate}
\end{enumerate}

\begin{proof}
16a)
\end{proof}

\begin{proof}
16b)
\end{proof}



\pagebreak



\begin{enumerate}
\item[17.] Let $E = \{x: x\in [0,1]$ and $x$'s decimal expansion contains only the digits 4 and 7$\}$. Show:
\begin{enumerate}
\item Whether $E$ is countable.
\item Whether $E$ is dense in $[0,1]$.
\item Whether $E$ is compact.
\item Whether $E$ is perfect.
\end{enumerate}
\end{enumerate}

\begin{proof}
17a)
\end{proof}

\begin{proof}
17b)
\end{proof}

\begin{proof}
17c)
\end{proof}

\begin{proof}
17d)
\end{proof}




\pagebreak




\begin{enumerate}
\item[18.] Is there a nonempty perfect set in $\R^1$ which contains no rational number?
\end{enumerate}

\begin{proof}

\end{proof}




\pagebreak




\begin{enumerate}
\item[19.]
\begin{enumerate}
\item If $A$ and $B$ are disjoint closed sets in some metric space $X$, prove that they are separated.
\item Prove the same for disjoint open sets.
\item Fix $p\in X, \delta >0,$ define $A$ to be the set of all $q\in X$ for which $d(p,q)<\delta$, define $B$ similarly, with $>$ in place of $<$. Prove that $A$ and $B$ are separated.
\item Prove that every connected metric space with at least two points in uncountable. \textit{Hint}: Use (c).
\end{enumerate}
\end{enumerate}

\begin{proof}

\end{proof}


\end{document} %Where the text for the document ends.
