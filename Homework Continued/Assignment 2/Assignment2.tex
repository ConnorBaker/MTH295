% Author: Connor Baker
% Date Created: August 4, 2017
% Last Edited: August 26, 2017
% Version: 0.1b

% Declare type of document
\documentclass[10pt]{article}

% Import Packages
\usepackage[utf8]{inputenc}
\usepackage[mathscr]{euscript}
\usepackage{amsfonts,amsmath,amssymb,amsthm}
\usepackage{mathtools,mathdots}
\usepackage{enumitem}
\usepackage{array}
\usepackage{tabu}
\usepackage{tabularx}
\usepackage{longtable}
\usepackage{dirtytalk}

% Page Formatting
% These settings let you manipulate the margins on the paper, and provide more options than you might be used to using in a word style document.  For example, the settings \oddsidemargin and \evensidemargin are allowed to be adjusted separately in case you are binding a book together.
\topmargin -0.25in \oddsidemargin -.25in \evensidemargin -.25in
\textheight 9in \textwidth 6.75in \headheight 0in \headsep .35in
\parindent 0in

% Define the basic math environments
\theoremstyle{definition}
\newtheorem{definition}[equation]{Definition}
\newtheorem{example}[equation]{Example}
\newtheorem{axiom}[equation]{Axiom}
\theoremstyle{plain}
\newtheorem{theorem}[equation]{Theorem}
\newtheorem*{theorem*}{Theorem}
\newtheorem{proposition}[equation]{Proposition}
\newtheorem{lemma}[equation]{Lemma}
\newtheorem*{lemma*}{Lemma}
\newtheorem{corollary}[equation]{Corollary}
\newtheorem*{corollary*}{Corollary}
\newtheorem{conjecture}[equation]{Conjecture}

% Define frequently used commands
\newcommand{\N}{\mathbb{N}}
\newcommand{\Z}{\mathbb{Z}}
\newcommand{\Q}{\mathbb{Q}}
\newcommand{\R}{\mathbb{R}}
\newcommand{\C}{\mathbb{C}}
\DeclareMathOperator\dom{dom}
\DeclareMathOperator\rang{rang}
\newcommand{\ds}{\displaystyle}

\makeatletter
\def\imod#1{\allowbreak\mkern10mu({\operator@font mod}\,\,#1)}
\makeatother

\begin{document} %This is where we type the text that we plan on having in our document.

% Create the Header
\begin{center}
  {\Large Rudin's \textsc{Principles of Mathematical Analysis, 3rd ed}}

  {\large Connor Baker, June 2017}

  \subsection*{Numerical Sequences and Series: Selected Exercises}
\end{center}

\begin{enumerate}
\item[1.] Prove that convergence of $\{s_n\}$ implies convergence of $\{|s_n|\}$. Is the converse true?
\end{enumerate}

\begin{proof}
1a) \\

Assume that $\{s_n\}$ is a convergent Cauchy sequence, and let $\epsilon > 0$. Since $\{s_n\}$ is a Cauchy sequence, for every $\epsilon > 0, \exists N\in\Z$ such that $|s_n-s_m|<\epsilon, \forall m,n\geq N$. \\

Claim: $||p|-|q|| \leq |p-q|.$ \\

Case 1: $p \geq 0, q\geq 0$. \\
Then $||p|-|q|| = |p - q|$. \\

Case 2: $p \leq 0, q\leq 0$. \\
Then $||p|-|q|| = |(-p) - (-q)| = |q - p| = |-1(p-q)| = |-1|\cdot|p-q| = |p-q|$. \\

Case 3: $p \leq 0, q\geq 0$. \\
Then $||p|-|q|| = |(-p) - q| =|-1(p+q)| = |-1|\cdot|p+q| = |p+q| \leq |p-q|$,  since subtracting a non-negative number from a non-positive number increases the magnitude of the result more so than adding a non-negative number to a non-positive number. \\

Case 4: $p \geq 0, q\leq 0$. \\
Then $||p|-|q|| = |p - (-q)| =|p+q| \leq |p-q|$, since subtracting a non-positive number from a non-negative number increases the magnitude of the result more so than adding a non-positive number to a non-negative number. \\


As such, $||s_n| - |s_m||\leq |s_n - s_m| < \epsilon$ for all $m,n\geq N$. \\

Therefore, $\{|s_n|\}$ is a convergent Cauchy sequence.
\end{proof}

\begin{proof}
1b) \\

Since $||p| - |q|| \leq |p-q|,$ the converse is not true -- having $||s_n|-|s_m|| < \epsilon, \forall m,n\geq N$ tells us nothing about whether $|s_n-s_m| < \epsilon, \forall m,n\geq N$ is true.
\end{proof}


\pagebreak



\begin{enumerate}
\item[2.] Calculate $\lim_{n\rightarrow\infty} (\sqrt{n^2+n}-n)$.
\end{enumerate}

\begin{proof}
We begin by multiplying by the algebraic conjugate:
$$\lim_{n\rightarrow\infty} \left[(\sqrt{n^2+n}-n) \cdot \frac{\sqrt{n^2+n}+n}{\sqrt{n^2+n}+n}\right]=\lim_{n\rightarrow\infty} \left[ \frac{n^2+n-n^2}{\sqrt{n^2+n}+n}\right] = \lim_{n\rightarrow\infty} \left[ \frac{n}{\sqrt{n^2+n}+n}\right].$$
Tentatively trying to evaluate the limit yields a composition of algebraic indeterminate forms:
$$\lim_{n\rightarrow\infty} \left[\frac{n}{\sqrt{n^2+n}+n}\right] = \frac{\infty}{\sqrt{\infty}+\infty} = \frac{\infty}{\infty+\infty}.$$
We proceed by multiplying by a form of one:
$$\lim_{n\rightarrow\infty} \left[\frac{n}{\sqrt{n^2+n}+n} \cdot \frac{\frac{1}{n}}{\frac{1}{n}}\right] = \lim_{n\rightarrow\infty}  \left[\frac{\frac{n}{n}}{\sqrt{\frac{n^2}{n^2}+\frac{n}{n^2}}+\frac{n}{n}}\right]=\lim_{n\rightarrow\infty}  \left[\frac{1}{\sqrt{1+\frac{1}{n}}+1}\right].$$
From this point, we can now successfully evaluate the limit.
$$\lim_{n\rightarrow\infty}  \left[\frac{1}{\sqrt{1+\frac{1}{n}}+1}\right] = \frac{1}{\sqrt{1+\frac{1}{\infty}}+1} = \frac{1}{\sqrt{1+0}+1} = \frac{1}{2}.$$
\end{proof}



\pagebreak



\begin{enumerate}
\item[3.] If $s_1 = \sqrt{2},$ and
$$s_{n+1} =\sqrt{2+s_n}, (n=1,2,3,\dots),$$
prove that $\{s_n\}$ converges, and that $s_n<2$ for $n=1,2,3,\dots$.
\end{enumerate}

\begin{proof}
Claim: $s_n < 2, \forall n\in\N$. \\

Base case: Let $n=1$. Then $s_1=\sqrt{2}<2$ and the base case holds. \\

Inductive hypothesis: Let $n=k$. Assume that for all $n\leq k$, $s_k < 2.$ \\

Inductive step: Let $n=k+1$. Then $s_{k+1} = \sqrt{2+s_k}$, which, by our inductive hypothesis, is less than $\sqrt{2+2}=\sqrt{4}=2$. \\

By the Principle of Mathematical Induction, $s_n < 2, \forall n\in\N$. As such, $s_n$ is bounded above. \\

Claim: $s_n<s_{n+1}, \forall n\in\N.$ \\

Base case: Let $n=1$. Then $s_1=\sqrt{2}<\sqrt{2+\sqrt{2}} = s_{2}$ and the base case holds. \\

Inductive hypothesis: Let $n=k$. Assume that for all $n\leq k$, $s_k < s_{k+1}.$ \\

Inductive step: Let $n=k+1$. Then $s_{k+1} = \sqrt{2+s_k}$, which, by our inductive hypothesis, is less than $\sqrt{2+s_{k+1}}=s_{k+2}$. \\

By the Principle of Mathematical Induction, $s_n < s_{n+1}, \forall n\in\N$. As such, $s_n$ is monotonic increasing. \\

Since $s_n$ is both bounded above and monotonic increasing, by the monotone convergence theorem, $s_n$ converges.

\end{proof}



\pagebreak



\begin{enumerate}
\item[4.] Find the upper and lower limits of the sequence $\{s_n\}$ defined by
$$s_1=0; \quad s_{2m}=\frac{s_{2m-1}}{2}; \quad s_{2m+1}=\frac{1}{2} + s_{2m}.$$
\end{enumerate}

\begin{proof}
We begin by trying to find the closed form of the sequence $\{s_n\}.$
\[
\renewcommand{\arraystretch}{2}
\begin{tabu}{|c|c|c|}
\hline
m & s_{2m} & s_{2m+1} \\ \hline
1 & s_2 = \frac{0}{2} & s_3 = \frac{1}{2}+0 = \frac{1}{2} \\ \hline
2 & s_4 = \frac{\frac{1}{2}}{2}  = \frac{1}{2^2} & s_5 = \frac{1}{2}+\frac{1}{2^2} = \frac{2+1}{2^2} \\ \hline
3 & s_6 = \frac{\frac{2+1}{2^2}}{2}  = \frac{2+1}{2^3} & s_7 = \frac{1}{2}+\frac{2+1}{2^3} = \frac{2^2+2+1}{2^3} \\ \hline
\vdots & \vdots & \vdots \\ \hline
m & s_{2m}=\frac{\sum{2^k}^{m-2}_{k=0}}{2^m} & s_{2m+1}=\frac{\sum{2^k}^{m-1}_{k=0}}{2^m} \\ \hline
\end{tabu}
\]
We then try to find the closed forms of the sums that we're using -- since these are geometric series, we can use the formula that follows:
$$\sum_{k}^{n} ar^k = \frac{a(1-r^n)}{1-r}.$$
For our series, $a=1$, and $r=2$, so:
$$\frac{a(1-2^n)}{1-2} = \frac{1-2^n}{-1} = 2^n-1.$$

As such, we now have:
$$s_{2m} = \frac{\sum{2^k}^{m-2}_{k=0}}{2^m} =\frac{2^{m-1}-1}{2^m} = \frac{1}{2} - \frac{1}{2^m},$$
and
$$s_{2m+1} = \frac{\sum{2^k}^{m-1}_{k=0}}{2^m} = \frac{2^m-1}{2^m} = 1 - \frac{1}{2^m}.$$

Claim: The closed forms of $s_{2m}$ and $s_{2m+1}$ we have derived hold $\forall m\in\N$. We proceed with induction. \\

Base case: Let $m=1$. Then:
$$s_{2} = \frac{s_1}{2} = \frac{0}{2} = 0,$$
and
$$\frac{2^{1-1}-1}{2^1} = \frac{0}{2} = 0 = s_2.$$
In addition,
$$s_{3} = \frac{s_2}{2} = \frac{1}{2} + s_2 = \frac{1}{2} + 0 = \frac{1}{2},$$
and
$$\frac{2^{1}-1}{2^1} = \frac{1}{2} = s_3.$$

Assume that $\forall m \leq r$, our closed forms of $s_{2m}$ and $s_{2m+1}$ hold. Then
$$s_{2r} = \frac{1}{2} - \frac{1}{2^r}; \quad s_{2r+1} = 1 - \frac{1}{2^r}.$$

Let $m=r+1$. Then
$$s_{2(r+1)} = s_{2r+2} = \frac{s_{2r+1}}{2} = \frac{1-\frac{1}{2^r}}{2} = \frac{1}{2} - \frac{1}{2^{r+1}},$$
and
$$s_{2(r+1)+1} = s_{2r+3} = \frac{1}{2} + s_{2r+2} = \frac{1}{2} + \frac{1}{2} -\frac{1}{2^{r+1}} = 1-\frac{1}{2^{r+1}}.$$

By the Principle of Mathematical Induction, $\forall m \in\N$,
$$s_{2m} = \frac{1}{2} - \frac{1}{2^m}; \quad s_{2m+1} = 1 - \frac{1}{2^m}$$
hold. \\

We now look at what the sequences $s_{2m}$ and $s_{2m+1}$ converge to:
$$\lim_{m\rightarrow\infty} s_{2m} = \lim_{m\rightarrow\infty} \frac{1}{2} - \frac{1}{2^m} = \frac{1}{2},$$
$$\lim_{m\rightarrow\infty} s_{2m+1} = \lim_{m\rightarrow\infty} 1 - \frac{1}{2^m} = 1.$$
Since the sequence converges, any sub-sequence must converge to the same point. As such, when we consider the limit supremum and infimum -- that is, the supremum and infimum of all subsequenes, they must converge to the same point that the limit of the sequence converges to. As such,
$$\lim_{m\rightarrow\infty}\inf s_{m} = \frac{1}{2},$$
$$\lim_{m\rightarrow\infty}\sup s_{m} = 1.$$

\end{proof}



\pagebreak



\begin{enumerate}
\item[5.] For any two real sequences $\{a_n\}, \{b_n\},$ prove that
$$\lim_{n\rightarrow\infty} \sup(a_n+b_n)\leq \lim_{n\rightarrow\infty} \sup(a_n) + \lim_{n\rightarrow\infty} \sup(b_n)$$
provided the sum on the right is not of the form $\infty-\infty$.
\end{enumerate}

\begin{proof}

\end{proof}



\pagebreak



\begin{enumerate}
\item[6.] Investigate the behavior (convergence or divergence) of $\sum a_n$ if
\begin{enumerate}
\item $a_n = \sqrt{n+1} - \sqrt{n}.$
\item $a_n = \frac{\sqrt{n+1} - \sqrt{n}}{n}.$
\item $a_n = (\sqrt[n]{n} - 1)^n.$
\end{enumerate}
\end{enumerate}
A note on notation: An easy test to see if a series diverges is to check if  $lim_{n\rightarrow\infty} \sup a_n \neq lim_{n\rightarrow\infty} \inf a_n$. If the series converges, then the two are equal, and we do not need to use the $\sup$ and $\inf$ notation. Below, we do not use $\sup$ and $\inf$ because either the series converges and it was needless, or it does not and we are done with the problem.
\begin{proof}
6a) \\

We begin by finding a more useful form of $a_n$:
$$a_n = \sqrt{n+1} - \sqrt{n} = (\sqrt{n+1} - \sqrt{n}) \cdot \frac{\sqrt{n+1} + \sqrt{n}}{\sqrt{n+1} + \sqrt{n}} = \frac{n+1-n}{\sqrt{n+1} + \sqrt{n}} = \frac{1}{\sqrt{n+1} + \sqrt{n}}.$$
Since $\sqrt{n} < \sqrt{n+1}$,
$$\frac{1}{\sqrt{n+1} + \sqrt{n}} > \frac{1}{\sqrt{n+1} + \sqrt{n+1}} = \frac{1}{2\sqrt{n+1}}.$$
Since we know that $\frac{1}{\sqrt{n+1}}$ diverges (it is a $p$-series where $p \leq 1$), we can use the limit comparison test.
$$\lim_{n\rightarrow\infty} \left(\frac{\frac{1}{2\sqrt{n+1}}}{\frac{1}{\sqrt{n+1}} }\right) = \frac{1}{2}.$$
Since $0<\frac{1}{2}<\infty$, by the limit comparison test, $a_n$ diverges.
\end{proof}

\begin{proof}
6b) \\

We begin by finding a more useful form of $a_n$:
$$a_n = \frac{\sqrt{n+1} - \sqrt{n}}{n} = (\frac{\sqrt{n+1} - \sqrt{n}}{n}) \cdot \frac{\sqrt{n+1} + \sqrt{n}}{\sqrt{n+1} + \sqrt{n}} = \frac{n+1-n}{n(\sqrt{n+1} + \sqrt{n})} = \frac{1}{n(\sqrt{n+1} + \sqrt{n})}.$$
Since $\sqrt{n}<\sqrt{n+1}$,
$$\frac{1}{n(\sqrt{n+1} + \sqrt{n})} < \frac{1}{n(\sqrt{n} + \sqrt{n})} = \frac{1}{2n\sqrt{n})} = \frac{1}{2n^{3/2}}.$$
To more clearly see that the sum of our newly created sequence is a $p$-series, we can re-write it as such:
$$\sum^{\infty} \left(\frac{1}{2n^{3/2}}\right) = \frac{1}{2} \cdot \sum^{\infty} \left(\frac{1}{n^{3/2}}\right).$$
Since $p = \frac{3}{2},$ and $p > 1,$ the sum must converge. Furthermore, since $a_n$ is smaller than the sequence which when summed converges, $a_n$ must also sum when converged.
\end{proof}

\begin{proof}
6c) \\

Let $\alpha = \lim_{n\rightarrow\infty} (\sqrt[n]{(\sqrt[n]{n} - 1)^n}).$ Then
$$\alpha = \lim_{n\rightarrow\infty} (\sqrt[n]{n} - 1) = \lim_{n\rightarrow\infty} (\sqrt[n]{n}) - \lim_{n\rightarrow\infty} (1) = \lim_{n\rightarrow\infty} (e^{\ln(\sqrt[n]{n})}) - 1 = \lim_{n\rightarrow\infty} (e^{\frac{\ln(n)}{n}}) - 1.$$
Plugging in for $n$, we arrive the basic indeterminate form $\frac{\infty}{\infty}$, so we can proceed with L'Hôpital's rule.
$$\alpha = \lim_{n\rightarrow\infty} (e^{\frac{1}{n}}) - 1 = e^0 - 1 = 1 - 1 = 0.$$
Since $\alpha = 0$, and $\alpha < 1$, by the root test $\sum a_n$ converges.
\end{proof}




\pagebreak



\begin{enumerate}
\item[7.] Prove that the convergence of $\sum a_n$ implies the convergence of
$$\sum \left(\frac{\sqrt{a_n}}{n}\right),$$
if $a_n \geq 0$.
\end{enumerate}

\begin{proof}
Since $\sum a_n$ converges, $\exists m\in \N: \forall n\geq m, a_n = 0.$ Furthermore, $\exists N\in \N: \forall n\in\N, s_n \in [0,N]$.
Take $s_{m-1}$. We know that $\exists p\in \R:$
$$s_{m-1} = \sum_n^{m-1} \frac{\sqrt{a_n}}{n} = p.$$

Claim: $s_\infty - s_{m-1} = 0$. \\
It is clear that
$$s_\infty - s_{m-1} = \sum_{n=m}^{\infty} \frac{\sqrt{a_n}}{n}.$$
Since $n\geq m$, $a_n = 0$ and the sum is zero. then, $s_\infty = s_{m-1} = p$, and the sum converges to $p$.
\end{proof}



\pagebreak



\begin{enumerate}
\item[11.] Suppose $a_n > 0$, $s_n = a_1+a_2+\cdots+a_n$, and $\sum a_n$ diverges.
\begin{enumerate}
\item Prove that $\sum\frac{a_n}{1+ a_n}$ diverges.
\item Prove that
$$\frac{a_{N+1}}{s_{N+1}} + \cdots + \frac{a_{N+k}}{s_{N+k}} \geq 1- \frac{s_{N}}{s_{N+k}}$$
and that $\sum\frac{a_n}{s_n}$ diverges.
\item Prove that
$$\frac{a_n}{s^2_n}\leq \frac{1}{s_{n-1}} - \frac{1}{s_n}$$
and that $\sum\frac{a_n}{s^2_n}$ converges.
\item What can be said about
$$\sum\frac{a_n}{1+n\cdot a_n} \quad\text{and} \quad \sum\frac{a_n}{1+n^2 \cdot a_n}?$$
\end{enumerate}
\end{enumerate}

\begin{proof}
11a) \\

By re-writing the sum, we see that
$$\sum \left(\frac{a_n}{1+ a_n}\right) = \sum\left(\frac{1}{(\frac{1+ a_n}{a_n})}\right) = \sum\left(\frac{1}{(\frac{1}{a_n}+ 1)}\right).$$
Since $a_n > 0, \exists N\in\N: N \geq \frac{1}{a_n},$ by the Archimedian principle. As such, $\frac{1}{a_n}\in [0,N].$ Then, $(\frac{1}{a_n}+1)\in [1,N+1],$ and $$\left(\frac{1}{(\frac{1}{a_n}+1)}\right)\geq \frac{1}{N+1}.$$
By the divergence theorem, since the terms of this infinite sum are positive, the sum must diverge.
\end{proof}


\begin{proof}
11b) \\

\end{proof}


\begin{proof}
11c) \\

\end{proof}


\begin{proof}
11d) \\

\end{proof}


\end{document}
