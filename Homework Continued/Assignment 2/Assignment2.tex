% Author: Connor Baker
% Date Created: August 4, 2017
% Last Edited: August 4, 2017
% Version: 0.1a

% Declare type of document
\documentclass[10pt]{article}

% Import Packages
\usepackage[utf8]{inputenc}
\usepackage[mathscr]{euscript}
\usepackage{amsfonts,amsmath,amssymb,amsthm}
\usepackage{mathtools,mathdots}
\usepackage{enumitem}
\usepackage{array}
\usepackage{longtable}
\usepackage{dirtytalk}

% Page Formatting
% These settings let you manipulate the margins on the paper, and provide more options than you might be used to using in a word style document.  For example, the settings \oddsidemargin and \evensidemargin are allowed to be adjusted separately in case you are binding a book together.
\topmargin -0.25in \oddsidemargin -.25in \evensidemargin -.25in
\textheight 9in \textwidth 6.75in \headheight 0in \headsep .35in
\parindent 0in

% Define the basic math environments
\theoremstyle{definition}
\newtheorem{definition}[equation]{Definition}
\newtheorem{example}[equation]{Example}
\newtheorem{axiom}[equation]{Axiom}
\theoremstyle{plain}
\newtheorem{theorem}[equation]{Theorem}
\newtheorem*{theorem*}{Theorem}
\newtheorem{proposition}[equation]{Proposition}
\newtheorem{lemma}[equation]{Lemma}
\newtheorem*{lemma*}{Lemma}
\newtheorem{corollary}[equation]{Corollary}
\newtheorem*{corollary*}{Corollary}
\newtheorem{conjecture}[equation]{Conjecture}

% Define frequently used commands
\newcommand{\N}{\mathbb{N}}
\newcommand{\Z}{\mathbb{Z}}
\newcommand{\Q}{\mathbb{Q}}
\newcommand{\R}{\mathbb{R}}
\newcommand{\C}{\mathbb{C}}
\DeclareMathOperator\dom{dom}
\DeclareMathOperator\rang{rang}
\newcommand{\ds}{\displaystyle}

\makeatletter
\def\imod#1{\allowbreak\mkern10mu({\operator@font mod}\,\,#1)}
\makeatother

\begin{document} %This is where we type the text that we plan on having in our document.

% Create the Header
\begin{center}
  {\Large Rudin's \textsc{Principles of Mathematical Analysis, 3rd ed}}

  {\large Connor Baker, June 2017}

  \subsection*{Numerical Sequences and Series: Selected Exercises}
\end{center}

\begin{enumerate}
\item[1.] Prove that convergence of $\{s_n\}$ implies convergence of $\{|s_n|\}$. Is the converse true?
\end{enumerate}

\begin{proof}

\end{proof}



\pagebreak



\begin{enumerate}
\item[2.] Calculate $\lim_{n\rightarrow\infty} (\sqrt{n^2+n}-n)$.
\end{enumerate}

\begin{proof}
We begin by multiplying by the algebraic conjugate:
$$\lim_{n\rightarrow\infty} \left[(\sqrt{n^2+n}-n) \cdot \frac{\sqrt{n^2+n}+n}{\sqrt{n^2+n}+n}\right]=\lim_{n\rightarrow\infty} \left[ \frac{n^2+n-n^2}{\sqrt{n^2+n}+n}\right] = \lim_{n\rightarrow\infty} \left[ \frac{n}{\sqrt{n^2+n}+n}\right].$$
Tentatively trying to evaluate the limit yields a composition of algebraic indeterminate forms:
$$\lim_{n\rightarrow\infty} \left[\frac{n}{\sqrt{n^2+n}+n}\right] = \frac{\infty}{\sqrt{\infty}+\infty} = \frac{\infty}{\infty+\infty}.$$
We proceed by multiplying by a form of one:
$$\lim_{n\rightarrow\infty} \left[\frac{n}{\sqrt{n^2+n}+n} \cdot \frac{\frac{1}{n}}{\frac{1}{n}}\right] = \lim_{n\rightarrow\infty}  \left[\frac{\frac{n}{n}}{\sqrt{\frac{n^2}{n^2}+\frac{n}{n^2}}+\frac{n}{n}}\right]=\lim_{n\rightarrow\infty}  \left[\frac{1}{\sqrt{1+\frac{1}{n}}+1}\right].$$
From this point, we can now successfully evaluate the limit.
$$\lim_{n\rightarrow\infty}  \left[\frac{1}{\sqrt{1+\frac{1}{n}}+1}\right] = \frac{1}{\sqrt{1+\frac{1}{\infty}}+1} = \frac{1}{\sqrt{1+0}+1} = \frac{1}{2}.$$
\end{proof}



\pagebreak



\begin{enumerate}
\item[3.] If $s_1 = \sqrt{2},$ and
$$s_{n+1} =\sqrt{2+s_n}, (n=1,2,3,\dots),$$
prove that $\{s_n\}$ converges, and that $s_n<2$ for $n=1,2,3,\dots$.
\end{enumerate}

\begin{proof}
We begin by proving that $s_n<2, \forall n\in\N$. \\

Let $n=1.$ Then $s_1=\sqrt{2}<2$, and the base case holds. \\

Inductive hypothesis: Assume that for $n\leq k,$ the following is true: $s_k < 2$. \\

Let $n=k+1$. Then $s_{k+1} = \sqrt{2\cdot s_k} = \sqrt{2}\cdot\sqrt{s_k}.$ By the inductive hypothesis, $\sqrt{2}\cdot\sqrt{s_k}< \sqrt{2}\cdot\sqrt{2} = 2.$ \\

By the Principle of Mathematical Induction, $s_n<2$ for all $n\in\N$.
\end{proof}



\pagebreak



\begin{enumerate}
\item[4.] Find the upper and lower limits of the sequence $\{s_n\}$ defined by
$$s_1=0; \quad s_{2m}=\frac{s_{2m-1}}{2}; \quad s_{2m+1}=\frac{1}{2} + s_{2m}.$$
\end{enumerate}

\begin{proof}

\end{proof}



\pagebreak



\begin{enumerate}
\item[5.] For any two real sequences $\{a_n\}, \{b_n\},$ prove that
$$\lim_{n\rightarrow\infty} \sup(a_n+b_n)\leq \lim_{n\rightarrow\infty} \sup(a_n) + \lim_{n\rightarrow\infty} \sup(b_n)$$
provided the sum on the right is not of the form $\infty-\infty$.
\end{enumerate}

\begin{proof}

\end{proof}

\end{document}
