% Author: Connor Baker
% Date Created: February 27, 2017
% Last Edited: February 27, 2017
% Version: 0.1a

% Declare type of document
\documentclass[10pt]{article}

% Import Packages
\usepackage[utf8]{inputenc}
\usepackage{amsfonts,amsmath,amssymb,amsthm}
\usepackage{mathtools,mathdots}
\usepackage{enumitem}
\usepackage{array}

% Page Formatting
% These settings let you manipulate the margins on the paper, and provide more options than you might be used to using in a word style document.  For example, the settings \oddsidemargin and \evensidemargin are allowed to be adjusted separately in case you are binding a book together.
\topmargin -0.25in \oddsidemargin -.25in \evensidemargin -.25in
\textheight 9in \textwidth 6.75in \headheight 0in \headsep .35in
\parindent 0in

% Define the basic math environments
\theoremstyle{definition}
\newtheorem{definition}[equation]{Definition}
\newtheorem{example}[equation]{Example}
\theoremstyle{plain}
\newtheorem{theorem}[equation]{Theorem}
\newtheorem{proposition}[equation]{Proposition}
\newtheorem{lemma}[equation]{Lemma}
\newtheorem{corollary}[equation]{Corollary}
\newtheorem{conjecture}[equation]{Conjecture}

% Define frequently used commands
\newcommand{\N}{\mathbb{N}}
\newcommand{\Z}{\mathbb{Z}}
\newcommand{\R}{\mathbb{R}}
\DeclareMathOperator\dom{dom}

\makeatletter
\def\imod#1{\allowbreak\mkern10mu({\operator@font mod}\,\,#1)}
\makeatother

% Begin the document
\begin{document}
% Create the Header
\begin{center}
  \subsection*{Homework 4\\Connor Baker, February 2017}
\end{center}

% Problem 1
\begin{enumerate}
\item[1.] Prove that if $R$ is a partial order on a set $A$, then $R^{-1}$ (the inverse relation) is also a partial order on $A$.
\end{enumerate}

% Proof 1
\begin{proof}
\end{proof}



\pagebreak



% Problem 2
\begin{enumerate}
  \item[2.] Let $R$ be a relation on the set $A$.  Prove that if $S$ is a symmetric relation on $A$, and $R \subseteq S$, then $R^{-1} \subseteq S$.
\end{enumerate}

% Proof 2
\begin{proof}
\end{proof}



\pagebreak



% Problem 3
\begin{enumerate}
  \item[3.] Let $R$ be an antisymmetric relation on the nonempty set $A$.  Prove that if $R$ is symmetric and $\dom(R) = A$, then $R = I_A$ (the identity relation on $A$).
\end{enumerate}

% Proof 3
\setcounter{equation}{0}
\begin{proof}
\end{proof}



\pagebreak



% Problem 4
\begin{enumerate}
  \item[4.] Prove that the subset of every well-ordered set is well ordered.
\end{enumerate}

% Proof 4
\setcounter{equation}{0}
\begin{proof}
\end{proof}



\pagebreak



% Problem 5
\begin{enumerate}
  \item[5.] Prove that $R$ is transitive on a set $A$ if and only if $R \circ R \subseteq R$.
\end{enumerate}

% Proof 5
\setcounter{equation}{0}
\begin{proof}
\end{proof}
% End problem set

\end{document} % End document
