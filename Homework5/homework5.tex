% Author: Connor Baker
% Date Created: March 18, 2017
% Last Edited: April 10, 2017
% Version: 0.4b

% Declare type of document
\documentclass[10pt]{article}

% Import Packages
\usepackage[utf8]{inputenc}
\usepackage[mathscr]{euscript}
\usepackage{amsfonts,amsmath,amssymb,amsthm}
\usepackage{mathtools,mathdots}
\usepackage{enumitem}
\usepackage{array}
\usepackage{longtable}

% Page Formatting
% These settings let you manipulate the margins on the paper, and provide more options than you might be used to using in a word style document.  For example, the settings \oddsidemargin and \evensidemargin are allowed to be adjusted separately in case you are binding a book together.
\topmargin -0.25in \oddsidemargin -.25in \evensidemargin -.25in
\textheight 9in \textwidth 6.75in \headheight 0in \headsep .35in
\parindent 0in

% Define the basic math environments
\theoremstyle{definition}
\newtheorem{definition}[equation]{Definition}
\newtheorem{example}[equation]{Example}
\theoremstyle{plain}
\newtheorem{theorem}[equation]{Theorem}
\newtheorem{proposition}[equation]{Proposition}
\newtheorem{lemma}[equation]{Lemma}
\newtheorem{corollary}[equation]{Corollary}
\newtheorem{conjecture}[equation]{Conjecture}

% Define frequently used commands
\newcommand{\N}{\mathbb{N}}
\newcommand{\Z}{\mathbb{Z}}
\newcommand{\Q}{\mathbb{Q}}
\newcommand{\R}{\mathbb{R}}
\DeclareMathOperator\dom{dom}
\DeclareMathOperator\rang{rang}
\newcommand{\ds}{\displaystyle}

\makeatletter
\def\imod#1{\allowbreak\mkern10mu({\operator@font mod}\,\,#1)}
\makeatother

% Begin the document
\begin{document}
% Create the Header
\begin{center}
  \subsection*{Homework 5\\Connor Baker, March 2017}
\end{center}

% Problem 1
\begin{enumerate}
\item Prove that if the real-valued function $f$ is strictly increasing or strictly decreasing on $\R$, then $f$ is one-to-one (Note:  You cannot assume $f$ is differentiable).
\end{enumerate}



% Proof 1
\begin{proof}
  Case 1: $f$ is strictly decreasing.

  Let $x,a\in\dom(f).$ Assume that $f(x)=f(a)$. If $x\neq a$, then by trichotomy either $x<a$ or $x>a.$

  Case A: If $x<a,$ then $f(x) > f(a)$, and $f(x)\neq f(a)$.

  Case B: If $x>a,$ then $f(x) < f(a)$, and $f(x)\neq f(a)$.

  In either case, if $x\neq a,$ then  $f(x)\neq f(a),$ and $f$ is one-to-one.

  Case 2: $f$ is strictly increasing.

  Let $x,a\in\dom(f).$ Assume that $f(x)=f(a)$. If $x\neq a$, then by trichotomy either $x<a$ or $x>a.$

  Case A: If $x<a,$ then $f(x) < f(a)$, and $f(x)\neq f(a)$.

  Case B: If $x>a,$ then $f(x) > f(a)$, and $f(x)\neq f(a)$.

  In either case, if $x\neq a,$ then  $f(x)\neq f(a),$ and $f$ is one-to-one.

  As such, $f$ is one-to-one if it is strictly increasing or strictly decreasing on $\R$.
\end{proof}



\pagebreak



% Problem 2
\begin{enumerate}
  \item[2.] Prove the following are metrics:
  \begin{enumerate}
    \item $X = \R, d(x,y) =  \begin{cases} 0 & \mbox{if } x = y \\ 1 & \mbox{if } x \neq y \end{cases} $
    \item $X = \R \times \R, d((x,y),(z,w)) = \sqrt{(x-z)^2 + (y-w)^2}$
  \end{enumerate}
  \begin{definition}[Metric]
    A metric on a set $X$ is a function $d: X \times X \to \R$ such that for all $x,y,z \in X$,
  \end{definition}
  \begin{enumerate}
    \item $d(x,y) \geq 0$
    \item $d(x,y) = 0$ if and only if $x = y$
    \item $d(x,y) = d(y,x),$
    \item $d(x,y) + d(y,z) \geq d(x,z)$.
  \end{enumerate}
\end{enumerate}



% Proof 2
\begin{proof}
  We begin by proving that the first function is a metric.
  \begin{enumerate}
    \item The $\rang(d)=\{0,1\}$ so the function is definitely greater than or equal to zero for any inputted pair of values.
    \item By the definition of $d$, $d(x,y)=0$ if and only if $x=y$.
    \item Since the equals relationship is symmetric, $x=y \implies y=x$. As such, $d(x,y) = d(y,x)$, since the order of the inputs does not generate a unique output.
    \item Not completed.
  \end{enumerate}
  Therefore the first function is a metric. \\

  We now prove that the second function is a metric.
  \begin{enumerate}
    \item The $\rang(d)=\{0,1\}$ so the function is definitely greater than zero.
    \item By the definition of $d$, if and only if $x=y$ does $d(x,y)=1$.
    \item Since the equals relationship is symmetric, $x=y \implies y=x$. As such, $d(x,y) = d(y,x)$.
    \item Not completed.
  \end{enumerate}
  Therefore the function is a metric.
\end{proof}



\pagebreak



% Problem 3
\begin{enumerate}
\item[3.] Let $f: \N \times \N \to \N$ be given by $f(m,n) = 2^{m-1}(2n-1)$.  Prove that $f$ is one-to-one and onto.
\end{enumerate}



% Proof 3
\begin{proof}
  If we can prove that there is a unique solution $f(m,n)$ for every $m,n\in\N$, then we will have proven that $f$ is one-to-one and onto.

  We begin by breaking apart the function. Let us consider two functions:
  $$g(m) = 2^{m-1}$$
  and
  $$h(n) = 2n-1,$$
  such that
  $$f(m,n) = g(m)\cdot h(n).$$

  The function $g$ is clearly one-to-one and onto for any $m$. Any value of $m$ produces a power of two, all of which are in $\N$. Looking at $f$, we see that if $m>1$, $g$ in effect creates the even factors found in the result, $f(m,n)$.

  Considering $h$, we see that it is also one-to-one and onto for any $n$. Any value of $n$ produces an odd number (because $h$ is the definition of an odd number) -- in fact, $h$ produces every odd number in $\N$.

  There are two cases to consider.

  Case 1: $m=1$. In this case, $f(1,n)$ will map $n$ to every odd number in $\N$. If $m=1$, the function $f$ is one-to-one and onto.

  Let $p_1 p_2 \dots p_q$ be the prime factorization of $f(m,n)$, where $p_i$, $1\leq i \leq q$, is a prime factor of $f(m,n)$ raised to some power. By the fundamental theorem of arithmetic, the prime factorization of a number is unique up to commutativity. In this case, we find that $f(1,n) = 2n-1$, so the prime factorization is entirely dependent on the value of $n$.

  Case 2: $m>1$. In this case, $f(m,n)$ has some power of two (that is not one) multiplying an odd number. Since any odd number multiplied by an even number is even, $f(m,n)$ will be even for all $n$.

  Let $p_1 p_2 \dots p_q$ be the prime factorization of $f(m,n)$, where $p_i$, $1\leq i \leq q$, is a prime factor of $f(m,n)$ raised to some power. By the fundamental theorem of arithmetic, the prime factorization of a number is unique up to commutativity. In this case, we find that $f(m,n) = 2^{m-1}(2n-1)$, so the prime factorization is entirely dependent on the value of both $m$ and $n$.

  Since different values of $n$ will yield different odd numbers, the prime factors will be the same if and only if the value of the input is the same. The same is true for $m$.

  As such, as long as $(m,n)\neq(p,q)$, $f(m,n)$ and $f(p,q)$ have different prime factors. Therefore, $f(m,n)$ is one-to-one.

  Furthermore, there is always at least one solution for all $m,n$, so $f(m,n)$ is onto as well.
\end{proof}

\pagebreak



% Problem 4
\begin{enumerate}
\item[4.] Let $f:A \to B$ be a function from a nonempty set $A$.  Prove that the set $ \mathcal{C} = \{f^{-1}(b): b \in \rang(f)\}$ is a partition of $A$.  Note:  $\mathcal{C}$ is a subset of $\mathscr{P}(A)$.
\end{enumerate}



% Proof 4
\setcounter{equation}{0}
\begin{definition}[Partition]
The set $\mathcal{C}$ is a partition of $A$ if:
\begin{enumerate}
  \item $\forall D\in\mathcal{C},\ D\neq \emptyset$
  \item $\forall D,E\in\mathcal{C},$ then either $D\cap E = \emptyset$ or $D=E$
  \item $\cup_{D\in\mathcal{C}} D = A$
\end{enumerate}
\end{definition}

\begin{proof}
We begin by showing that $\mathcal{C}$ does not contain the empty set. By the definition of the inverse, we know that $f^{-1} = \{a\in A:f(a) = b\}$. Choosing any $D\in\mathcal{C}\subseteq\mathscr{P}(A)$, we can see that $\exists b\in\rang(f):f^{-1}(b)=D$. The set $D$ is nonempty, since if $b\in\rang(f)$, then there exists $a\in A:f(a)=b$.

We now show that any two subsets of $\mathcal{C}$ are pairwise disjoint if they are not the same set. Since $f$ is a function, there is only one $f(a)$ for any $a\in A$. Then, the intersection of any two subsets of $\mathcal{C}$, which is the set containing the inverse of all elements of the range, will be the empty set unless they are the same: this follows as a result of there being only one $f(a)$ for any $a \in A$ -- if there were more than one, then multiple elements in the range could have the same preimage, but this is not the case. Indeed, the intersect of any two sets in $\mathcal{C}$ is not the empty set if they are the same set.

Finally, we show that the union of all sets in $\mathcal{C}$ are equal to $A$. since $f$ is a function, $\dom(f)=A$. That means that every point in $A$ is either mapped to $B$ by $f$, or does not have a solution. By the definition of $\mathcal{C}$, the other partition must be the set of $a\in A: f^{-1}(b)\neq a, b\in B$. The union of these two partitions is equal to $A$.

Therefore, $\mathcal{C}$ is a partition of $A$.


\end{proof}



\pagebreak



% Problem 5
\begin{enumerate}
\item[5.] Let $f:A \to B$ be a function from a nonempty set $A$ which is surjective.  Find a new function $g:C \to B$ which is one-to-one such that $C \subseteq A$, $\rang(g) = \rang(f)$, and for every $x \in C, f(x) = g(x)$.  Explain why $g$ has an inverse function, $g^{-1}$.  Then, compute $f(g^{-1}(x))$ for all $x \in B$.
\end{enumerate}



% Proof 5
\begin{proof}
Since $f$ is surjective, there is at least one solution for $f(a)$ for all $a\in A$. Consider the function $g:C\to B$, where $C$ is the set of all values in $A$ such that a single element is chosen from every different partition of $A$. Then, since we can partition $A$ such that each partition's elements send all its points to the same point in $B$, by restricting the domain to a single point from each partition, we can make $g$ one-to-one -- and since $f$ was onto, simply restricting the domain doesn't cause $g$ to cease being onto, which makes $g$ bijective. Furthermore, $\rang(g) = \rang(f)$, and for every $x \in C, f(x) = g(x)$, since $g$ is essentially $f$ with a restricted domain. Since $g$ is bijective, it has an inverse that is also bijective.

The composed function $f(g^{-1}(x))$ for all $x \in B$ is $B$. The inverse $g^{-1}$ takes every element of $B$ and maps it to a partition of $A$. Then, $f$ operates on those elements (which are essentially a single element from every partition of $A$, making $f$ look like $g$ since it's operating on what is essentially a restricted domain) and sends them to the range. Since $f$ is onto, every element that $g^{-1}$ returned in $A$ is sent back to $B$. As such, the composition yields the set $B$.
\end{proof}

\end{document} % End the document
