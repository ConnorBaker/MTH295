% Author: Connor Baker
% Date Created: March 18, 2017
% Last Edited: March 27, 2017
% Version: 0.3a

% Declare type of document
\documentclass[10pt]{article}

% Import Packages
\usepackage[utf8]{inputenc}
\usepackage[mathscr]{euscript}
\usepackage{amsfonts,amsmath,amssymb,amsthm}
\usepackage{mathtools,mathdots}
\usepackage{enumitem}
\usepackage{array}
\usepackage{longtable}

% Page Formatting
% These settings let you manipulate the margins on the paper, and provide more options than you might be used to using in a word style document.  For example, the settings \oddsidemargin and \evensidemargin are allowed to be adjusted separately in case you are binding a book together.
\topmargin -0.25in \oddsidemargin -.25in \evensidemargin -.25in
\textheight 9in \textwidth 6.75in \headheight 0in \headsep .35in
\parindent 0in

% Define the basic math environments
\theoremstyle{definition}
\newtheorem{definition}[equation]{Definition}
\newtheorem{example}[equation]{Example}
\theoremstyle{plain}
\newtheorem{theorem}[equation]{Theorem}
\newtheorem{proposition}[equation]{Proposition}
\newtheorem{lemma}[equation]{Lemma}
\newtheorem{corollary}[equation]{Corollary}
\newtheorem{conjecture}[equation]{Conjecture}

% Define frequently used commands
\newcommand{\N}{\mathbb{N}}
\newcommand{\Z}{\mathbb{Z}}
\newcommand{\Q}{\mathbb{Q}}
\newcommand{\R}{\mathbb{R}}
\DeclareMathOperator\dom{dom}
\DeclareMathOperator\rang{rang}
\newcommand{\ds}{\displaystyle}

\makeatletter
\def\imod#1{\allowbreak\mkern10mu({\operator@font mod}\,\,#1)}
\makeatother

% Begin the document
\begin{document}
% Create the Header
\begin{center}
  \subsection*{Homework 5\\Connor Baker, March 2017}
\end{center}

% Problem 1
\begin{enumerate}
\item Prove that if the real-valued function $f$ is strictly increasing or strictly decreasing on $\R$, then $f$ is one-to-one (Note:  You cannot assume $f$ is differentiable).
\end{enumerate}



% Proof 1
\begin{proof}
  Case 1: $f$ is strictly decreasing.

  If $f$ is strictly decreasing, then $\forall\ x,a\in\dom(f),\ x\leq a,\ f(x) \geq f(a)$. If it is the case that $f(x)=f(a)$, then it must also be the case that $x=a$. Then, $f(x)=f(a)$ if and only if $x=a$, and $f(x) > f(a)\ \forall x,a\in\dom(f),\ x < a$, the function $f$ must be one-to-one.

  Case 2: $f$ is strictly increasing.

  If $f$ is strictly increasing, then $\forall\ x,a\in\dom(f),\ x\leq a,\ f(x) \leq f(a)$. If it is the case that $f(x)=f(a)$, then it must also be the case that $x=a$. Then, $f(x)=f(a)$ if and only if $x=a$, and $f(x) < f(a)\ \forall x,a\in\dom(f),\ x < a$, the function $f$ must be one-to-one.
\end{proof}



\pagebreak



% Problem 2
\begin{enumerate}
  \item[2.] Prove the following are metrics:
  \begin{enumerate}
    \item $X = \R, d(x,y) =  \begin{cases} 0 & \mbox{if } x = y \\ 1 & \mbox{if } x \neq y \end{cases} $
    \item $X = \R \times \R, d((x,y),(z,w)) = \sqrt{(x-z)^2 + (y-w)^2}$
  \end{enumerate}
  \begin{definition}[Metric]
    A metric on a set $X$ is a function $d: X \times X \to \R$ such that for all $x,y,z \in X$,
  \end{definition}
  \begin{enumerate}
    \item $d(x,y) \geq 0$
    \item $d(x,y) = 0$ if and only if $x = y$
    \item $d(x,y) = d(y,x),$
    \item $d(x,y) + d(y,z) \geq d(x,z)$.
  \end{enumerate}
\end{enumerate}



% Proof 2
\begin{proof}
  We begin by proving that the first function is a metric.
  \begin{enumerate}
    \item The $\rang(d)=\{0,1\}$ so the function is definitely greater than or equal to zero for any inputted pair of values.
    \item By the definition of $d$, $d(x,y)=0$ if and only if $x=y$.
    \item Since the equals relationship is symmetric, $x=y \implies y=x$. As such, $d(x,y) = d(y,x)$, since the order of the inputs does not generate a unique output.
    \item Not completed.
  \end{enumerate}
  Therefore the first function is a metric. \\

  We now prove that the second function is a metric.
  \begin{enumerate}
    \item The $\rang(d)=\{0,1\}$ so the function is definitely greater than zero.
    \item By the definition of $d$, if and only if $x=y$ does $d(x,y)=1$.
    \item Since the equals relationship is symmetric, $x=y \implies y=x$. As such, $d(x,y) = d(y,x)$.
    \item Not completed.
  \end{enumerate}
  Therefore the function is a metric.
\end{proof}



\pagebreak



% Problem 3
\begin{enumerate}
\item[3.] Let $f: \N \times \N \to \N$ be given by $f(m,n) = 2^{m-1}(2n-1)$.  Prove that $f$ is one-to-one and onto.
\end{enumerate}



% Proof 3
\begin{proof}
  Let $a,b,c\in\N$. Then, $f(b,c)\geq 1$, and $\forall a\in\N,\ \exists (b,c)\in\N\times\N:\ f(b,c)=a$, since $a,b$ and $c$ are all natural numbers, and the natural numbers are closed under multiplication. As such, $f$ is surjective.

  The function $f$ is injective if $\forall x,y\in\N\times\N, x\neq y, f(x)\neq f(y).$ Then, assuming $f(x)=f(y)$ we have $2^{a-1} (2b-1) = 2^{c-1} (2d-1).$ Assuming that $x\neq y$, $x=(a,c)$ and $y=(c,d)$, it must be that the case that $a\neq c$ or $b\neq d$.

  Since the prime factor decomposition of a number is unique up to commutativity, they must have the same factor of two, which must come from the term outside the parenthesis (since $2k-1 \ \forall k\in\N$ is the definition of an odd number and will therefore have no factor of two), so $a=c$.

  Furthermore, since the two sides of the equation have the same factor of two and the prime factor decomposition is unique, the terms in the parenthesis must be equal. Using cancellation, we find that $b=d$.

  This is a contradiction of our assumption that $x\neq y$, so it must be that $f(x)=f(y)$ if and only if $x=y$, and $f$ is injective.

  Since $f$ is both injective and surjective, it must be bijective.
\end{proof}



\pagebreak



% Problem 4
\begin{enumerate}
\item[4.] Let $f:A \to B$ be a function from a nonempty set $A$.  Prove that the set $ \mathcal{C} = \{f^{-1}(b): b \in \rang(f)\}$ is a partition of $A$.  Note:  $\mathcal{C}$ is a subset of $\mathscr{P}(A)$.
\end{enumerate}



% Proof 4
\begin{proof}
  The range of the function $f$ is by definiton a subset of the codomain, $B$. If $b\in\rang(f)$, then $\exists a\in A:\ f(a) = b$. Then, it must be that $\forall b\in\rang(f)$, there exists $a\in A:\ f^{-1}(b) = a$. Then, the set of all $a:\ f^{-1}(b) = a$ is a subset of $A$, which is in turn a subest of the power set of $A$, which contains partitions of $A$. Therefore, $\mathcal{C}$ is a partition.
\end{proof}



\pagebreak



% Problem 5
\begin{enumerate}
\item[5.] Let $f:A \to B$ be a function from a nonempty set $A$ which is surjective.  Find a new function $g:C \to B$ which is one-to-one such that $C \subseteq A$, $\rang(g) = \rang(f)$, and for every $x \in C, f(x) = g(x)$.  Explain why $g$ has an inverse function, $g^{-1}$.  Then, compute $f(g^{-1}(x))$ for all $x \in B$.
\end{enumerate}



% Proof 5
\begin{proof}
  Since $\rang(g) = \rang(f)$, and $\rang(f) = B$ (because $f$ is surjective), $\rang(g) = B$, and as a result $g$ is bijective. The function $g$ has an inverse because it is bijective. For all $x \in B$, the function $f(g^{-1}(x))$ is equivalent to $f(C)$ since $g$ has an inverse and is bijective. Furthermore, $f(C)=B$, since $C\subseteq A$ and $\rang(g) = \rang(f)$.
\end{proof}

\end{document} % End the document
