% Author: Connor Baker
% Date Created: February 14, 2017
% Last Edited: February 20, 2017
% Version: 0.1b

% Declare type of document
\documentclass[10pt]{article}

% Import Packages
\usepackage[utf8]{inputenc}
\usepackage{amsfonts,amsmath,amssymb,amsthm}
\usepackage{mathtools,mathdots}
\usepackage{enumitem}
\usepackage{array}

% Page Formatting
% These settings let you manipulate the margins on the paper, and provide more options than you might be used to using in a word style document.  For example, the settings \oddsidemargin and \evensidemargin are allowed to be adjusted separately in case you are binding a book together.
\topmargin -0.25in \oddsidemargin -.25in \evensidemargin -.25in
\textheight 9in \textwidth 6.75in \headheight 0in \headsep .35in
\parindent 0in

% Define the basic math environments
\theoremstyle{definition}
\newtheorem{definition}[equation]{Definition}
\newtheorem{example}[equation]{Example}
\theoremstyle{plain}
\newtheorem{theorem}[equation]{Theorem}
\newtheorem{proposition}[equation]{Proposition}
\newtheorem{lemma}[equation]{Lemma}
\newtheorem{corollary}[equation]{Corollary}
\newtheorem{conjecture}[equation]{Conjecture}

% Define frequently used commands
\newcommand{\N}{\mathbb{N}}
\newcommand{\Z}{\mathbb{Z}}
\newcommand{\R}{\mathbb{R}}

% Begin the document
\begin{document}
% Create the Header
\begin{center}
  \subsection*{Homework 3\\Connor Baker, February 2017}
\end{center}

% Problem 1
\begin{enumerate}
\item[1.] Let $R$ be a relation from a nonempty set $A$ to itself.  Prove that if $R$ is symmetric, transitive, and $dom(R) = A$, then $R$ is an equivalence relation.
\end{enumerate}

% Proof 1
\begin{proof}
\end{proof}



\pagebreak



% Problem 2
\begin{enumerate}
  \item[2.] Use the Principle of Mathematical Induction to prove $3^n \geq 2^n + 1$ for all $n \in \N$.
\end{enumerate}

% Proof 2
\begin{proof}
  Let $n=1$. Then,
  \begin{equation}
    3^1 \geq 2^1 + 1.
  \end{equation}
  Next, we must now prove that if the inequality is true for some $n\in\N$, the above inequality also holds for $n+1$.

  \par Let $n=k$. Then, assume the following equation to be true:
  \begin{equation}
    3^k \geq 2^k + 1.
  \end{equation}

  \par We now try to prove that the inequality holds for $n=k+1$:
  \begin{equation}
    3^{k+1} \geq 2^{k+1} + 1,
  \end{equation}
  which is equivalent to
  \begin{equation}
    3^{k}*3 \geq 2^{k}*2 + 1.
  \end{equation}

  \par Multiplying Inequality (2) by two gives us:
  \begin{equation}
    3^{k}*2 \geq 2^{k}*2 + 2.
  \end{equation}

  \par So:
    $$3^{k+1}=3^{k}*3 > 3^{k}*2 \geq 2^{k}*2 + 2 > 2^{k}*2 + 1 = 2^{k+1}+1$$

  \par Therefore, by the Principal of Mathematical Induction, $3^n \geq 2^n + 1$, $\forall n\in\N$.
\end{proof}



\pagebreak



% Problem 3
\begin{enumerate}
  \item[3.] Use the Principle of Mathematical Induction to prove
  $$\sqrt{2\sqrt{2\sqrt{2\sqrt{2\dots}}}} \leq 2.$$
  (Hint:  Construct a recursively defined sequence.)
\end{enumerate}

% Proof 3
\setcounter{equation}{0}
\begin{proof}
  We construct a recursively defined sequence to model the left side of the inequality we want to prove. Let
    $$a_1 = \sqrt{2}$$ and,
    $$a_n = \sqrt{2*a_{n-1}}.$$
    Now, with this recursive sequence, we can prove by induction.

  Let $n=1$. Then,
  \begin{equation}
    \sqrt{2} \leq 2
  \end{equation}
  holds true.

  \par Next, we must now prove that if the inequality is true for some $n\in\N$, the above inequality also holds for $n+1$.

  \par Let $n=k$. Then, assume the following equation to be true:
  \begin{equation}
    a_k = \sqrt{2*a_{k-1}} \leq 2.
  \end{equation}

  \par We now try to prove that the inequality holds for $n=k+1$:
  \begin{equation}
    a_{k+1} = \sqrt{2*a_{k}} \leq 2
  \end{equation}

  \par Since $a_k \leq 2$, it must be the case that $\sqrt{a_k} \leq \sqrt{2}.$ Therefore, since
  \begin{equation}
    a_{k+1} = \sqrt{2}*\sqrt{a_{k}}
  \end{equation}
  it must be the case that
  \begin{equation}
    a_{k+1} \leq 2.
  \end{equation}
  Therefore, by the Principle of Mathematical Induction, it has been proven that $\sqrt{2\sqrt{2\sqrt{2\sqrt{2\dots}}}} \leq 2.$
\end{proof}



\pagebreak



% Problem 4
\begin{enumerate}
  \item[4.] Let $a_1 = 2, a_2 = 4$, and $a_{n+2} = 5a_{n+1} - 6a_n$ for $n \geq 1$.  Prove that $a_n = 2^n$ for all natural numbers $n$.
\end{enumerate}

% Proof 4
\setcounter{equation}{0}
\begin{proof}
  Let $n=1$. Then:
  $$a_1 = 2 = 2^1.$$

  \par Next, we must now prove that if the inequality is true for some $n\in\N$, the above inequality also holds for $n+1$.

  \par Let $n=k$. Then, assume the following equation to be true:
  \begin{equation}
    a_{k+2} = 5a_{k+1} - 6a_k = 2^k.
  \end{equation}

  \par We now try to prove that the inequality holds for $n=k+1$:
  \begin{equation}
    a_{k+3} = 5a_{k+2} - 6a_{k+1} = 2^{k+1}.
  \end{equation}

  \par We substitute Equation (1) into Equation (2):
  \begin{equation}
    a_{k+3} = 5(2^k) - 6a_{k+1} = 2^{k+1}.
  \end{equation}

  \par Then, we find the equation for $a_{k+1}$ so that we can substitute that as well:
  \begin{equation}
    a_{k+1} = 5a_{k} - 6a_{k-1} = 2^{k-1},
  \end{equation}
  which we then substitute into Equation (3):
  \begin{equation}
    a_{k+3} = 5(2^k) - 6(2^{k-1}) = 2^{k+1}.
  \end{equation}

  \par Expanding this equaiton yeilds:
  \begin{equation}
    2^{k-1} (5*2 - 6) = 2^{k-1} (10 - 6) = 2^{k-1}*4 = 2^{k+1}.
  \end{equation}

  \par Therefore, by the Principle of Mathematical Induction, $a_n = 2^n$ holds for all natural numbers $n$.
\end{proof}



\pagebreak



% Problem 5
\begin{enumerate}
  \item[5.] Use the Principle of Mathematical Induction to prove that
  $$1 \cdot 1! + 2 \cdot 2! + 3 \cdot 3! + ... + n \cdot n! = (n+1)! - 1.$$
\end{enumerate}

% Proof 5
\begin{proof}
\end{proof}
% End problem set

\end{document} % End document
