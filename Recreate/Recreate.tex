% Author: Connor Baker
% Date Created: January 18, 2016
% Last Edited: January 18, 2016
% Version: 0.1a

\documentclass[10pt]{article}
% Import Packages

\usepackage[utf8]{inputenc}
\usepackage{amsfonts,amsmath,amssymb,amsthm}
\usepackage{mathtools,mathdots}
\usepackage{enumitem}
\usepackage{array}

% Page Formatting
% These settings let you manipulate the margins on the paper, and provide more options than you might be used to using in a word style document.  For example, the settings \oddsidemargin and \evensidemargin are allowed to be adjusted separately in case you are binding a book together.
\usepackage[left=0.905in,right=0.8661in,top=0.827in,bottom=0.827in]{geometry}
\parindent 0in

% Define the basic math environments
\theoremstyle{definition}
\newtheorem{definition}[equation]{Definition}
\newtheorem{example}[equation]{Example}
\theoremstyle{plain}
\newtheorem{theorem}[equation]{Theorem}
\newtheorem{proposition}[equation]{Proposition}
\newtheorem{lemma}[equation]{Lemma}
\newtheorem{corollary}[equation]{Corollary}
\newtheorem{conjecture}[equation]{Conjecture}

% Define frequently used commands
\newcommand{\N}{\mathbb{N}}
\newcommand{\Z}{\mathbb{Z}}
\newcommand{\R}{\mathbb{R}}

% Use a black square for QED
\renewcommand\qedsymbol{$\blacksquare$}

% Begin the document
\begin{document}
$\forall x\in\R,\ x\leq x+1$. \\
This sentence should start be on the next line.

\begin{equation*}
  2^x,2^{2^x},2^{2^{2^{\iddots^{2^{x}}}}},a_{b_{c_{d_{e_{f_{g_{h_{i_{j_{k_{l_{m_{n_{n_{o_{p_{q_{r_{s_{t_{u_{v_{w_{x_{y_{z_{_{_{_{_{}}}}}}}}}}}}}}}}}}}}}}}}}}}}}}}
\end{equation*}

$\arctan(\theta\varphi),\ \sin(x),\ \ln(x),\ \log_{10}(y)$ \\
$(f\circ g)(x) = f(g(x))$. \\

\begin{equation*}
  \int{f(x)dx}, \int_{0}^{\text{ln(2)}}{e^xdx=e^x \bigg\rvert_{0}^{\text{ln(2)}}} = e^{\textln(2)} - e^0 = 2 - 1 = 1.
\end{equation*}

\begin{definition}[Convergent Sequence]
  Let $a_n:\N\rightarrow\R$ be a sequence. We say that the sequence $\{a_n\}_{n=1}^{\infty}$ converges to a real number $L$ if for any $\epsilon > 0$, there exists $N\in\N$ such that $\forall n \geq N$, $|a_n-L|<\epsilon$.
\end{definition}

\begin{theorem}[Squeeze Lemma]
  Let $a_n,b_n,c_n$ be sequences. If there exists $N\in\N$ such that for all $n\geq N$, $a_n\leq b_n \leq c_n$ and $\displaystyle\lim_{n\rightarrow\infty} a_n = \lim_{n\rightarrow\infty} c_n = L$, then $\displaystyle\lim_{n\rightarrow\infty} b_n = L$.
\end{theorem}

Since $-1\leq \sin(n)\leq-1$ for all $n$, it follows that $\frac{-1}{n}\leq\frac{\sin(n)}{n}\leq\frac{1}{n}$ for all $n\geq 1$. Since $\displaystyle\lim_{n\rightarrow\infty} \frac{-1}{n}=\displaystyle\lim_{n\rightarrow\infty}\frac{1}{n}=0$, it follows from the Squeeze Lemma that

\begin{equation*}
  \lim_{n\rightarrow\infty} \frac{\sin(n)}{n} = 0
\end{equation*}

\begin{equation*}
  f(x) = |x| = \left\{
    \begin{array}{ll}
      x & \text{if }x\geq 0 \\
      -x & \text{if }x < 0
    \end{array}.
\end{equation*}

Let $a_{ij}= i + j$, for any $i,j\in\N$. If $A$ is $2\times3$ matrix with $A=[a_{ij}]$, then

\begin{equation*}
  A =
  \[
  \begin{bmatrix}
    2 & 3 & 4 \\
    3 & 4 & 5
  \end{bmatrix}.
  \]
\end{equation*}

The following equation,

\setcounter{equation}{2}
\begin{equation} \label{QuadraticFormula}
  x=\frac{-b\pm \sqrt{b^2 - 4ac}}{2a}
\end{equation}

is labelled, so I can reference it later. For example, one might say: We can use equation \eqref{QuadraticFormula} to solve for $x$ in $x^2-2x+3=0$.

If I want a numerical list, I can use an enumerated environment. In fact, we could have an enumerated environment within an enumerated environment, as follows:

\begin{definition}[Group]
   The pair $(G,\cdot)$ is a group if:
  \begin{enumerate}
    \item $G$ is a non-empty section
    \item $\cdot$ a binary operation on $G$
    \begin{enumerate}
      \item this means that for any two elements $a,b\in G$, $a\cdot b\in G$
      \item in other words, $\cdot$ is a function from $G\times G$ to $G$
      \item in symbols: $\cdot : G\times G \rightarrow G$
    \end{enumerate}
    \item $(G,\cdot)$ is associative
    \begin{enumerate}
      \item this means that for any elements $a,b,c\in G$, $a\cdot (b\cdot c)= (a\cdot b)\cdot c$
    \end{enumerate}
    \item There is an element in $G$, denoted $e$, such that for any $a\in G$, $a\cdot e = e\cdot a = a$
    \item For each $a\in G$, there exists an element $b\in G$, such that $a\cdot b = b\cdot a = e$
  \end{enumerate}
\end{definition}
\end{document} % End document
