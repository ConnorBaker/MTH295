% Author: Connor Baker
% Date Created: January 26, 2016
% Last Edited: January 30, 2016
% Version: 0.1d

% Declare type of document
\documentclass[10pt]{article}

% Import Packages
\usepackage[utf8]{inputenc}
\usepackage{amsfonts,amsmath,amssymb,amsthm}
\usepackage{mathtools,mathdots}
\usepackage{enumitem}
\usepackage{array}

% Page Formatting
% These settings let you manipulate the margins on the paper, and provide more options than you might be used to using in a word style document.  For example, the settings \oddsidemargin and \evensidemargin are allowed to be adjusted separately in case you are binding a book together.
\topmargin -0.25in \oddsidemargin -.25in \evensidemargin -.25in
\textheight 9in \textwidth 6.75in \headheight 0in \headsep .35in
\parindent 0in

% Define the basic math environments
\theoremstyle{definition}
\newtheorem{definition}[equation]{Definition}
\newtheorem{example}[equation]{Example}
\theoremstyle{plain}
\newtheorem{theorem}[equation]{Theorem}
\newtheorem{proposition}[equation]{Proposition}
\newtheorem{lemma}[equation]{Lemma}
\newtheorem{corollary}[equation]{Corollary}
\newtheorem{conjecture}[equation]{Conjecture}

% Define frequently used commands
\newcommand{\N}{\mathbb{N}}
\newcommand{\Z}{\mathbb{Z}}
\newcommand{\R}{\mathbb{R}}

% Begin the document
\begin{document}
% Create the Header
\begin{center}
  \subsection*{Homework 2\\Connor Baker, January 2017}
\end{center}

% Problem 1
\begin{enumerate}
\item Determine whether the following expressions are true or false.  Give a complete explanation for each part.
  \begin{enumerate}
  \item $\emptyset \subseteq \{\emptyset, \{ \emptyset \} \}$
  \item $\{\emptyset\} \subseteq \{\emptyset, \{ \emptyset \} \}$
  \item $\{\{\emptyset\}\} \subseteq \{\emptyset, \{ \emptyset \} \}$
  \item For every set $A$, $\{\emptyset\} \subseteq A.$
  \item $\{1,2\} \in \{\{1,2,3\},\{1,3\}, 1, 2\}$
  \item $\{\{4\}\} \subseteq \{1,2,3,\{4\}\}$
  \end{enumerate}
\end{enumerate}

\begin{definition}[Subset]
  Given two subsets $A,B$, $A$ is said to be a subset of $B$ if and only if all elements of $A$ are also in $B$. That is to say:
  \begin{equation*}
    X \subseteq Y \iff \forall x(x\in X \implies x\in Y)
  \end{equation*}
\end{definition}

% Proof 1a
\begin{proof}{(a)}
  Let $A=\emptyset, B = \{\emptyset,\{\emptyset\}\}$. Then, by the definition of a subset,
  \begin{equation*}
    \emptyset \subseteq \{\emptyset,\{\emptyset\}\} \iff \forall a(a\in \emptyset \implies a\in \{\emptyset,\{\emptyset\}\})
  \end{equation*}
  However, $a\not\in\emptyset$ (the empty set contains no elements). As such, the statement is vacuously true (because for all $a$, of which there are none, we cannot tell whether it is in both sets or not). \\

  \par Therefore, by the definition of a subset, the expression is true.
\end{proof}

% Proof 1b
\begin{proof}{(b)}
  Let $A=\{\emptyset\}, B = \{\emptyset,\{\emptyset\}\}$. Then, by the definition of a subset,
  \begin{equation*}
    \{\emptyset\} \subseteq \{\emptyset,\{\emptyset\}\} \iff \forall a(a\in \{\emptyset\} \implies a\in \{\emptyset,\{\emptyset\}\})
  \end{equation*}
  Let $a=\emptyset$, the only element of $A$. Then,
  \begin{equation*}
    \{\emptyset\} \subseteq \{\emptyset,\{\emptyset\}\} \iff (\emptyset \in \{\emptyset\} \implies \emptyset \in \{\emptyset,\{\emptyset\}\})
  \end{equation*}
  which is true. \\

  \par Therefore, by the definition of a subset, the expression is true.
\end{proof}

% Proof 1c
\begin{proof}{(c)}
  Let $A=\{\{\emptyset\}\}, B = \{\emptyset,\{\emptyset\}\}$. Then, by the definition of a subset,
  \begin{equation*}
    \{\{\emptyset\}\} \subseteq \{\emptyset,\{\emptyset\}\} \iff \forall a(a\in \{\{\emptyset\}\} \implies a\in \{\emptyset,\{\emptyset\}\})
  \end{equation*}
  Let $a=\{\emptyset\}$, the only element of $A$. Then,
  \begin{equation*}
    \{\{\emptyset\}\} \subseteq \{\emptyset,\{\emptyset\}\} \iff (\{\emptyset\} \in \{\{\emptyset\}\} \implies \emptyset \in \{\emptyset,\{\emptyset\}\})
  \end{equation*}
  which is true. \\

  \par Therefore, by the definition of a subset, the expression is true.
\end{proof}


% Proof 1d
\begin{proof}{(d)}
  Let $B=\{\emptyset\}$. Then, by the definition of a subset,
  \begin{equation*}
    \{\emptyset\} \subseteq C \iff \forall b(b\in \{\emptyset\} \implies b\in C)
  \end{equation*}
  Let $b=\emptyset$, the only element of $B$. Then,
  \begin{equation*}
    \{\emptyset\} \subseteq C \iff (\emptyset \in \{\emptyset\} \implies \emptyset \in C)
  \end{equation*}
  which is contingent on the elements of $C$. There is no guarantee that $C$ contains the empty set. \\

  \par Therefore, by the definition of a subset, the expression is false.
\end{proof}


% Proof 1e
\begin{proof}{(e)}
  This statement is false because the set does not contain the set $\{1,2\}$.
\end{proof}

% Proof 1f
\begin{proof}{(f)}
  Let $A=\{\{4\}\}, B = \{1,2,3,\{4\}\}$. Then, by the definition of a subset,
  \begin{equation*}
    \{\{4\}\} \subseteq \{1,2,3,\{4\}\} \iff \forall a(a\in \{\{4\}\} \implies a\in \{1,2,3,\{4\}\})
  \end{equation*}
  Let $a=\{4\}$, the only element of $A$. Then,
  \begin{equation*}
      \{\{4\}\} \subseteq \{1,2,3,\{4\}\} \iff (\{4\}\in \{\{4\}\} \implies \{4\}\in \{1,2,3,\{4\}\})
  \end{equation*}
  which is true. \\

  \par Therefore, by the definition of a subset, the expression is true.
\end{proof}



\pagebreak



\begin{enumerate}
  % Problem 2
  \item[2.] Let $\Delta = [0,1) = \{x \in \R: 0 \leq x < 1\}$ and let $A_\alpha = (-\alpha,\alpha] = \{x \in \R: -\alpha < x \leq \alpha\} \subseteq \R$, where $\alpha \in \Delta$.  Prove that
      $$\bigcup_{\alpha \in \Delta} A_\alpha = (-1,1),$$
   and
      $$\bigcap_{\alpha \in \Delta} A_\alpha = \emptyset$$
\end{enumerate}

% Proof 2a
\begin{proof}{(a)}
  Assume
\end{proof}

% Proof 2b
\begin{proof}{(b)}
  For this to be true, then for all $\alpha,\beta\in\Delta$, $A_\alpha \neq A_\beta$, which means the union of all these sets is pairwise disjoint.
\end{proof}



\pagebreak



\begin{enumerate}
  % Problem 3
  \item[3.] Let $A, B, C,$ and $D$ be sets with $C \subseteq A$ and $D \subseteq B$.  Prove that $C \cup D \subseteq A \cup B$.
\end{enumerate}

% Proof 3
\begin{proof}
  Since $C\subseteq A$, the set $A$ must be at least as large as $C$ and contain every element $C$ has. The same follows for $D$ and $B$, since $D\subseteq B$. Then, $A\cup B$ is the set containing at least every element in the sets $C$ and $D$, and as such must contain $C\cup D$.
\end{proof}



\pagebreak



\begin{enumerate}
  % Problem 4
  \item[4.] Prove that if $\mathcal{A}$ is a non-empty family of sets, then $$\bigcap_{A \in \mathcal{A}}A \subseteq \bigcup_{A \in \mathcal{A}} A.$$
\end{enumerate}

% Proof 4
\begin{proof}
  The intersection over a family of sets is defined as:
  $$\bigcap_{A\in\mathcal{A}}A=\{x|(\forall A)(A\in\mathcal{A}\implies x\in A)\}$$
  And the union over a family of sets is defined as:
  $$\bigcup_{A\in\mathcal{A}}A=\{x|(\exists A)((A\in\mathcal{A}) \land (x\in A))\}$$
\end{proof}



\pagebreak



\begin{enumerate}
  % Problem 5
  \item[5.] Use the principle of mathematical induction to prove $4^{n+4} > (n+4)^4,$ for all natural numbers $n$.
\end{enumerate}

% Proof 5
\begin{proof}
  Let $n=1$: $4^5>5^4$. We see that the base case is true. Then, let $n=k$: $4^{k+4} > (k+4)^4$. Let $n=k+1$: $4^{k+5} > (k+5)^4$. Then:
    $$4^{k+5} = 4^k*4^5$$
    $$4^{k+4} = 4^k*4^4$$
    $$4(4^{k+4})>4^{k+4}$$
    $$4^k*4^5=4(4^{k+4})$$
    $$(k+5)^4 = k^4 + 20k^3 + 150k^2 + 500k + 625$$
    $$(k+4)^4 = k^4 + 16k^3 + 96k^2 + 256k + 256$$
    $$4(k+4)^4 = 4k^4 + 64k^3 + 384k^2 + 1024k + 1024 > (k+5)^4$$
    $$4^{k+5} = 4(4^{k+4})>4(k+4)^4 > (k+5)^4$$
  Then, by the principle of mathematical induction, $4^{n+4} > (n+4)^4$ $\forall n\in\N$.
\end{proof}
% End problem set

\end{document} % End document
