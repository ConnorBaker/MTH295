\documentclass[10pt]{article} %this indicates the type of document you are making along with the size of font.  There are several other options available for this command as well.

%Note:  Any text entered on a line after a percent symbol % will not be read by the LaTeX compiler.  We will use this to add comments to anyone why might read the code (including yourself, you would be suprised how much you forget when you haven't look at a document in serval years.

%This section is the header, which include the libraries we plan on using, along with any short cut we define that will be used in this document.  You can also change properties of the document, like margins, page numbering, ect.

%Here we indicate the libraries we are planning on using.  There are tons of LaTeX libraries out there, you may find them useful not only in this course, but also in future courses.  I will start you off with some barebones libraries.

\usepackage{amsfonts,amsmath,amssymb,amsthm} %these libaries are provided by the AMS (American Mathematical Society) and are the most common fonts for mathematical publications

\usepackage{amsthm}


\usepackage{comment} %this library allows you comment out entire sections of LaTeX code by simpliy surrounding the code by the enviroment \begin{comment} \end{comment}.  This can be great for troubleshooting.

\usepackage{mathdots}

\theoremstyle{definition} %defines the basic math enviroments
\newtheorem{definition}[equation]{Definition}
\newtheorem{example}[equation]{Example}
\theoremstyle{plain}
\newtheorem{theorem}[equation]{Theorem}
\newtheorem{proposition}[equation]{Proposition}
\newtheorem{lemma}[equation]{Lemma}
\newtheorem{corollary}[equation]{Corollary}
\newtheorem{conjecture}[equation]{Conjecture}
%The following is an example of a user defined command.  You can use these to help simplify your LaTeX code

\newcommand{\R}{\mathbb{R}}
\newcommand{\ds}{\displaystyle} %instead of writing \mathbb{R} to the R that represent the real numbers, you can just write \R.  However, since it is a math command, you will have to be in a math environment.  For example, $\R$, or $$\R$$, or \[ \R\], or \begin{equation}\R\end{equation}.

\makeatletter
\def\imod#1{\allowbreak\mkern10mu({\operator@font mod}\,\,#1)}
\makeatother

%Page Formatting

\topmargin -0.25in \oddsidemargin -.25in \evensidemargin -.25in
\textheight 9in \textwidth 6.75in \headheight 0in \headsep .35in
\parindent 0in %these settings let you manipulate the margins on the paper, and provide more options than you might be used to using in a word style document.  For example, the settings \oddsidemargin and \evensidemargin are allowed to be adjusted separately in case you are binding a book together.

\begin{document} %This is where we type the text that we plan on having in our document.

\begin{center}
Homework 2
\end{center}

\begin{enumerate}
\item Determine whether the following expressions are true or false.  Give a complete explanation for each part.

\begin{enumerate}
\item $\emptyset \subseteq \{\emptyset, \{ \emptyset \} \}$
\item $\{\emptyset\} \subseteq \{\emptyset, \{ \emptyset \} \}$
\item $\{\{\emptyset\}\} \subseteq \{\emptyset, \{ \emptyset \} \}$
\item For every set $A$, $\{\emptyset\} \subseteq A.$
\item $\{1,2\} \in \{\{1,2,3\},\{1,3\}, 1, 2\}$
\item $\{\{4\}\} \subseteq \{1,2,3,\{4\}\}$
\end{enumerate}

\item Let $\Delta = [0,1) = \{x \in \R: 0 \leq x < 1\}$ and let $A_\alpha = 
(-\alpha,\alpha] = \{x \in \R: -\alpha < x \leq \alpha\} \subseteq \R$, where $\alpha \in \Delta$.  Prove that 
$$\bigcup_{\alpha \in \Delta} A_\alpha = (-1,1),$$
and
$$\bigcap_{\alpha \in \Delta} A_\alpha = \emptyset,$$

\item Let $A, B, C,$ and $D$ be sets with $C \subseteq A$ and $D \subseteq B$.  Prove that $C \cup D \subseteq A \cup B$.

\item Prove that if $\mathcal{A}$ is a non-empty family of sets, then

$$\bigcap_{A \in \mathcal{A}}A \subseteq \bigcup_{A \in \mathcal{A}} A.$$

\item Use the principle of mathematical induction to prove 

$$4^{n+4} > (n+4)^4,$$

for all natural numbers $n$.

\end{enumerate}

\end{document} %Where the text for the document ends.