% Author: Connor Baker
% Date Created: January 26, 2017
% Last Edited: February 2, 2017
% Version: 0.2d

% Declare type of document
\documentclass[10pt]{article}

% Import Packages
\usepackage[utf8]{inputenc}
\usepackage{amsfonts,amsmath,amssymb,amsthm}
\usepackage{mathtools,mathdots}
\usepackage{enumitem}
\usepackage{array}

% Page Formatting
% These settings let you manipulate the margins on the paper, and provide more options than you might be used to using in a word style document.  For example, the settings \oddsidemargin and \evensidemargin are allowed to be adjusted separately in case you are binding a book together.
\topmargin -0.25in \oddsidemargin -.25in \evensidemargin -.25in
\textheight 9in \textwidth 6.75in \headheight 0in \headsep .35in
\parindent 0in

% Define the basic math environments
\theoremstyle{definition}
\newtheorem{definition}[equation]{Definition}
\newtheorem{example}[equation]{Example}
\theoremstyle{plain}
\newtheorem{theorem}[equation]{Theorem}
\newtheorem{proposition}[equation]{Proposition}
\newtheorem{lemma}[equation]{Lemma}
\newtheorem{corollary}[equation]{Corollary}
\newtheorem{conjecture}[equation]{Conjecture}

% Define frequently used commands
\newcommand{\N}{\mathbb{N}}
\newcommand{\Z}{\mathbb{Z}}
\newcommand{\R}{\mathbb{R}}

% Begin the document
\begin{document}
% Create the Header
\begin{center}
  \subsection*{Homework 2\\Connor Baker, February 2017}
\end{center}

% Problem 1
\begin{enumerate}
\item Determine whether the following expressions are true or false.  Give a complete explanation for each part.
  \begin{enumerate}
  \item $\emptyset \subseteq \{\emptyset, \{ \emptyset \} \}$
  \item $\{\emptyset\} \subseteq \{\emptyset, \{ \emptyset \} \}$
  \item $\{\{\emptyset\}\} \subseteq \{\emptyset, \{ \emptyset \} \}$
  \item For every set $A$, $\{\emptyset\} \subseteq A.$
  \item $\{1,2\} \in \{\{1,2,3\},\{1,3\}, 1, 2\}$
  \item $\{\{4\}\} \subseteq \{1,2,3,\{4\}\}$
  \end{enumerate}
\end{enumerate}

\begin{definition}[Subset]
  Given two subsets $A,B$, $A$ is said to be a subset of $B$ if and only if all elements of $A$ are also in $B$. That is to say:
  \begin{equation*}
    X \subseteq Y \iff \forall x(x\in X \implies x\in Y)
  \end{equation*}
\end{definition}

% Proof 1a
\begin{proof}{(a)}
  Let $A=\emptyset, B = \{\emptyset,\{\emptyset\}\}$. Then, by the definition of a subset,
  \begin{equation*}
    \emptyset \subseteq \{\emptyset,\{\emptyset\}\} \iff \forall a(a\in \emptyset \implies a\in \{\emptyset,\{\emptyset\}\})
  \end{equation*}
  However, $a\not\in\emptyset$ (the empty set contains no elements). As such, the statement is vacuously true (because for all $a$, of which there are none, we cannot tell whether it is in both sets or not). \\

  \par Therefore, by the definition of a subset, the expression is true.
\end{proof}

% Proof 1b
\begin{proof}{(b)}
  Let $A=\{\emptyset\}, B = \{\emptyset,\{\emptyset\}\}$. Then, by the definition of a subset,
  \begin{equation*}
    \{\emptyset\} \subseteq \{\emptyset,\{\emptyset\}\} \iff \forall a(a\in \{\emptyset\} \implies a\in \{\emptyset,\{\emptyset\}\})
  \end{equation*}
  Let $a=\emptyset$, the only element of $A$. Then,
  \begin{equation*}
    \{\emptyset\} \subseteq \{\emptyset,\{\emptyset\}\} \iff (\emptyset \in \{\emptyset\} \implies \emptyset \in \{\emptyset,\{\emptyset\}\})
  \end{equation*}
  which is true. \\

  \par Therefore, by the definition of a subset, the expression is true.
\end{proof}

% Proof 1c
\begin{proof}{(c)}
  Let $A=\{\{\emptyset\}\}, B = \{\emptyset,\{\emptyset\}\}$. Then, by the definition of a subset,
  \begin{equation*}
    \{\{\emptyset\}\} \subseteq \{\emptyset,\{\emptyset\}\} \iff \forall a(a\in \{\{\emptyset\}\} \implies a\in \{\emptyset,\{\emptyset\}\})
  \end{equation*}
  Let $a=\{\emptyset\}$, the only element of $A$. Then,
  \begin{equation*}
    \{\{\emptyset\}\} \subseteq \{\emptyset,\{\emptyset\}\} \iff (\{\emptyset\} \in \{\{\emptyset\}\} \implies \emptyset \in \{\emptyset,\{\emptyset\}\})
  \end{equation*}
  which is true. \\

  \par Therefore, by the definition of a subset, the expression is true.
\end{proof}


% Proof 1d
\begin{proof}{(d)}
  Let $B=\{\emptyset\}$. Then, by the definition of a subset,
  \begin{equation*}
    \{\emptyset\} \subseteq C \iff \forall b(b\in \{\emptyset\} \implies b\in C)
  \end{equation*}
  Let $b=\emptyset$, the only element of $B$. Then,
  \begin{equation*}
    \{\emptyset\} \subseteq C \iff (\emptyset \in \{\emptyset\} \implies \emptyset \in C)
  \end{equation*}
  which is contingent on the elements of $C$. There is no guarantee that $C$ contains the empty set. \\

  \par Therefore, by the definition of a subset, the expression is false.
\end{proof}


% Proof 1e
\begin{proof}{(e)}
  This statement is false because the set does not contain the set $\{1,2\}$.
\end{proof}

% Proof 1f
\begin{proof}{(f)}
  Let $A=\{\{4\}\}, B = \{1,2,3,\{4\}\}$. Then, by the definition of a subset,
  \begin{equation*}
    \{\{4\}\} \subseteq \{1,2,3,\{4\}\} \iff \forall a(a\in \{\{4\}\} \implies a\in \{1,2,3,\{4\}\})
  \end{equation*}
  Let $a=\{4\}$, the only element of $A$. Then,
  \begin{equation*}
      \{\{4\}\} \subseteq \{1,2,3,\{4\}\} \iff (\{4\}\in \{\{4\}\} \implies \{4\}\in \{1,2,3,\{4\}\})
  \end{equation*}
  which is true. \\

  \par Therefore, by the definition of a subset, the expression is true.
\end{proof}



\pagebreak



\begin{enumerate}
  % Problem 2
  \item[2.] Let $\Delta = [0,1) = \{x \in \R: 0 \leq x < 1\}$ and let $A_\alpha = (-\alpha,\alpha] = \{x \in \R: -\alpha < x \leq \alpha\} \subseteq \R$, where $\alpha \in \Delta$.  Prove that
      $$\bigcup_{\alpha \in \Delta} A_\alpha = (-1,1),$$
   and
      $$\bigcap_{\alpha \in \Delta} A_\alpha = \emptyset$$
\end{enumerate}

% Proof 2a
\begin{proof}{(a)}
  Proving $\cup_{\alpha \in \Delta} A_\alpha \subseteq (-1,1)$: \\

  \par There exists some $x\in\Delta = [0,1)$ such that $a\in\ A_x=(-x,x]$ Since $0 \leq x<1$, letting $x$ be the largest value it can, the set $A_x$ which is $(-x,x]$ must be a subset of the set $(-1,1)$. \\

  \par Proving $(-1,1) \subseteq \cup_{\alpha \in \Delta} A_\alpha$: \\

  \par
\end{proof}

% Proof 2b
\begin{proof}{(b)}

\end{proof}



\pagebreak



\begin{enumerate}
  % Problem 3
  \item[3.] Let $A, B, C,$ and $D$ be sets with $C \subseteq A$ and $D \subseteq B$.  Prove that $C \cup D \subseteq A \cup B$.
\end{enumerate}

% Proof 3
\begin{proof}
  By the definition of a subset, $(\forall c)(c\in C \implies c\in A)$. Again, by the same definition, $(\forall d) (d\in D \implies d\in B)$. \\

  \par Then, by the definition of the union, $(\forall c\in C$ and $\forall d\in D)(c\in C \implies c\in \cup D$ and $d\in D \implies d\in C\cup D)$. \\

  \par Additionally, by the same definition, it follows from the first two statements that since all $c,d\in C\cup D$, then all $c,d \in A\cup B$. \\

  \par Therefore, since all $c,d\in C\cup D$ and all $c,d \in A\cup B$, $C\cup D \subseteq A\cup B$.
\end{proof}



\pagebreak



\begin{enumerate}
  % Problem 4
  \item[4.] Prove that if $\mathcal{A}$ is a non-empty family of sets, then $$\bigcap_{A \in \mathcal{A}}A \subseteq \bigcup_{A \in \mathcal{A}} A.$$
\end{enumerate}

% Proof 4
\begin{proof}
  The intersection of a family of sets creates a set containing only elements shared between all sets in the family (by definition). \\

  \par The union of a family of sets creates a set containing all elements that exist in any set in the family (by definition). \\

  \par Let $x$ be in the union of the family of sets. If $x$ is in the union, then that means that it is in at least one of the sets in the family. If it is in every set in the family, then it is in the intersection. However, if $x$ was to be in any number of sets (excluding being contained in every set), then $x$ would not be in the intersection of the family of sets. Because of this, elements of the union are not necessarily all in the intersection, and therefore the set generated by the union over the family is not necessarily a subset of the intersection. \\

  \par Let $y$ be in the intersection of the family of sets. This implies that $y$ is shared between every set in the family, and is therefore also in the union. As such, the intersection over a family of sets contains elements found in the union over the same family, and is a subset.
\end{proof}



\pagebreak



\begin{enumerate}
  % Problem 5
  \item[5.] Use the principle of mathematical induction to prove $4^{n+4} > (n+4)^4,$ for all natural numbers $n$.
\end{enumerate}

% Proof 5
\begin{proof}
  Let $n=1$:
  $$4^5>5^4$$
  so the base case is true.
  Then, let $n=k$:
  \setcounter{equation}{0}
  \begin{equation}
    4^{k+4} > (k+4)^4.
  \end{equation}
    Since $k\in\N$, then $(k+1)\in\N$ as well. As such, when $n=k+1$:
  \begin{equation}
    4^{k+5} > (k+5)^4.
  \end{equation}
  To prove that this inequality holds for all natural numbers, we will establish bounds using Inequality (1). Equation (3) shows the left hand side of Inequality (2) re-written in expanded form.
  \begin{equation}
    4^{k+5} = 4^k*4^5
  \end{equation}
  With this knowledge, we can see that:
  \begin{equation}
    4^{k+5} = 4^k*4^5 = 4(4^{k+4}).
  \end{equation}
  which we can compare directly to the left hand side of Inequality (1)
  \begin{equation*}
    4(4^{k+4}) > 4^{k+4}
  \end{equation*}
  and find that it is greater by four times. In terms of building a series of inequalities, we now have:
  \begin{equation}
    4^{k+5} = 4(4^{k+4}) > 4^{k+4}
  \end{equation}
  Now, let us look at the right hand side of Inequalities (1) and (2):
  \begin{equation}
    (k+4)^4 = k^4 + 16k^3 + 96k^2 + 256k + 256
  \end{equation}
  \begin{equation}
    (k+5)^4 = k^4 + 20k^3 + 150k^2 + 500k + 625.
  \end{equation}
  Referring to Equation (4), where we used four times the left hand side of Inequality (2), we now use four times the right hand side of Inequality (1) to keep the relationship that the two share:
  \begin{equation}
    4(4^{k+4}) > 4(k+4)^4.
  \end{equation}
  From this point, it is important to note the distributed form of $4(k+4)^4$ is greater than $(k+5)^4$ (given of course that $k\in\N$):
  \begin{equation}
    4(k+4)^4 = 4k^4 + 64k^3 + 384k^2 + 1024k + 1024 > k^4 + 20k^3 + 150k^2 + 500k + 625 = (k+5)^4.
  \end{equation}
  Therefore
  \begin{equation}
    4(k+4)^4 > (k+5)^4.
  \end{equation}
  Finally, we build the last inequality, taking parts of Equation (5), Inequality (8), and Inequality (10):
  \begin{equation}
    4^{k+5} = 4(4^{k+4})>4(k+4)^4 > (k+5)^4
  \end{equation}

  Then, by the principle of mathematical induction, $4^{n+4} > (n+4)^4$ $\forall n\in\N$.
\end{proof}
% End problem set

\end{document} % End document
